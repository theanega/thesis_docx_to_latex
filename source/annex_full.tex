% Options for packages loaded elsewhere
\PassOptionsToPackage{unicode}{hyperref}
\PassOptionsToPackage{hyphens}{url}
\documentclass[
]{book}
\usepackage{xcolor}
\usepackage{amsmath,amssymb}
\setcounter{secnumdepth}{-\maxdimen} % remove section numbering
\usepackage{iftex}
\ifPDFTeX
  \usepackage[T1]{fontenc}
  \usepackage[utf8]{inputenc}
  \usepackage{textcomp} % provide euro and other symbols
\else % if luatex or xetex
  \usepackage{unicode-math} % this also loads fontspec
  \defaultfontfeatures{Scale=MatchLowercase}
  \defaultfontfeatures[\rmfamily]{Ligatures=TeX,Scale=1}
\fi
\usepackage{lmodern}
\ifPDFTeX\else
  % xetex/luatex font selection
\fi
% Use upquote if available, for straight quotes in verbatim environments
\IfFileExists{upquote.sty}{\usepackage{upquote}}{}
\IfFileExists{microtype.sty}{% use microtype if available
  \usepackage[]{microtype}
  \UseMicrotypeSet[protrusion]{basicmath} % disable protrusion for tt fonts
}{}
\makeatletter
\@ifundefined{KOMAClassName}{% if non-KOMA class
  \IfFileExists{parskip.sty}{%
    \usepackage{parskip}
  }{% else
    \setlength{\parindent}{0pt}
    \setlength{\parskip}{6pt plus 2pt minus 1pt}}
}{% if KOMA class
  \KOMAoptions{parskip=half}}
\makeatother
\usepackage{longtable,booktabs,array}
\newcounter{none} % for unnumbered tables
\usepackage{multirow}
\usepackage{calc} % for calculating minipage widths
% Correct order of tables after \paragraph or \subparagraph
\usepackage{etoolbox}
\makeatletter
\patchcmd\longtable{\par}{\if@noskipsec\mbox{}\fi\par}{}{}
\makeatother
% Allow footnotes in longtable head/foot
\IfFileExists{footnotehyper.sty}{\usepackage{footnotehyper}}{\usepackage{footnote}}
\makesavenoteenv{longtable}
\usepackage{graphicx}
\makeatletter
\newsavebox\pandoc@box
\newcommand*\pandocbounded[1]{% scales image to fit in text height/width
  \sbox\pandoc@box{#1}%
  \Gscale@div\@tempa{\textheight}{\dimexpr\ht\pandoc@box+\dp\pandoc@box\relax}%
  \Gscale@div\@tempb{\linewidth}{\wd\pandoc@box}%
  \ifdim\@tempb\p@<\@tempa\p@\let\@tempa\@tempb\fi% select the smaller of both
  \ifdim\@tempa\p@<\p@\scalebox{\@tempa}{\usebox\pandoc@box}%
  \else\usebox{\pandoc@box}%
  \fi%
}
% Set default figure placement to htbp
\def\fps@figure{htbp}
\makeatother
\usepackage{svg}
\setlength{\emergencystretch}{3em} % prevent overfull lines
\providecommand{\tightlist}{%
  \setlength{\itemsep}{0pt}\setlength{\parskip}{0pt}}
\usepackage{bookmark}
\IfFileExists{xurl.sty}{\usepackage{xurl}}{} % add URL line breaks if available
\urlstyle{same}
\hypersetup{
  hidelinks,
  pdfcreator={LaTeX via pandoc}}

\author{}
\date{}

\begin{document}
\frontmatter

\mainmatter
Contents

\hyperref[image-aquisition-details]{Image Aquisition Details
\hyperref[image-aquisition-details]{2}}

\hyperref[identification-of-precise-handcrafted-features-for-habitat-imaging]{Identification
of Precise Handcrafted Features for Habitat Imaging
\hyperref[identification-of-precise-handcrafted-features-for-habitat-imaging]{5}}

\hyperref[development-and-validation-of-biologically-informed-ct-habitats]{Development
and Validation of Biologically-Informed CT Habitats
\hyperref[development-and-validation-of-biologically-informed-ct-habitats]{16}}

\hyperref[clinical-relevance-of-ct-habitats]{Clinical Relevance of CT
Habitats \hyperref[clinical-relevance-of-ct-habitats]{24}}

Annex A

\chapter{Image Aquisition Details}\label{image-aquisition-details}

\subsection{A.1. MRI Acquisition Details
(PREDICT)}\label{a.1.-mri-acquisition-details-predict}

Patients were scanned on either a 1.5T Siemens Avanto or a 3T GE SIGNA
Pioneer system. Each patient was scanned on only one of the two
scanners. The protocol included anatomical imaging (T2-weighted and
T1-weighted), diffusion MRI, variable flip angle spoiled gradient echo
(SGrE) imaging for T1 mapping, and dynamic contrast-enhanced (DCE) MRI.

\subsubsection{\texorpdfstring{\textbf{1.5T Siemens Avanto
system}}{1.5T Siemens Avanto system}}\label{t-siemens-avanto-system}

The protocol included high-resolution anatomical T2w and T1w scans,
diffusion MRI and different spoiled gradient echo (SGrE) sequences, such
as those for T1 mapping and dynamic contrast enhanced (DCE) MRI.

\begin{itemize}
\item
  Anatomical T2w scan: turbo spin echo, TE = 82 ms, TR = 4500 ms, turbo
  factor of 29, echo spacing 8.2 ms, NEX = 8, 2 concatenations,
  resolution of 1.4mm × 1.4mm, slice thickness of 5 mm, GRAPPA = 2,
  acquisition in free breathing.
\item
  Anatomical T1w scan: turbo spin echo, TE = 6.3 ms, TR = 470 ms, turbo
  factor of 11, echo spacing 6.26 ms, NEX = 6, 6 concatenations,
  resolution of 1.4mm × 1.4mm, slice thickness of 5 mm, GRAPPA = 2,
  acquisition in free breathing.
\item
  Diffusion MRI: single-shot twice-refocused spin echo EPI, b = \{0, 50,
  100, 400, 900, 1200, 1600\} s/mm\textsuperscript{2}, TR = 7900 ms,
  averaging of 3 mutually-orthogonal directions, NEX = 2, 1
  concatenation, resolution of 1.9mm × 1.9mm, slice thickness of 6 mm,
  SPAIR fat suppression, GRAPPA = 2, EPI factor 150, echo spacing 0.82
  ms, each b-value acquired at TE = \{93 ms, 105 ms, 120 ms\},
  acquisition in free breathing. Additionally, one b = 0 image at TE =
  93 ms was acquired with reversed phase encoding polarity.
\item
  SGrE for T1 mapping: FLASH, TE = 1.76 ms, TR = 4.59 ms, NEX = 1, 1
  concatenation, resolution of 2.7mm × 2.7mm, slice thickness of 6 mm,
  flip angles of \{5°, 15°, 20°\}, GRAPPA = 2, acquisition in free
  breathing.
\item
  SGrE for DCE: same acquisition as for T1 mapping with fixed flip angle
  of 15°; 26 dynamic acquisitions with temporal resolution of 10s,
  Gadovist with dose of 0.5ml/Kg injected at 3ml/s followed by a bolus
  of physiological solution of 20ml at 3ml/s, injection delay of 10s,
  acquisition in free breathing.
\end{itemize}

\subsubsection{\texorpdfstring{\textbf{3T GE SIGNA Pioneer
system}}{3T GE SIGNA Pioneer system}}\label{t-ge-signa-pioneer-system}

The protocol included high-resolution anatomical T2w and T1w scans,
diffusion MRI and different spoiled gradient echo (SGrE) sequences, as
those for T1 mapping and DCE imaging.

\begin{itemize}
\item
  Anatomical T2w scan: fast spin echo, TE = 50 ms, TR = 3750 ms, turbo
  factor of 16, NEX = 2, resolution of 1.4mm × 1.4mm, slice thickness of
  6 mm, respiratory-gated acquisition.
\item
  Anatomical T1w scan: navigated SGrE LAVA-Flex providing water/fat
  images, TE = 2.60 ms, TR = 5.38 ms, resolution of 1.4mm × 1.4mm, slice
  thickness of 6 mm, flip angle of 12°, acquisition in free-breathing
  after liver motion measurement.
\item
  Diffusion MRI: single-shot pulsed gradient spin echo EPI, b = \{0, 50,
  100, 400, 900, 1200, 1500\} s/mm\textsuperscript{2}, TR = 3500ms,
  averaging of 3 mutually-orthogonal directions, NEX = 2, resolution of
  2.4mm × 2.4mm, slice thickness of 6 mm, ASPIR fat suppression, ASSET =
  2, echo spacing 0.80 ms, each b-value acquired at TE = \{75 ms, 90 ms,
  105 ms\}, respiratory-gated acquisition.
\item
  SGrE for T1 mapping: LAVA, TE = 1.2 ms, TR = 2.72 ms, NEX = 1,
  resolution of 2.4mm × 2.4mm; slice thickness of 6 mm; flip angles of
  \{5°, 10°, 15°\}, ASSET = 2, acquisition of two separate images in
  breathhold, acquisition of the vendor's B1 map.
\item
  SGrE for DCE: same acquisition as for T1 mapping with fixed flip angle
  of 12°; 26 dynamic acquisitions with temporal resolution of 10s,
  Clariscan 0.5 mmol/ml with dose of 0.2ml/Kg injected at of 0.5ml/kg at
  3ml/s followed by a bolus of physiological solution of 20ml at 3ml/s,
  injection delay of 10s, acquisition in free breathing.
\end{itemize}

\subsection{A.2. mpMRI Maps Biological
Ranges}\label{a.2.-mpmri-maps-biological-ranges}

\textbf{Table A.1. mpMRI maps dervied with biological ranges.} Thirteen
quantitative maps were derived from diffusion-relaxation MRI, variable
flip angle T1 mapping, and dynamic contrast-enhanced MRI.

{\def\LTcaptype{none} % do not increment counter
\begin{longtable}[]{@{}
  >{\centering\arraybackslash}p{(\linewidth - 8\tabcolsep) * \real{0.0651}}
  >{\centering\arraybackslash}p{(\linewidth - 8\tabcolsep) * \real{0.2603}}
  >{\centering\arraybackslash}p{(\linewidth - 8\tabcolsep) * \real{0.1544}}
  >{\centering\arraybackslash}p{(\linewidth - 8\tabcolsep) * \real{0.2124}}
  >{\centering\arraybackslash}p{(\linewidth - 8\tabcolsep) * \real{0.2967}}@{}}
\toprule\noalign{}
\multicolumn{2}{@{}>{\centering\arraybackslash}p{(\linewidth - 8\tabcolsep) * \real{0.3254} + 2\tabcolsep}}{%
\begin{minipage}[b]{\linewidth}\centering
\textbf{mpMRI metric}
\end{minipage}} & \begin{minipage}[b]{\linewidth}\centering
\textbf{Computed from}
\end{minipage} & \begin{minipage}[b]{\linewidth}\centering
\textbf{Units}
\end{minipage} & \begin{minipage}[b]{\linewidth}\centering
\textbf{Biological Range}
\end{minipage} \\
\midrule\noalign{}
\endhead
\bottomrule\noalign{}
\endlastfoot
ADCₜ & Tissue ADC & Diffusion-relaxation MRI & μm²/ms & 0.0-3.0 \\
ADCᵥ & Vascular ADC & Diffusion-relaxation MRI & μm²/ms & 3.0-150.0 \\
Kₜ & Tissue kurtosis excess & Diffusion-relaxation MRI & Dimensionless &
0.0-5.0 \\
fᵥ & Vascular signal fraction & Diffusion-relaxation MRI & Normalized &
0.0-1.0 \\
T₂ₜ & Tissue T₂ & Diffusion-relaxation MRI & ms & 0.0-800.0 \\
D₀ & Intrinsic diffusivity & Advanced diffusion model & μm²/ms &
0.8-3.0 \\
vCS & Volume-weighted cell size & Advanced diffusion model & μm &
5.0-40.0 \\
fᵢₙ & Intracellular fraction & Advanced diffusion model & Normalized &
0.0-1.0 \\
CD & Cell density & Advanced diffusion model & Cells/mm³ &
0.0-8000000.0 \\
T₁ & T₁ & Variable flip angle SGrE & ms & 0.0-5000.0 \\
T2\textsuperscript{*} & T2\textsuperscript{*} & Multiecho SGrE & ms &
0.0-200.0 \\
K\textsuperscript{trans} & Capillary permeability & DCE MRI & min⁻¹ &
0.0-4.0 \\
vₑ & Extracellular-extravascular volume & DCE MRI & Normalized &
0.0-1.0 \\
\end{longtable}
}

Annex B

\chapter{Identification of Precise Handcrafted Features for Habitat
Imaging}\label{identification-of-precise-handcrafted-features-for-habitat-imaging}

\subsection{B.1. Cohort
Characteristics}\label{b.1.-cohort-characteristics}

\textbf{Table B.1. Total number of patients, images, and lesions per
cohort and lesion location.}

{\def\LTcaptype{none} % do not increment counter
\begin{longtable}[]{@{}
  >{\raggedright\arraybackslash}p{(\linewidth - 12\tabcolsep) * \real{0.1976}}
  >{\raggedright\arraybackslash}p{(\linewidth - 12\tabcolsep) * \real{0.1330}}
  >{\raggedright\arraybackslash}p{(\linewidth - 12\tabcolsep) * \real{0.1327}}
  >{\raggedright\arraybackslash}p{(\linewidth - 12\tabcolsep) * \real{0.1331}}
  >{\raggedright\arraybackslash}p{(\linewidth - 12\tabcolsep) * \real{0.1327}}
  >{\raggedright\arraybackslash}p{(\linewidth - 12\tabcolsep) * \real{0.1331}}
  >{\raggedright\arraybackslash}p{(\linewidth - 12\tabcolsep) * \real{0.1327}}@{}}
\toprule\noalign{}
\multirow{2}{=}{\begin{minipage}[b]{\linewidth}\raggedright
\textbf{Primary tumor}
\end{minipage}} &
\multicolumn{2}{>{\raggedright\arraybackslash}p{(\linewidth - 12\tabcolsep) * \real{0.2657} + 2\tabcolsep}}{%
\begin{minipage}[b]{\linewidth}\raggedright
\textbf{Patients}
\end{minipage}} &
\multicolumn{2}{>{\raggedright\arraybackslash}p{(\linewidth - 12\tabcolsep) * \real{0.2658} + 2\tabcolsep}}{%
\begin{minipage}[b]{\linewidth}\raggedright
\textbf{Images}
\end{minipage}} &
\multicolumn{2}{>{\raggedright\arraybackslash}p{(\linewidth - 12\tabcolsep) * \real{0.2658} + 2\tabcolsep}@{}}{%
\begin{minipage}[b]{\linewidth}\raggedright
\textbf{Lesions}
\end{minipage}} \\
& \begin{minipage}[b]{\linewidth}\raggedright
Liver
\end{minipage} & \begin{minipage}[b]{\linewidth}\raggedright
Lung
\end{minipage} & \begin{minipage}[b]{\linewidth}\raggedright
Liver
\end{minipage} & \begin{minipage}[b]{\linewidth}\raggedright
Lung
\end{minipage} & \begin{minipage}[b]{\linewidth}\raggedright
Liver
\end{minipage} & \begin{minipage}[b]{\linewidth}\raggedright
Lung
\end{minipage} \\
\midrule\noalign{}
\endhead
\bottomrule\noalign{}
\endlastfoot
Colorectal & 63 & 12 & 186 & 29 & 959 & 122 \\
Lung & 13 & 72 & 22 & 102 & 89 & 141 \\
Neuroendocrine & 86 & 0 & 86 & 0 & 447 & 0 \\
Mixed & 44 & 41 & 93 & 87 & 366 & 312 \\
\multirow{2}{=}{Total} & 206 & 125 & 387 & 218 & 1861 & 575 \\
&
\multicolumn{2}{>{\raggedright\arraybackslash}p{(\linewidth - 12\tabcolsep) * \real{0.2657} + 2\tabcolsep}}{%
331} &
\multicolumn{2}{>{\raggedright\arraybackslash}p{(\linewidth - 12\tabcolsep) * \real{0.2658} + 2\tabcolsep}}{%
605} &
\multicolumn{2}{>{\raggedright\arraybackslash}p{(\linewidth - 12\tabcolsep) * \real{0.2658} + 2\tabcolsep}@{}}{%
2436} \\
\end{longtable}
}

\textbf{Table B.2. Image acquisition parameters per cohort.}

{\def\LTcaptype{none} % do not increment counter
\begin{longtable}[]{@{}
  >{\raggedright\arraybackslash}p{(\linewidth - 8\tabcolsep) * \real{0.2898}}
  >{\raggedright\arraybackslash}p{(\linewidth - 8\tabcolsep) * \real{0.1927}}
  >{\raggedright\arraybackslash}p{(\linewidth - 8\tabcolsep) * \real{0.1549}}
  >{\raggedright\arraybackslash}p{(\linewidth - 8\tabcolsep) * \real{0.1886}}
  >{\raggedright\arraybackslash}p{(\linewidth - 8\tabcolsep) * \real{0.1739}}@{}}
\toprule\noalign{}
\begin{minipage}[b]{\linewidth}\raggedright
\end{minipage} & \begin{minipage}[b]{\linewidth}\raggedright
\textbf{Colorectal}

\textbf{(n=215)}
\end{minipage} & \begin{minipage}[b]{\linewidth}\raggedright
\textbf{Lung}

\textbf{(n=124)}
\end{minipage} & \begin{minipage}[b]{\linewidth}\raggedright
\textbf{Neuroendocrine}

\textbf{(n=86)}
\end{minipage} & \begin{minipage}[b]{\linewidth}\raggedright
\textbf{Mixed}

\textbf{(n=180)}
\end{minipage} \\
\midrule\noalign{}
\endhead
\bottomrule\noalign{}
\endlastfoot
\textbf{Manufacturers}

SIEMENS/PHILIPS/

TOSHIBA/GENERAL ELECTRIC & 138/58/9/10 & 63/40/0/21 & 22/35/6/23 &
144/23/3/10 \\
\textbf{Tube Voltage (kVP)}

100/110/120/130/140/unknown & 25/17/161/0/0/12 & 6/1/105/3/0/9 &
10/3/70/1/2/0 & 14/7/158/0/1/0 \\
\textbf{Reconstruction kernel}

SOFT/STANDARD/B

B20f/B30f/B31f

I31s/I50s/unkown & 1/7/96

16/38/12

15/6/20 & 14/7/43

11/23/0

0/0/22 & 1/19/36

2/2/4

0/0/19 & 2/11/21

9/112/4

0/0/21 \\
\textbf{Slice thickness (mm)*} & 2.0 {[}2.00-5.00{]} & 2.5 {[}2.0-5.0{]}
& 2.0 {[}2.0-3.0{]} & 5.0 {[}1.00-5.00{]} \\
\textbf{Pixel spacing (mm)*} & 0.92 {[}0.77-0.98{]} & 0.91
{[}0.81-0.98{]} & 0.75 {[}0.70-0.82{]} & 0.98 {[}0.82-0.98{]} \\
\multicolumn{5}{@{}>{\raggedright\arraybackslash}p{(\linewidth - 8\tabcolsep) * \real{1.0000} + 8\tabcolsep}@{}}{%
(*) Median {[}IQR{]}} \\
\end{longtable}
}

\textbf{Table B.3. List of primary tumor types included within the mixed
cohort.}

{\def\LTcaptype{none} % do not increment counter
\begin{longtable}[]{@{}
  >{\raggedright\arraybackslash}p{(\linewidth - 12\tabcolsep) * \real{0.1457}}
  >{\raggedright\arraybackslash}p{(\linewidth - 12\tabcolsep) * \real{0.1425}}
  >{\raggedright\arraybackslash}p{(\linewidth - 12\tabcolsep) * \real{0.1423}}
  >{\raggedright\arraybackslash}p{(\linewidth - 12\tabcolsep) * \real{0.1424}}
  >{\raggedright\arraybackslash}p{(\linewidth - 12\tabcolsep) * \real{0.1423}}
  >{\raggedright\arraybackslash}p{(\linewidth - 12\tabcolsep) * \real{0.1424}}
  >{\raggedright\arraybackslash}p{(\linewidth - 12\tabcolsep) * \real{0.1423}}@{}}
\toprule\noalign{}
\multirow{2}{=}{\begin{minipage}[b]{\linewidth}\raggedright
\textbf{Primary tumor}
\end{minipage}} &
\multicolumn{2}{>{\raggedright\arraybackslash}p{(\linewidth - 12\tabcolsep) * \real{0.2848} + 2\tabcolsep}}{%
\begin{minipage}[b]{\linewidth}\raggedright
\textbf{Patients}
\end{minipage}} &
\multicolumn{2}{>{\raggedright\arraybackslash}p{(\linewidth - 12\tabcolsep) * \real{0.2847} + 2\tabcolsep}}{%
\begin{minipage}[b]{\linewidth}\raggedright
\textbf{Images}
\end{minipage}} &
\multicolumn{2}{>{\raggedright\arraybackslash}p{(\linewidth - 12\tabcolsep) * \real{0.2847} + 2\tabcolsep}@{}}{%
\begin{minipage}[b]{\linewidth}\raggedright
\textbf{Lesions}
\end{minipage}} \\
& \begin{minipage}[b]{\linewidth}\raggedright
\textbf{Liver}
\end{minipage} & \begin{minipage}[b]{\linewidth}\raggedright
\textbf{Lung}
\end{minipage} & \begin{minipage}[b]{\linewidth}\raggedright
\textbf{Liver}
\end{minipage} & \begin{minipage}[b]{\linewidth}\raggedright
\textbf{Lung}
\end{minipage} & \begin{minipage}[b]{\linewidth}\raggedright
\textbf{Liver}
\end{minipage} & \begin{minipage}[b]{\linewidth}\raggedright
\textbf{Lung}
\end{minipage} \\
\midrule\noalign{}
\endhead
\bottomrule\noalign{}
\endlastfoot
Adrenal & 3 & 0 & 5 & 0 & 35 & 5 \\
Biliary Tract & 11 & 5 & 24 & 11 & 71 & 66 \\
Bladder & 3 & 3 & 5 & 5 & 41 & 20 \\
Bone & 0 & 1 & 0 & 3 & 0 & 21 \\
Breast & 4 & 2 & 9 & 6 & 32 & 6 \\
Cervix & 2 & 2 & 3 & 3 & 21 & 21 \\
Esophagus & 1 & 2 & 2 & 4 & 8 & 9 \\
Head\&Neck & 2 & 4 & 4 & 11 & 13 & 13 \\
Kidney & 1 & 2 & 2 & 3 & 4 & 18 \\
Liver & 2 & 1 & 3 & 1 & 11 & 1 \\
Ovary & 1 & 2 & 2 & 4 & 4 & 26 \\
Pancreas & 2 & 0 & 4 & 0 & 12 & 0 \\
Penis & 1 & 0 & 3 & 0 & 9 & 0 \\
Skin & 6 & 11 & 13 & 19 & 68 & 39 \\
Stomach & 4 & 0 & 10 & 0 & 29 & 0 \\
Thymus & 1 & 0 & 4 & 0 & 8 & 0 \\
Thyroid & 0 & 6 & 0 & 17 & 0 & 52 \\
Total & 44 & 41 & 93 & 87 & 366 & 312 \\
&
\multicolumn{2}{>{\raggedright\arraybackslash}p{(\linewidth - 12\tabcolsep) * \real{0.2848} + 2\tabcolsep}}{%
85} &
\multicolumn{2}{>{\raggedright\arraybackslash}p{(\linewidth - 12\tabcolsep) * \real{0.2847} + 2\tabcolsep}}{%
180} &
\multicolumn{2}{>{\raggedright\arraybackslash}p{(\linewidth - 12\tabcolsep) * \real{0.2847} + 2\tabcolsep}@{}}{%
678} \\
\end{longtable}
}

\subsection{B.2. Radiomics Features and
Computation}\label{b.2.-radiomics-features-and-computation}

\textbf{Table B.4. List of Radiomics features analyzed in this study.}

\begin{longtable}[]{@{}
  >{\raggedright\arraybackslash}p{(\linewidth - 6\tabcolsep) * \real{0.1042}}
  >{\raggedright\arraybackslash}p{(\linewidth - 6\tabcolsep) * \real{0.3432}}
  >{\raggedright\arraybackslash}p{(\linewidth - 6\tabcolsep) * \real{0.1072}}
  >{\raggedright\arraybackslash}p{(\linewidth - 6\tabcolsep) * \real{0.4344}}@{}}
\caption{93 voxel-wise features were computed. However, 91 were analyzed
after excluding GLCM\_MCC (for having missing values in many cases due
to memory error) and FirstOrder\_TotalEnergy (for being equal to
FirstOrder\_Energy for constant kernel size during computation). Feature
definitions are available in the IBSI reference manual (Zwanenburg et
al., 2020).}\tabularnewline
\toprule\noalign{}
\begin{minipage}[b]{\linewidth}\raggedright
\textbf{Class}
\end{minipage} & \begin{minipage}[b]{\linewidth}\raggedright
\textbf{Feature}
\end{minipage} & \begin{minipage}[b]{\linewidth}\raggedright
\textbf{Class}
\end{minipage} & \begin{minipage}[b]{\linewidth}\raggedright
\textbf{Feature}
\end{minipage} \\
\midrule\noalign{}
\endfirsthead
\toprule\noalign{}
\begin{minipage}[b]{\linewidth}\raggedright
\textbf{Class}
\end{minipage} & \begin{minipage}[b]{\linewidth}\raggedright
\textbf{Feature}
\end{minipage} & \begin{minipage}[b]{\linewidth}\raggedright
\textbf{Class}
\end{minipage} & \begin{minipage}[b]{\linewidth}\raggedright
\textbf{Feature}
\end{minipage} \\
\midrule\noalign{}
\endhead
\bottomrule\noalign{}
\endlastfoot
\multirow{2}{=}{\textbf{First order}} & \multirow{2}{=}{10Percentile

90Percentile

Energy

Entropy

InterquartileRange

Kurtosis

Maximum

MeanAbsoluteDeviation

Mean

Median

Minimum

Range

RobustMeanAbsoluteDeviation

RootMeanSquared

Skewness

TotalEnergy

Uniformity

Variance} & \textbf{GLRLM} & GrayLevelNonUniformity

GrayLevelNonUniformityNormalized

GrayLevelVariance

HighGrayLevelRunEmphasis

LongRunEmphasis

LongRunHighGrayLevelEmphasis

LongRunLowGrayLevelEmphasis

LowGrayLevelRunEmphasis

RunEntropy

RunLengthNonUniformity

RunLengthNonUniformityNormalized

RunPercentage

RunVariance

ShortRunEmphasis

ShortRunHighGrayLevelEmphasis

ShortRunLowGrayLevelEmphasis \\
& & \multirow{2}{=}{\textbf{GLSZM}} &
\multirow{2}{=}{GrayLevelNonUniformity

GrayLevelNonUniformityNormalized

GrayLevelVariance

HighGrayLevelZoneEmphasis

LargeAreaEmphasis

LargeAreaHighGrayLevelEmphasis

LargeAreaLowGrayLevelEmphasis

LowGrayLevelZoneEmphasis

SizeZoneNonUniformity

SizeZoneNonUniformityNormalized

SmallAreaEmphasis

SmallAreaHighGrayLevelEmphasis

SmallAreaLowGrayLevelEmphasis

ZoneEntropy

ZonePercentage

ZoneVariance} \\
\multirow{2}{=}{\textbf{GLCM}} & \multirow{2}{=}{Autocorrelation

ClusterProminence

ClusterShade

ClusterTendency

Contrast

Correlation

DifferenceAverage

DifferenceEntropy

DifferenceVariance

Id

Idm

Idmn

Idn

Imc1

Imc2

InverseVariance

JointAverage

JointEnergy

JointEntropy

MCC

MaximumProbability

SumAverage

SumEntropy

SumSquares} \\
& & \multirow{2}{=}{\textbf{GLDM}} & \multirow{2}{=}{DependenceEntropy

DependenceNonUniformity

DependenceNonUniformityNormalized

DependenceVariance

GrayLevelNonUniformity

GrayLevelVariance

HighGrayLevelEmphasis

LargeDependenceEmphasis

LargeDependenceHighGrayLevelEmphasis

LargeDependenceLowGrayLevelEmphasis

LowGrayLevelEmphasis

SmallDependenceEmphasis

SmallDependenceHighGrayLevelEmphasis

SmallDependenceLowGrayLevelEmphasis} \\
\textbf{NGTDM} & Busyness

Coarseness

Complexity

Contrast

Strength \\
\end{longtable}

Table B.5. Image processing and radiomics feature computation
parameters.

{\def\LTcaptype{none} % do not increment counter
\begin{longtable}[]{@{}
  >{\raggedright\arraybackslash}p{(\linewidth - 2\tabcolsep) * \real{0.4819}}
  >{\raggedright\arraybackslash}p{(\linewidth - 2\tabcolsep) * \real{0.5181}}@{}}
\toprule\noalign{}
\multicolumn{2}{@{}>{\raggedright\arraybackslash}p{(\linewidth - 2\tabcolsep) * \real{1.0000} + 2\tabcolsep}@{}}{%
\begin{minipage}[b]{\linewidth}\raggedright
\textbf{Image Processing}
\end{minipage}} \\
\midrule\noalign{}
\endhead
\bottomrule\noalign{}
\endlastfoot
Software & PyRadiomics v3.0.1, installed in Python 3.7.10 \\
Bounding box & Defined by the segmentation, extended by default padding
distance. \\
Resampled voxel spacing (mm) & 1 x 1 x 1 \\
Image interpolation method & B-spline \\
Intensity rounding & None \\
ROI interpolation method & Nearest neighbor \\
Resegmentation & None \\
\multicolumn{2}{@{}>{\raggedright\arraybackslash}p{(\linewidth - 2\tabcolsep) * \real{1.0000} + 2\tabcolsep}@{}}{%
\textbf{Feature Computation}} \\
Kernel radius & 1 / 3mm \\
Discretization (fixed bin size) & 12/25HU \\
Image filter & None \\
maskedKernel & True (only voxels in the kernel that were also segmented
in the ROI were used for calculation) \\
Initvalue & NaN (voxels outside ROI were considered as transparent) \\
Distance weighting for GLCM, GLRLM, NGTDM & No weighting \\
GLCM Symmetry & Symmetric \\
GLCM distance, GLSZM linkage distance, GLDZM linkage distance, NGTDM
distance & Chebyshev distance δ = 1 \\
NGTDM coarseness & Coarseness parameter α = 0 \\
\end{longtable}
}

\subsection{B.3. Image Perturbation}\label{b.3.-image-perturbation}

Image perturbation was carried out in three ways: rotation, translation,
and noise addition. While the first two emulate changes in patient
positioning, the latter represents the noise present in different voxel
intensities in CT images. Perturbations were performed as described in
(Bernatowicz et al., 2021), where the authors demonstrated that the
combination of these three perturbations simulate the retest scenario.
Briefly, we added Gaussian noise (mean 0, standard deviation as present
in the image); for translation, we shifted the voxel grid by a fraction
of the image voxel spacing following; finally, we rotated the image
around the z-axis by an angle of 0.5°.

\subsection{B.4. Habitat Computation}\label{b.4.-habitat-computation}

To take into account intravoxel heterogeneity, we decided to choose a
probabilistic model, Gaussian Mixture Models (GMMs), for clustering
rather than a deterministic approach. GMMs, which have been previously
used in similar contexts (J. Chen et al., 2019; Jardim-Perassi et al.,
2019), are generative probabilistic models that find a mixture of
multiple Gaussian probability distributions that best fit the data. The
Expectation-Maximization (EM) algorithm is used to estimate the model
parameters (Bishop, 2006). A GMM is represented by the following
formula:

\[P\ (x) = \sum_{}^{}(\pi_{k}\ N\ (x\ |\ \mu_{k},\ \sum_{k})\]

where

P(x) : probability density of the data point x

\(\mathbf{\pi}_{\mathbf{k}}\): mixing coefficient for the kth Gaussian
component

\(\mathbf{N}\ (\mathbf{x}\ |\ \mathbf{\mu}_{\mathbf{k}},\ \sum_{\mathbf{k}})\):
kth Gaussian component with mean \(\mathbf{\mu}_{\mathbf{k}}\) and
covariance matrix \(\sum_{\mathbf{k}}\)

To determine the optimal number of habitats (k), we used the Bayesian
Information Criterion (BIC). The formula for BIC is:

\[BIC = \ {- \ 2\ log}{(L)\  + \ d\ log(n)}\]

where

L : likelihood of the data given the model

d: number of parameters

n: number of data points

The BIC score is a measure of the trade-off between model complexity and
goodness of fit. It penalizes models with more parameters, such as GMMs
with more clusters. In general, lower BIC scores indicate better model
fit. However, depending on data characteristics, a clear minimum in BIC
scores might not be observed and thus, the gradient can be used to
determine the optimal number of clusters. This was our case and
therefore we performed a GMM fit for different values of clusters (k):
\{2, 3, 4 and 5\}. The maximum number of 5 clusters was determined by
being the maximum number of tissue types observed in histology by an
experienced pathologist. The optimal value of k was the one where the
change in BIC score with respect to k was maximal, which was an
indication that adding more clusters after that point does not improve
the model fit significantly. A cluster number was automatically selected
by BIC using the precise original radiomics data and was given as a
parameter to the GMM model to compute imaging habitats in both the
original and perturbed data. GMM was implemented using Python package
scikit-learn (v1.0.2) with a random seed of 123, and default parameters
(except for the number of clusters), specifically maximum iteration of
100, convergence threshold of 10\textsuperscript{-3}, full covariance
type and initialization with kmeans.

In addition, The Hungarian algorithm (also known as the Kuhn-Munkres
algorithm) (Kuhn, 1955), was used to match habitats between original and
perturbed data. The Hungarian algorithm is a combinatorial optimization
algorithm that solves the assignment problem in polynomial time. It
finds an optimal one-to-one matching between two sets by minimizing the
total cost (in our case, the difference in cluster assignments).

Finally, to quantify habitat stability, we computed the Dice Similarity
Coefficient (DSC) (Zou et al., 2004) between original and perturbed
habitats for each habitat within a lesion, across all lesions. The DSC
is a widely used metric for evaluating the overlap between two sets,
with a higher DSC indicating greater similarity.

All codes are publicly available at
\url{https://github.com/radiomicsgroup/precise-habitats}.

\subsection{B.5. Intraclass Correlation
Coefficient}\label{b.5.-intraclass-correlation-coefficient}

An Intraclass Correlation Coefficient (ICC) value of 1 indicates that a
feature is highly repeatable/reproducible whereas a value of 0 implies
no reliability. Negative ICC values were truncated at 0 as proposed and
done previously (Bartko, 1976; Fornacon-Wood et al., 2020). The ICC is
calculated by mean squares obtained through the analysis of variance
(ANOVA). In this study, we use two versions of the ICC that are based on
a two-way mixed effects ANOVA model, following Koo's guidelines. Below
we describe the formulas used to compute the ICC formulas. More
information regarding such formulas can be found in the highly cited
paper from McGraw and Wong.

To compute the ANOVA model let's consider a dataframe with dimensions
\(\mathbf{n}\  \times \mathbf{k}\) dataframe where \(\mathbf{n}\) is the
total number of voxels (rows) for one region of interest (ROI) and
\(\mathbf{k}\) is the total number of conditions or measurements
(columns). In our case, \(\mathbf{k}\)=2. For repeatability the two
conditions are original-perturbed (test-retest) and for reproducibility
against kernel size the two conditions are computation with radius
kernel 1mm or radius kernel 3mm (or bin size 12HU or 25HU in the case of
reproducibility against bin size). Each voxel measurement is indexed as
\(\mathbf{Y}_{\mathbf{ij}}\) where \emph{i} denotes the voxel (\emph{i}
= 1, \ldots{} \emph{n)} and \emph{j} denotes the measurement under the
repeatability/reproducibility condition (\emph{j = 1 \ldots{} k)}. We
define the following concepts:

\({\overline{\mathbf{Y}}}_{\mathbf{i}}\): mean of all voxel values in a
column

\[{\overline{\mathbf{Y}}}_{\mathbf{i}} = \frac{\sum_{\mathbf{j} = \mathbf{1}}^{\mathbf{k}}\mathbf{Y}_{\mathbf{ij}}}{\mathbf{k}}\]

\({\overline{\mathbf{Y}}}_{\mathbf{j}}:\) mean of all voxel values in a
column

\[{\overline{\mathbf{Y}}}_{\mathbf{j}} = \frac{\sum_{\mathbf{i} = \mathbf{1}}^{\mathbf{n}}\mathbf{Y}_{\mathbf{ij}}}{\mathbf{n}}\]

\(\mathbf{\mu}:\) mean of all values (also called \emph{grand mean})

\[\mathbf{\mu} = \frac{\sum_{\mathbf{j} = \mathbf{1}}^{\mathbf{k}}{\sum_{\mathbf{i} = \mathbf{1}}^{\mathbf{n}}\mathbf{Y}_{\mathbf{ij}}}}{\mathbf{n}*\mathbf{k}}\]

\({\mathbf{\sigma}_{\mathbf{w}}}^{\mathbf{2}}:\ \)Within-voxel variance,
the estimated variance of repeated measurements

\[{\mathbf{\sigma}_{\mathbf{w}}}^{\mathbf{2}} = \frac{\sum_{\mathbf{j} = \mathbf{1}}^{\mathbf{k}}{(\mathbf{Y}_{\mathbf{ij}} - {\overline{\mathbf{Y}}}_{\mathbf{i}})\ }^{\mathbf{2}}}{\mathbf{k} - \mathbf{1}}\]

\(\mathbf{\sigma}_{\mathbf{w}}\ :\ \ \)Within-voxel standard deviation,
the standard deviation we get if we measure the voxel multiple times.
Calculated by averaging the within-subject sample variances. Since we
have a variance per voxel and we can't meaningfully take the average of
a list of standard deviations, we first calculate the variance for each
voxel, and then compute the average of those, and finally square root
that mean variance (Ye et al., 2022).

\[\mathbf{\sigma}_{\mathbf{w}} = \sqrt{\frac{\sum_{\mathbf{i} = \mathbf{1}}^{\mathbf{n}}\frac{\sum_{\mathbf{j} = \mathbf{1}}^{\mathbf{k}}{(\mathbf{Y}_{\mathbf{ij}} - {\overline{\mathbf{Y}}}_{\mathbf{i}})\ }^{\mathbf{2}}}{\mathbf{k} - \mathbf{1}}}{\mathbf{n}}}\]

The degrees of freedom, sum squares and mean square expectations that
correspond to a two-way mixed ANOVA model are summarized below.

{\def\LTcaptype{none} % do not increment counter
\begin{longtable}[]{@{}
  >{\raggedright\arraybackslash}p{(\linewidth - 6\tabcolsep) * \real{0.2511}}
  >{\raggedright\arraybackslash}p{(\linewidth - 6\tabcolsep) * \real{0.1648}}
  >{\raggedright\arraybackslash}p{(\linewidth - 6\tabcolsep) * \real{0.2996}}
  >{\raggedright\arraybackslash}p{(\linewidth - 6\tabcolsep) * \real{0.2846}}@{}}
\toprule\noalign{}
\begin{minipage}[b]{\linewidth}\raggedright
\textbf{Source of Variation}
\end{minipage} & \begin{minipage}[b]{\linewidth}\raggedright
\textbf{Degrees of freedom}
\end{minipage} & \begin{minipage}[b]{\linewidth}\raggedright
\textbf{Sum Squares}
\end{minipage} & \begin{minipage}[b]{\linewidth}\raggedright
\textbf{Mean Square Expectations}
\end{minipage} \\
\midrule\noalign{}
\endhead
\bottomrule\noalign{}
\endlastfoot
Conditions (columns) & dfc = k -1 &
\(\mathbf{SSC} = \ \sum_{\mathbf{j} = \mathbf{1}}^{\mathbf{k}}{\mathbf{n}\  \times \ {({\overline{\mathbf{Y}}}_{\mathbf{j}} - \mathbf{\mu})\ }^{\mathbf{2}}}\)
&
\(\mathbf{MSC} = \frac{\mathbf{SSC}}{\mathbf{dfc}\  \times \ \mathbf{n}\ }\) \\
Voxels (rows) & dfr = n -1 &
\(\mathbf{SSR} = \ \sum_{\mathbf{i} = \mathbf{1}}^{\mathbf{n}}{\mathbf{k}\  \times \ {({\overline{\mathbf{Y}}}_{\mathbf{i}} - \mathbf{\mu})\ }^{\mathbf{2}}}\)
& \(\mathbf{MSR} = \frac{\mathbf{SSR}}{\mathbf{dfr}\ }\) \\
Total & &
\(\mathbf{SST} = \ \sum_{\mathbf{j} = \mathbf{1}}^{\mathbf{k}}{\sum_{\mathbf{i} = \mathbf{1}}^{\mathbf{n}}{(\mathbf{Y}_{\mathbf{ij}} - \mathbf{\mu})\ }^{\mathbf{2}}\ }\)
& \\
Error (or residual) & dfe = (n -1)(k -1) &
\(\mathbf{SSE} = \mathbf{SST} - \mathbf{SSC} - \mathbf{SSR}\) &
\(\mathbf{MSE} = \frac{\mathbf{SSE}}{\mathbf{dfe}\ }\) \\
\multicolumn{4}{@{}>{\raggedright\arraybackslash}p{(\linewidth - 6\tabcolsep) * \real{1.0000} + 6\tabcolsep}@{}}{%
MSC: mean square columns, MSR: mean square rows, MSE: mean square error,
SSC= sum of squares columns, SSR=sum of squares rows, SST= sum of
squares total, SSE= sum of squares error, dfc= degrees of freedom
columns, dfr=degrees of freedom rows, dfe=degrees of freedom errors} \\
\end{longtable}
}

We compute the two versions of ICC for repeatability and
reproducibility:

Repeatibility ICC(3A,1): ICC based on single-measurement,
absolute-agreement, two-way mixed-effects model. \emph{\hfill\break
}\[\mathbf{ICC}\ (\mathbf{3A},\mathbf{1}) = \frac{\mathbf{MSR} - \mathbf{MSE}}{\mathbf{MSR} + \mathbf{dfc}\  \times \ \mathbf{MSE} + \ \frac{\mathbf{k}}{\mathbf{n}\ }\  \times (\mathbf{MSC} - \mathbf{MSE})\ \ }\]

Reproducibility ICC(3C,1): ICC based on single-measurement, consistency,
two-way mixed-effects model.

\[\mathbf{ICC}(\mathbf{3C},\mathbf{1}) = \frac{\mathbf{MSR} - \mathbf{MSE}}{\mathbf{MSR} + \mathbf{dfc} \times \ \mathbf{MSE}\ }\]

We compute the lower bound of the 95\% CI of the ICC (LCL) and the upper
bound (UCL):

\[\mathbf{LCL =}\frac{\frac{\mathbf{FR}}{\mathbf{F}}\mathbf{- 1}}{\frac{\mathbf{FR}}{\mathbf{F}}\mathbf{+ k - \ 1}}\mathbf{\ \ \ \ \ \ \ \ \ \ \ \ \ \ \ \ \ \ \ \ \ \ \ \ \ \ \ \ \ \ \ \ \ \ \ \ \ \ \ \ \ \ \ \ \ \ \ \ \ \ \ \ \ \ \ \ \ \ \ \ \ \ \ \ \ \ \ \ \ \ \ \ \ \ \ \ \ \ \ \ \ \ \ \ UCL =}\frac{\mathbf{(FR\  \times \ F)\  - 1}}{\mathbf{(FR\  \times \ F) + k - \ 1}}\]

Where F is the (1-\(\frac{\mathbf{\alpha}}{\mathbf{2}}\) ) x
100\textsuperscript{th} percentile of the F distribution with n-1
numerator degrees of freedom and (n-1)(k-1) denominator degrees of
freedom and FR is the F-statistic for voxels computed as:
\(\mathbf{FR =}\frac{\mathbf{MSR}}{\mathbf{MSE}}\)\emph{\textbf{.}}

Custom codes used to calculate ICC (3A,1) and ICC (3C,1)\textbf{,} based
on Nipype's (Esteban et al., 2022) module ICC (v1.8.5) and approved by
an statistician (VN) are available at
https://github.com/radiomicsgroup/precise-habitats.

\subsection{B.6. Justification for NGTDM Coarseness
Inclusion}\label{b.6.-justification-for-ngtdm-coarseness-inclusion}

NGTDM (Neighborhood Gray-Tone-Difference Matrix) coarseness describes
the roughness (i.e. how fine or coarse) the texture of an image is. In
the radiomics literature, evidence has been found regarding its
usefulness to characterize heterogeneity and predict progression-free
survival in oncology (Gupta et al., 2021).

In our study, we identified precise features by linking repeatability
and reproducibility results. That is, for every feature, we considered
results obtained in the three relevant experiments: repeatability
(setting R3B12), reproducibility against R (fixed B=12HU), and
reproducibility against B (fixed R=3mm). A feature was selected as
precise if it presented LCL ≥ 0.50 (i.e. moderate, good or excellent
repeatability/reproducibility) in the three experiments. NGTDM
Coarseness presented excellent repeatability and reproducibility against
bin size, but was not selected as precise as it presented poor
reproducibility against kernel radius. However, by the nature of its
definition, the poor reproducibility against kernel radius is
acceptable: the feature captures the distribution of differences in
gray-tone values between pairs of neighboring pixels. Considering the
feature's excellent results in two out of three experiments, its
potential usefulness and in light of the fact that we were already being
stringent, first by using LCL rather than ICC and second by linking
results of three different experiments, , we decided to include it as a
precise feature for both liver and lung lesions.

\subsection{B.7. Primary Tumor Has No Effect on
Precision}\label{b.7.-primary-tumor-has-no-effect-on-precision}

\begin{figure}
\centering
\includegraphics[width=6.49444in,height=6.03472in]{source/figures/annex/media/image1.png}
\caption{\textbar{} Repeatability by primary tumor type}
\end{figure}

\begin{figure}
\centering
\includegraphics[width=6.49444in,height=6.03472in]{source/figures/annex/media/image2.png}
\caption{Repeatability distribution of radiomics features computed with
setting R1B12 (\textbf{A),} R1B25 (\textbf{B),} R3B12 \textbf{(C)} and
R3B25 \textbf{(D)} per cohort for lung and liver lesions separately.
Primary tumor has no effect on repeatability. LCL, 95\% lower confidence
limit of the Intraclass Correlation Coefficient; R1B12, features
computed with kernel radius 1mm and bin size 12HU; R1B25, features
computed with kernel radius 1mm and bin size 25HU; R3B12, features
computed with kernel radius 3mm and bin size 12HU; R3B25, features
computed with kernel radius 3mm and bin size 25HU; CRC: colorectal
cohort; NET: neuroendocrine cohort; ALL: all cohorts combined.}
\end{figure}

\textbar{} Reproducibility by primary tumor type

Reproducibility distribution against R of radiomics features computed
with fixed bin size of 12HU \textbf{(A)} and fixed bin size of 25HU
\textbf{(B)} per cohort for lung and liver lesions separately. Similary,
\textbf{(C)} and \textbf{(D)} depict the reproducibility distribution
against B of radiomics features computed with fixed radius of 1mm
\textbf{(C)} and 3mm \textbf{(D)} per cohort for lung and liver lesions
separately. LCL, 95\% lower confidence limit of the Intraclass
Correlation Coefficient; CRC: colorectal cohort; NET: neuroendocrine
cohort; ALL: all cohorts combined.

Annex C

\chapter{Development and Validation of Biologically-Informed CT
Habitats}\label{development-and-validation-of-biologically-informed-ct-habitats}

This annex provides extended methods and results supporting Chapter 7.

\subsection{C.1. CT--mpMRI Co-Registration: Extended
Methods}\label{c.1.-ctmpmri-co-registration-extended-methods}

Prior to registration, all images were cropped to a tumor-centered
bounding box with a 5--7 mm margin beyond the segmentation boundary.
Cropping served two purposes: it reduced computational cost and focused
the registration algorithm on the tumor region, avoiding spurious
alignment driven by distant anatomical structures (e.g., ribs, spine)
that may differ in position between CT and MRI acquisitions.

Images were resampled to 2×2×2 mm isotropic resolution to match the
T2-weighted reference. Resampling to a common grid is necessary for
voxelwise comparison; we chose the T2w resolution as the reference
because it represented the coarsest native resolution among the
sequences and avoided artificial upsampling of M RI data.

Three pipelines were required:

\textbf{CT→T2w:} Aligns contrast-enhanced CT to the anatomical MRI
reference. This is the most challenging registration due to differences
in tissue contrast between modalities.

\textbf{DWI→T2w:} Aligns diffusion-weighted images (and derived ADC
maps) to T2w. DWI and T2w are both MRI sequences, but DWI suffers from
geometric distortion, particularly near air-tissue interfaces.

\textbf{DCE→T2w:} Aligns dynamic contrast-enhanced images (and derived
perfusion maps) to T2w. DCE images were acquired with a GRE sequence at
different resolution than T2w.

All registrations were performed using NiftyReg (reg\_aladin for
rigid/affine, reg\_f3d for B-spline deformable registration). The
pipeline proceeded sequentially: rigid registration was performed first;
if affine registration improved DSC, it replaced the rigid result; if
B-spline registration improved DSC further, it replaced the affine
result. This conservative approach avoided overfitting from unnecessary
deformable registration.

Registration quality was assessed by computing the Dice similarity
coefficient between the tumor mask on the fixed image (T2w) and the
warped tumor mask from the moving image. Tumors with CT→T2w DSC
\textless{} 0.50 were excluded from analysis, as low overlap indicates
registration failure that would propagate errors into habitat
comparisons

\subsection{C.2. Handcrafted Feature Reduction by Correlation
Analysis}\label{c.2.-handcrafted-feature-reduction-by-correlation-analysis}

Chapter 6 identified 26 radiomics features with acceptable repeatability
and reproducibility for liver lesions. However, many of these features
are highly correlated, capturing overlapping information. Clustering on
redundant features can distort distance metrics and bias cluster
assignments toward the correlated feature set.

To obtain a non-redundant feature set, we computed pairwise Spearman
correlations across all voxels in the PREDICT cohort. Features with
\textbar ρ\textbar{} ≥ 0.80 were considered redundant. From each
correlated pair, we retained the feature with higher mean repeatability
(ICC) from the Chapter 6 analysis. This procedure reduced the 26 precise
features to 6 non-redundant features:

\begin{enumerate}
\def\labelenumi{\arabic{enumi}.}
\item
  10th percentile intensity (first-order)
\item
  GLDM dependence entropy
\item
  GLDM small dependence high gray level emphasis
\item
  GLRLM gray level non-uniformity
\item
  GLRLM run length non-uniformity
\item
  NGTDM coarseness
\end{enumerate}

\hyperref[_Ref219944195]{Figure C.1} shows the correlation matrix of the
original 26 features and \hyperref[_Ref219944203]{Figure C.2} shows the
retained features, which span different texture families (GLDM, GLRLM,
NGTDM) and capture distinct aspects of local intensity variation,
ensuring that the clustering input is diverse rather than dominated by a
single texture property.

\begin{figure}
\centering
\includesvg[width=6.5in,height=5.74931in]{source/figures/annex/media/image4.svg}
\caption{\textbar{} Correlation matrix of the 26 precise handcrafted
radiomics features identified in Chapter 6}
\end{figure}

Pairwise Spearman correlations computed across all voxels in the PREDICT
cohort. Features with \textbar ρ\textbar{} ≥ 0.80 were considered
redundant.

\begin{figure}
\centering
\includesvg[width=6.02174in,height=4.5163in]{source/figures/annex/media/image6.svg}
\caption{}
\end{figure}

\textbar{} Correlation matrix of the 6 selected precise handcrafted
radiomics features identified in Chapter 6

The 6 non-redundant features retained for habitat computation are
highlighted: 10th percentile intensity, GLDM dependence entropy, GLDM
small dependence high gray level emphasis, GLRLM gray level
non-uniformity, GLRLM run length non-uniformity, and NGTDM coarseness.

\subsection{C.3. Sensitivity Analysis for Number of Habitats
(K)}\label{c.3.-sensitivity-analysis-for-number-of-habitats-k}

To determine the optimal number of habitats, we compared K = 2, 3, and 4
using the handcrafted feature representation. For each K, we computed
habitats and assessed their separation of mpMRI-derived biophysical
metrics using Kendall\textquotesingle s W effect size.

\begin{itemize}
\item
  \textbf{K = 2} achieved the highest effect sizes for most metrics but
  collapsed biologically distinct compartments into a single
  "vascularized" cluster. The two-habitat solution could not distinguish
  the cellular-perfused tumor (H2) from the vascular interface (H3).
\item
  \textbf{K = 3} provided the best interpretive balance. It separated
  the avascular core (H1) from two distinct vascularized phenotypes: a
  cellular-perfused compartment with high Ktrans (H2) and a vascular
  compartment with high fv but moderate Ktrans (H3). This three-way
  distinction aligns with the known histological architecture of
  colorectal liver metastases.
\item
  \textbf{K = 4} introduced a fourth habitat by splitting H3 into two
  subgroups. However, this split showed no clear biological
  rationale---both subhabitats had similar mpMRI profiles---and the
  fourth cluster showed unstable membership across bootstrap resampling
  (ARI = 0.91 vs. 0.97 for K = 3).
\end{itemize}

Based on these findings, K = 3 was selected as the final model
configuration.

{\def\LTcaptype{none} % do not increment counter
\begin{longtable}[]{@{}
  >{\centering\arraybackslash}p{(\linewidth - 14\tabcolsep) * \real{0.0512}}
  >{\centering\arraybackslash}p{(\linewidth - 14\tabcolsep) * \real{0.0692}}
  >{\centering\arraybackslash}p{(\linewidth - 14\tabcolsep) * \real{0.0672}}
  >{\centering\arraybackslash}p{(\linewidth - 14\tabcolsep) * \real{0.1118}}
  >{\centering\arraybackslash}p{(\linewidth - 14\tabcolsep) * \real{0.1040}}
  >{\centering\arraybackslash}p{(\linewidth - 14\tabcolsep) * \real{0.0981}}
  >{\centering\arraybackslash}p{(\linewidth - 14\tabcolsep) * \real{0.0903}}
  >{\centering\arraybackslash}p{(\linewidth - 14\tabcolsep) * \real{0.4083}}@{}}
\toprule\noalign{}
\begin{minipage}[b]{\linewidth}\centering
\textbf{K}
\end{minipage} & \begin{minipage}[b]{\linewidth}\centering
\textbf{fv\_W}
\end{minipage} & \begin{minipage}[b]{\linewidth}\centering
\textbf{fv\_p}
\end{minipage} & \begin{minipage}[b]{\linewidth}\centering
\textbf{Ktrans\_W}
\end{minipage} & \begin{minipage}[b]{\linewidth}\centering
\textbf{Ktrans\_p}
\end{minipage} & \begin{minipage}[b]{\linewidth}\centering
\textbf{ADCt\_W}
\end{minipage} & \begin{minipage}[b]{\linewidth}\centering
\textbf{ADCt\_p}
\end{minipage} & \begin{minipage}[b]{\linewidth}\centering
\textbf{Interpretation}
\end{minipage} \\
\midrule\noalign{}
\endhead
\bottomrule\noalign{}
\endlastfoot
2 & 0.98 & 0.002 & 0.78 & 0.014 & 0.62 & 0.049 & Strong separation but
oversimplified; merges biologically distinct vascular phenotypes \\
3 & 0.67 & 0.005 & 0.52 & 0.018 & 0.16 & 0.328 & Best balance; separates
vascular gradient while identifying cellular-perfused compartment \\
4 & 0.85 & 0 & 0.37 & 0.025 & 0.32 & 0.039 & Fourth habitat splits H3
without biological justification; unstable across bootstrap \\
\end{longtable}
}

\textbf{Table C.1 \textbar{} Biophysical separation
(Kendall\textquotesingle s W) across different numbers of habitats (K).}

\subsection{C.4. Technical Validation: Extended
Results}\label{c.4.-technical-validation-extended-results}

\begin{itemize}
\item
  \textbf{Initialization stability:} All representations achieved ARI
  \textgreater{} 0.96, indicating that clustering solutions were
  reproducible regardless of random seed initialization.
\item
  \textbf{Data stability:} All representations showed high bootstrap
  stability (median ARI \textgreater{} 0.96), indicating that habitat
  definitions were not driven by a few influential patients.
\item
  \textbf{Spatial coherence:} Handcrafted features produced the highest
  Moran\textquotesingle s I (0.804), indicating strong spatial
  autocorrelation---habitats formed contiguous regions rather than
  scattered voxels. DL-SALSA and DL-FM produced lower spatial coherence
  (0.513--0.622), reflecting more fragmented "salt-and-pepper" patterns.
  Raw HU showed the lowest spatial coherence (0.420).
\end{itemize}

The combination of high initialization stability, adequate data
stability, and superior spatial coherence supported the selection of
handcrafted features for the final CT habitat model.

{\def\LTcaptype{none} % do not increment counter
\begin{longtable}[]{@{}
  >{\centering\arraybackslash}p{(\linewidth - 6\tabcolsep) * \real{0.1838}}
  >{\centering\arraybackslash}p{(\linewidth - 6\tabcolsep) * \real{0.2015}}
  >{\centering\arraybackslash}p{(\linewidth - 6\tabcolsep) * \real{0.3137}}
  >{\centering\arraybackslash}p{(\linewidth - 6\tabcolsep) * \real{0.2919}}@{}}
\toprule\noalign{}
\multicolumn{4}{@{}>{\centering\arraybackslash}p{(\linewidth - 6\tabcolsep) * \real{0.9909} + 6\tabcolsep}@{}}{%
\begin{minipage}[b]{\linewidth}\centering
Table C.2 \textbar{} Technical robustness of candidate CT feature
representations.
\end{minipage}} \\
\midrule\noalign{}
\endhead
\bottomrule\noalign{}
\endlastfoot
& \textbf{Initialization Stability (ARI)} & \textbf{Data Stability
(ARI)}

{[}Median (IQR){]} & \textbf{Moran's I}

{[}Mean ± SD{]} \\
\begin{minipage}[t]{\linewidth}\centering
\begin{quote}
Raw HU
\end{quote}
\end{minipage} & 0.965 & \textbf{0.991 (0.980-0.997)} & 0.420
±0.0.143 \\
\begin{minipage}[t]{\linewidth}\centering
\begin{quote}
Handcrafted
\end{quote}
\end{minipage} & \textbf{0.997} & 0.966 (0.959-0.985) & \textbf{0.804
±0.056} \\
\begin{minipage}[t]{\linewidth}\centering
\begin{quote}
DL-SALSA
\end{quote}
\end{minipage} & 0.984 & 0.985 (0.977-0.988) & 0.513 ±0.110 \\
\begin{minipage}[t]{\linewidth}\centering
\begin{quote}
DL-FM
\end{quote}
\end{minipage} & 0.978 & 0.989 (0.976-0.992) & 0.622 ±0.119 \\
\end{longtable}
}

\subsection{C.5. mpMRI Characterization: Pairwise
Comparison}\label{c.5.-mpmri-characterization-pairwise-comparison}

For metrics with significant Friedman test (p \textless{} 0.05,
BH-corrected), we report Wilcoxon signed-rank tests comparing habitat
pairs. Effect size r = Z/√N. All p-values are BH-corrected within each
metric. N = 10 patients.

\begin{longtable}[]{@{}
  >{\centering\arraybackslash}p{(\linewidth - 12\tabcolsep) * \real{0.0901}}
  >{\centering\arraybackslash}p{(\linewidth - 12\tabcolsep) * \real{0.1714}}
  >{\centering\arraybackslash}p{(\linewidth - 12\tabcolsep) * \real{0.1307}}
  >{\centering\arraybackslash}p{(\linewidth - 12\tabcolsep) * \real{0.1712}}
  >{\centering\arraybackslash}p{(\linewidth - 12\tabcolsep) * \real{0.1306}}
  >{\centering\arraybackslash}p{(\linewidth - 12\tabcolsep) * \real{0.1734}}
  >{\centering\arraybackslash}p{(\linewidth - 12\tabcolsep) * \real{0.1328}}@{}}
\caption{\textbf{Table C.3 \textbar{} Post-hoc pairwise comparisons for
mpMRI metrics across CT habitats.}}\tabularnewline
\toprule\noalign{}
\begin{minipage}[b]{\linewidth}\centering
\textbf{Metric}
\end{minipage} & \begin{minipage}[b]{\linewidth}\centering
\textbf{H1\_vs\_H2\_p\_BH}
\end{minipage} & \begin{minipage}[b]{\linewidth}\centering
\textbf{H1\_vs\_H2\_r}
\end{minipage} & \begin{minipage}[b]{\linewidth}\centering
\textbf{H1\_vs\_H3\_p\_BH}
\end{minipage} & \begin{minipage}[b]{\linewidth}\centering
\textbf{H1\_vs\_H3\_r}
\end{minipage} & \begin{minipage}[b]{\linewidth}\centering
\textbf{H2\_vs\_H3\_p\_BH}
\end{minipage} & \begin{minipage}[b]{\linewidth}\centering
\textbf{H2\_vs\_H3\_r}
\end{minipage} \\
\midrule\noalign{}
\endfirsthead
\toprule\noalign{}
\begin{minipage}[b]{\linewidth}\centering
\textbf{Metric}
\end{minipage} & \begin{minipage}[b]{\linewidth}\centering
\textbf{H1\_vs\_H2\_p\_BH}
\end{minipage} & \begin{minipage}[b]{\linewidth}\centering
\textbf{H1\_vs\_H2\_r}
\end{minipage} & \begin{minipage}[b]{\linewidth}\centering
\textbf{H1\_vs\_H3\_p\_BH}
\end{minipage} & \begin{minipage}[b]{\linewidth}\centering
\textbf{H1\_vs\_H3\_r}
\end{minipage} & \begin{minipage}[b]{\linewidth}\centering
\textbf{H2\_vs\_H3\_p\_BH}
\end{minipage} & \begin{minipage}[b]{\linewidth}\centering
\textbf{H2\_vs\_H3\_r}
\end{minipage} \\
\midrule\noalign{}
\endhead
\bottomrule\noalign{}
\endlastfoot
ADCt & 0.029 & 0.82 & 0.24 & 0.44 & 1 & 0 \\
ADCv & 0.126 & 0.55 & 0.111 & 0.66 & 0.232 & 0.38 \\
fv & 0.375 & 0.28 & 0.006 & 0.98 & 0.006 & 0.91 \\
D0 & 0.049 & 0.62 & 0.003 & 0.98 & 0.003 & 0.98 \\
T2star & 0.02 & 0.74 & 0.003 & 0.98 & 0.003 & 0.98 \\
T1 & 0.029 & 0.74 & 0.012 & 0.91 & 0.037 & 0.66 \\
Ktrans & 0.015 & 0.87 & 0.015 & 0.82 & 0.16 & 0.44 \\
\end{longtable}

\textbf{Key findings:}

• \textbf{Vascular gradient (H1 → H3):} fv, D0, T2*, and T1 showed
significant differences between H1 and H3, with large effect sizes (r
\textgreater{} 0.90), confirming a vascular gradient from the avascular
core to the vascular rim.

• \textbf{Cellular peak (H2):} ADCt was significantly lower in H2 than
H1 (p = 0.029, r = 0.82), indicating H2 as the most cellular habitat.
Ktrans was significantly higher in H2 than H1 (p = 0.015, r = 0.87),
consistent with leaky tumor neovessels.

• \textbf{H2 vs H3 distinction:} fv distinguished H2 from H3 (p =
0.006), but Ktrans did not (p = 0.160). This supports the interpretation
that H3 represents a vascular compartment with mature (less leaky)
vessels, possibly reflecting partial volume with normal liver.

\subsection{C.6. Habitat-Histology
Correlations}\label{c.6.-habitat-histology-correlations}

As an exploratory analysis, we computed Spearman correlations between
whole-tumor habitat proportions and histological tissue percentages in
the POEM cohort (N = 6 tumors). Direct voxel-to-voxel co-registration
between CT and histology was not feasible; correlations therefore
reflect whole-tumor associations only.

\begin{figure}
\centering
\includegraphics[width=5.71739in,height=5.7847in]{source/figures/annex/media/image7.png}
\caption{\textbar{} \textbar{} Exploratory correlations between CT
habitat proportions and histological tissue percentages (POEM cohort, N
= 6)}
\end{figure}

Each panel shows the relationship between a habitat proportion (rows:
H1, H2, H3) and a histological tissue percentage (columns: necrosis,
fibrosis, viable tumor). Spearman correlation coefficients (ρ) and
p-values are shown. All correlations were non-significant, reflecting
the small sample size and CT\textquotesingle s inability to distinguish
necrosis from fibrosis within the avascular compartment.

\textbf{Table C.4 \textbar{} Spearman correlations between habitat
proportions and histological tissue percentages.}

{\def\LTcaptype{none} % do not increment counter
\begin{longtable}[]{@{}
  >{\centering\arraybackslash}p{(\linewidth - 12\tabcolsep) * \real{0.0949}}
  >{\centering\arraybackslash}p{(\linewidth - 12\tabcolsep) * \real{0.1451}}
  >{\centering\arraybackslash}p{(\linewidth - 12\tabcolsep) * \real{0.1246}}
  >{\centering\arraybackslash}p{(\linewidth - 12\tabcolsep) * \real{0.1404}}
  >{\centering\arraybackslash}p{(\linewidth - 12\tabcolsep) * \real{0.1199}}
  >{\centering\arraybackslash}p{(\linewidth - 12\tabcolsep) * \real{0.1979}}
  >{\centering\arraybackslash}p{(\linewidth - 12\tabcolsep) * \real{0.1774}}@{}}
\toprule\noalign{}
\begin{minipage}[b]{\linewidth}\centering
\textbf{Habitat}
\end{minipage} & \begin{minipage}[b]{\linewidth}\centering
\textbf{Necrosis\_rho}
\end{minipage} & \begin{minipage}[b]{\linewidth}\centering
\textbf{Necrosis\_p}
\end{minipage} & \begin{minipage}[b]{\linewidth}\centering
\textbf{Fibrosis\_rho}
\end{minipage} & \begin{minipage}[b]{\linewidth}\centering
\textbf{Fibrosis\_p}
\end{minipage} & \begin{minipage}[b]{\linewidth}\centering
\textbf{Viable\_Tumor\_rho}
\end{minipage} & \begin{minipage}[b]{\linewidth}\centering
\textbf{Viable\_Tumor\_p}
\end{minipage} \\
\midrule\noalign{}
\endhead
\bottomrule\noalign{}
\endlastfoot
H1 & 0.49 & 0.329 & -0.6 & 0.208 & -0.31 & 0.544 \\
H2 & -0.37 & 0.468 & 0.43 & 0.397 & 0.14 & 0.787 \\
H3 & 0.03 & 0.957 & 0.77 & 0.072 & -0.09 & 0.872 \\
\end{longtable}
}

All correlations were weak to moderate and non-significant. Several
trends were observed:

\begin{itemize}
\item
  \textbf{H1 and necrosis:} A positive trend (ρ = 0.49) suggests that
  the avascular habitat may partially capture necrotic tissue,
  consistent with its biological profile (low vascularity, low
  cellularity).
\item
  \textbf{H1 and fibrosis:} A negative trend (ρ = −0.60) suggests that
  H1 does not specifically correspond to fibrosis. This is expected: CT
  cannot distinguish necrosis from fibrosis, and both may appear as
  avascular tissue.
\item
  \textbf{H3 and fibrosis:} The strongest trend observed (ρ = 0.77, p =
  0.072) suggests a potential association between the vascular habitat
  and fibrotic tissue. However, this finding is difficult to interpret
  biologically and may reflect confounding by tumor size or treatment
  history.
\end{itemize}

The sample size (N = 6) provides insufficient power to detect moderate
correlations. Additionally, the scale mismatch between voxel-level
imaging (mm resolution) and microscopic histology (μm resolution),
combined with tissue deformation during resection and processing, limits
the interpretability of whole-tumor correlations. These exploratory
findings should be interpreted with caution and require validation in
larger cohorts with spatially co-registered imaging and histology.

Annex D

\chapter{Clinical Relevance of CT
Habitats}\label{clinical-relevance-of-ct-habitats}

\subsection{D.1. Habitat Metrics and Tumor
Volume}\label{d.1.-habitat-metrics-and-tumor-volume}

Table D.1 presents the full correlation analysis between habitat-derived
metrics and tumor volume for both cohorts.

\textbf{Table D.1 \textbar{} Spearman correlations between habitat
metrics and tumor volume.}

{\def\LTcaptype{none} % do not increment counter
\begin{longtable}[]{@{}
  >{\raggedright\arraybackslash}p{(\linewidth - 4\tabcolsep) * \real{0.3915}}
  >{\centering\arraybackslash}p{(\linewidth - 4\tabcolsep) * \real{0.1162}}
  >{\centering\arraybackslash}p{(\linewidth - 4\tabcolsep) * \real{0.1261}}@{}}
\toprule\noalign{}
\begin{minipage}[b]{\linewidth}\centering
\textbf{Metric}
\end{minipage} & \begin{minipage}[b]{\linewidth}\centering
\textbf{TCIA (ρ)}
\end{minipage} & \begin{minipage}[b]{\linewidth}\centering
\textbf{VHIO (ρ)}
\end{minipage} \\
\midrule\noalign{}
\endhead
\bottomrule\noalign{}
\endlastfoot
\textbf{Rim metrics} & & \\
Rim entropy vs Volume & -0.77*** & -0.71*** \\
Rim cellular-perfused vs Volume & 0.49*** & 0.45*** \\
Rim avascular vs Volume & -0.32*** & -0.28*** \\
Rim vascular vs Volume & -0.21** & -0.19** \\
\textbf{Whole-tumor metrics} & & \\
Whole entropy vs Volume & -0.44*** & -0.62*** \\
Whole avascular vs Volume & 0.44*** & 0.70*** \\
Whole cellular-perfused vs Volume & 0.12 & 0.08 \\
Whole vascular vs Volume & -0.38*** & -0.52*** \\
\textbf{Core metrics} & & \\
Core entropy vs Volume & -0.28*** & -0.41*** \\
Core avascular vs Volume & 0.31*** & 0.55*** \\
Core cellular-perfused vs Volume & -0.18* & -0.22** \\
\end{longtable}
}

*p\textless0.05, **p\textless0.01, ***p\textless0.001

\subsection{D.2. TCIA: Neoadjuvant Chemotherapy Remodels Tumor
Composition}\label{d.2.-tcia-neoadjuvant-chemotherapy-remodels-tumor-composition}

Table D.2 presents FDR-adjusted p-values for all habitat metric
comparisons between treatment-naive and neoadjuvant-treated TCIA
patients.

\textbf{Table D.2 \textbar{} Habitat metric comparisons: treatment-naive
vs neoadjuvant-treated (TCIA).}

{\def\LTcaptype{none} % do not increment counter
\begin{longtable}[]{@{}
  >{\centering\arraybackslash}p{(\linewidth - 8\tabcolsep) * \real{0.1674}}
  >{\centering\arraybackslash}p{(\linewidth - 8\tabcolsep) * \real{0.2891}}
  >{\centering\arraybackslash}p{(\linewidth - 8\tabcolsep) * \real{0.2740}}
  >{\centering\arraybackslash}p{(\linewidth - 8\tabcolsep) * \real{0.1322}}
  >{\centering\arraybackslash}p{(\linewidth - 8\tabcolsep) * \real{0.1373}}@{}}
\toprule\noalign{}
\begin{minipage}[b]{\linewidth}\centering
\textbf{Metric}
\end{minipage} & \begin{minipage}[b]{\linewidth}\centering
\textbf{Treatment-naive (n=74) Median {[}IQR{]}}
\end{minipage} & \begin{minipage}[b]{\linewidth}\centering
\textbf{Neoadjuvant (n=115) Median {[}IQR{]}}
\end{minipage} & \begin{minipage}[b]{\linewidth}\centering
\textbf{Unadjusted p}
\end{minipage} & \begin{minipage}[b]{\linewidth}\centering
\textbf{FDR-adjusted p}
\end{minipage} \\
\midrule\noalign{}
\endhead
\bottomrule\noalign{}
\endlastfoot
Tumor volume (mm³) & 14,639 {[}5,812--38,421{]} & 8,620
{[}3,284--24,108{]} & 0.018 & 0.057 \\
Whole entropy & 1.45 {[}1.38--1.52{]} & 1.51 {[}1.44--1.57{]} &
\textless0.001 & 0.001 \\
Whole avascular & 0.398 {[}0.31--0.49{]} & 0.354 {[}0.27--0.44{]} &
0.019 & 0.057 \\
Whole cellular-perfused & 0.412 {[}0.34--0.48{]} & 0.408
{[}0.35--0.47{]} & 0.75 & 0.75 \\
Whole vascular & 0.168 {[}0.12--0.22{]} & 0.214 {[}0.16--0.27{]} &
\textless0.001 & 0.001 \\
Rim entropy & 1.13 {[}1.02--1.24{]} & 1.25 {[}1.14--1.35{]} &
\textless0.001 & 0.001 \\
Rim avascular & 0.152 {[}0.09--0.22{]} & 0.138 {[}0.08--0.20{]} & 0.28 &
0.42 \\
Rim cellular-perfused & 0.485 {[}0.41--0.56{]} & 0.512 {[}0.44--0.58{]}
& 0.057 & 0.11 \\
Rim vascular & 0.342 {[}0.28--0.41{]} & 0.328 {[}0.27--0.39{]} & 0.35 &
0.47 \\
Core entropy & 1.08 {[}0.95--1.21{]} & 1.12 {[}1.01--1.24{]} & 0.09 &
0.16 \\
\end{longtable}
}

Mann-Whitney U test. FDR correction using Benjamini-Hochberg procedure.
Bold indicates FDR-adjusted p\textless0.05.

\subsection{D.3. Cox Regression Analyses by Treatment
(VHIO)}\label{d.3.-cox-regression-analyses-by-treatment-vhio}

\textbf{Table D.3a \textbar{} Cox regression: Chemotherapy alone (n=122,
89 events).}

{\def\LTcaptype{none} % do not increment counter
\begin{longtable}[]{@{}
  >{\centering\arraybackslash}p{(\linewidth - 8\tabcolsep) * \real{0.1837}}
  >{\centering\arraybackslash}p{(\linewidth - 8\tabcolsep) * \real{0.2523}}
  >{\centering\arraybackslash}p{(\linewidth - 8\tabcolsep) * \real{0.0680}}
  >{\centering\arraybackslash}p{(\linewidth - 8\tabcolsep) * \real{0.2700}}
  >{\centering\arraybackslash}p{(\linewidth - 8\tabcolsep) * \real{0.0696}}@{}}
\toprule\noalign{}
\begin{minipage}[b]{\linewidth}\centering
\textbf{Variable}
\end{minipage} & \begin{minipage}[b]{\linewidth}\centering
\textbf{Univariable HR {[}95\% CI{]}}
\end{minipage} & \begin{minipage}[b]{\linewidth}\centering
\textbf{p}
\end{minipage} & \begin{minipage}[b]{\linewidth}\centering
\textbf{Multivariable HR {[}95\% CI{]}}
\end{minipage} & \begin{minipage}[b]{\linewidth}\centering
\textbf{p}
\end{minipage} \\
\midrule\noalign{}
\endhead
\bottomrule\noalign{}
\endlastfoot
Tumor volume & 5.46 {[}3.19--9.34{]} & \textless0.001 & 5.21
{[}2.98--9.12{]} & \textless0.001 \\
Rim entropy & 0.88 {[}0.54--1.43{]} & 0.61 & --- & --- \\
Extrahepatic disease & 1.82 {[}1.18--2.81{]} & 0.007 & 1.54
{[}0.98--2.42{]} & 0.06 \\
Age & 1.24 {[}0.92--1.67{]} & 0.16 & --- & --- \\
\textbf{C-index} & & & 0.703 & \\
\end{longtable}
}

\textbf{Table D.3b \textbar{} Cox regression: Chemotherapy + Bevacizumab
(n=133, 95 events).}

{\def\LTcaptype{none} % do not increment counter
\begin{longtable}[]{@{}
  >{\centering\arraybackslash}p{(\linewidth - 8\tabcolsep) * \real{0.1837}}
  >{\centering\arraybackslash}p{(\linewidth - 8\tabcolsep) * \real{0.2523}}
  >{\centering\arraybackslash}p{(\linewidth - 8\tabcolsep) * \real{0.0680}}
  >{\centering\arraybackslash}p{(\linewidth - 8\tabcolsep) * \real{0.2700}}
  >{\centering\arraybackslash}p{(\linewidth - 8\tabcolsep) * \real{0.0572}}@{}}
\toprule\noalign{}
\begin{minipage}[b]{\linewidth}\centering
\textbf{Variable}
\end{minipage} & \begin{minipage}[b]{\linewidth}\centering
\textbf{Univariable HR {[}95\% CI{]}}
\end{minipage} & \begin{minipage}[b]{\linewidth}\centering
\textbf{p}
\end{minipage} & \begin{minipage}[b]{\linewidth}\centering
\textbf{Multivariable HR {[}95\% CI{]}}
\end{minipage} & \begin{minipage}[b]{\linewidth}\centering
\textbf{p}
\end{minipage} \\
\midrule\noalign{}
\endhead
\bottomrule\noalign{}
\endlastfoot
Tumor volume & 1.69 {[}1.31--2.17{]} & \textless0.001 & 1.58
{[}1.21--2.06{]} & 0.001 \\
Rim entropy & 0.68 {[}0.52--0.88{]} & 0.004 & 0.71 {[}0.54--0.93{]} &
0.012 \\
Extrahepatic disease & 1.44 {[}0.95--2.18{]} & 0.09 & 1.38
{[}0.90--2.12{]} & 0.14 \\
Left primary & 0.58 {[}0.38--0.89{]} & 0.012 & 0.62 {[}0.40--0.96{]} &
0.031 \\
\textbf{C-index} & & & 0.674 & \\
\end{longtable}
}

\textbf{Table D.3c \textbar{} Cox regression: RAS-mutant patients
(n=195, 138 events).}

{\def\LTcaptype{none} % do not increment counter
\begin{longtable}[]{@{}
  >{\centering\arraybackslash}p{(\linewidth - 8\tabcolsep) * \real{0.1837}}
  >{\centering\arraybackslash}p{(\linewidth - 8\tabcolsep) * \real{0.2523}}
  >{\centering\arraybackslash}p{(\linewidth - 8\tabcolsep) * \real{0.0680}}
  >{\centering\arraybackslash}p{(\linewidth - 8\tabcolsep) * \real{0.2700}}
  >{\centering\arraybackslash}p{(\linewidth - 8\tabcolsep) * \real{0.0696}}@{}}
\toprule\noalign{}
\begin{minipage}[b]{\linewidth}\centering
\textbf{Variable}
\end{minipage} & \begin{minipage}[b]{\linewidth}\centering
\textbf{Univariable HR {[}95\% CI{]}}
\end{minipage} & \begin{minipage}[b]{\linewidth}\centering
\textbf{p}
\end{minipage} & \begin{minipage}[b]{\linewidth}\centering
\textbf{Multivariable HR {[}95\% CI{]}}
\end{minipage} & \begin{minipage}[b]{\linewidth}\centering
\textbf{p}
\end{minipage} \\
\midrule\noalign{}
\endhead
\bottomrule\noalign{}
\endlastfoot
Tumor volume & 1.89 {[}1.41--2.53{]} & \textless0.001 & 1.78
{[}1.32--2.41{]} & \textless0.001 \\
Rim entropy & 0.80 {[}0.65--1.00{]} & 0.047 & 0.82 {[}0.66--1.02{]} &
0.07 \\
Extrahepatic disease & 1.61 {[}1.14--2.27{]} & 0.007 & 1.48
{[}1.04--2.11{]} & 0.03 \\
Age & 1.18 {[}0.96--1.45{]} & 0.11 & --- & --- \\
\textbf{C-index} & & & 0.700 & \\
\end{longtable}
}

\subsection{D.4. Longitudinal Analysis Summary
Statistics}\label{d.4.-longitudinal-analysis-summary-statistics}

\textbf{Table D.4 \textbar{} Rim entropy change by RECIST category
(n=38).}

{\def\LTcaptype{none} % do not increment counter
\begin{longtable}[]{@{}
  >{\centering\arraybackslash}p{(\linewidth - 12\tabcolsep) * \real{0.1770}}
  >{\centering\arraybackslash}p{(\linewidth - 12\tabcolsep) * \real{0.0285}}
  >{\centering\arraybackslash}p{(\linewidth - 12\tabcolsep) * \real{0.2306}}
  >{\centering\arraybackslash}p{(\linewidth - 12\tabcolsep) * \real{0.1268}}
  >{\centering\arraybackslash}p{(\linewidth - 12\tabcolsep) * \real{0.1097}}
  >{\centering\arraybackslash}p{(\linewidth - 12\tabcolsep) * \real{0.1150}}
  >{\centering\arraybackslash}p{(\linewidth - 12\tabcolsep) * \real{0.2124}}@{}}
\toprule\noalign{}
\begin{minipage}[b]{\linewidth}\centering
\textbf{RECIST Category}
\end{minipage} & \begin{minipage}[b]{\linewidth}\centering
\textbf{n}
\end{minipage} & \begin{minipage}[b]{\linewidth}\centering
\textbf{Median Δ Rim Entropy}
\end{minipage} & \begin{minipage}[b]{\linewidth}\centering
\textbf{IQR}
\end{minipage} & \begin{minipage}[b]{\linewidth}\centering
\textbf{n Increase}
\end{minipage} & \begin{minipage}[b]{\linewidth}\centering
\textbf{n Decrease}
\end{minipage} & \begin{minipage}[b]{\linewidth}\centering
\textbf{Median OS (months)}
\end{minipage} \\
\midrule\noalign{}
\endhead
\bottomrule\noalign{}
\endlastfoot
All patients & 38 & +0.013 & {[}-0.04, +0.10{]} & 21 & 17 & 27.1 \\
PR & 23 & +0.053 & {[}-0.02, +0.10{]} & 14 & 9 & 32.8 \\
SD & 8 & +0.023 & {[}-0.06, +0.16{]} & 4 & 4 & 18.3 \\
PD & 7 & -0.041 & {[}-0.25, +0.03{]} & 3 & 4 & 9.0 \\
\end{longtable}
}

\chapter{}\label{section}

\backmatter
\end{document}
