% ============================================================================
% RESUMEN (Español)
% ============================================================================

\chapter*{Resumen}
\addcontentsline{toc}{chapter}{Resumen}

En el cáncer colorrectal, la mortalidad está impulsada principalmente
por la enfermedad metastásica, más comúnmente al hígado. Las metástasis
hepáticas de cáncer colorrectal son biológicamente heterogéneas y están
compuestas por proporciones variables de células tumorales viables,
fibrosis y necrosis. Tanto la presencia como la organización espacial de
estos componentes tisulares influyen en la respuesta al tratamiento y en
el pronóstico del paciente. Sin embargo, en la práctica clínica, esta
información solo está disponible mediante biopsia, que evalúa una región
tumoral limitada, o tras la resección quirúrgica, que solo es factible
en una minoría de pacientes. Como resultado, la mayoría de los pacientes
recibe tratamiento sistémico sin conocimiento directo de la composición
tisular del tumor completo.

La tomografía computarizada (TC) es la modalidad de imagen estándar para
el manejo de las metástasis hepáticas de cáncer colorrectal y se
adquiere de forma repetida a lo largo de la evolución de la enfermedad.
A pesar de proporcionar información no invasiva y espacialmente resuelta
a nivel de todo el tumor, la interpretación clínica continúa centrándose
principalmente en el tamaño, número y localización de las lesiones. En
consecuencia, la capacidad de la TC para caracterizar la heterogeneidad
intratumoral aún no se ha aprovechado plenamente.

El análisis de hábitats de imagen se ha propuesto como una herramienta
para capturar la heterogeneidad tumoral mediante la partición de los
tumores en subregiones espaciales con propiedades de imagen similares
(hábitats). La mayoría de los estudios de imagen de hábitats se basan en
resonancia magnética multiparamétrica (RMmp), cuyos mapas cuantitativos
son biológicamente interpretables, mientras que las aplicaciones en TC
siguen siendo limitadas a pesar de su amplio uso clínico. Además, los
estudios existentes rara vez evalúan la robustez de las características
radiómicas derivadas de TC, a menudo definen los hábitats mediante
optimización puramente basada en datos sin una base biológica, y
generalmente no informan sobre su relevancia clínica más allá del
volumen tumoral.

Esta tesis aborda estas limitaciones investigando si la TC de rutina
puede capturar heterogeneidad intratumoral biológica y clínicamente
relevante en las metástasis hepáticas de cáncer colorrectal. En primer
lugar, identificamos 26 características radiómicas adecuadas para el
cálculo robusto de hábitats basados en TC, según criterios de
repetibilidad y reproducibilidad. En segundo lugar, desarrollamos un
modelo de hábitats en TC con anclaje biológico incorporando RMmp
coregistrada como referencia durante la definición de los hábitats, en
lugar de basarnos exclusivamente en la optimización estadística. Dentro
de este marco, comparamos múltiples representaciones de TC, incluidas
características radiómicas artesanales y representaciones obtenidas
mediante aprendizaje profundo, y observamos que las características
artesanales producían hábitats biológicamente más coherentes. Los tres
hábitats resultantes reflejaron la arquitectura vascular: un núcleo
avascular, una zona intermedia celular y perfundida, y un borde externo
altamente vascularizado.

En tercer lugar, evaluamos la relevancia clínica analizando la
asociación entre métricas derivadas de los hábitats y los resultados
clínicos en dos cohortes independientes. Las métricas de hábitat
proporcionaron información pronóstica más allá del volumen tumoral, pero
solo en contextos terapéuticos específicos. En particular, la entropía
del hábitat en la interfaz tumor--hígado predijo la supervivencia en
situaciones en las que el tratamiento puede alterar la composición
tisular sin inducir cambios medibles en el tamaño. En todos los
contextos terapéuticos, la información pronóstica se localizó de forma
consistente en el borde tumoral invasivo, en lugar de distribuirse
uniformemente en toda la lesión.

En conjunto, esta tesis aporta avances tanto metodológicos como
clínicos: un pipeline de imagen de hábitats en TC de código abierto, una
evaluación exhaustiva de la robustez de las características radiómicas
artesanales, la primera comparación de representaciones de TC para el
cálculo de hábitats, el anclaje biológico de los hábitats derivados de
TC utilizando RMmp y la demostración de su relevancia clínica
dependiente del contexto. En conjunto, estos hallazgos establecen que
las TC de rutina contienen información clínicamente relevante sobre la
heterogeneidad tumoral que las estrategias actuales de evaluación no
capturan, y proporcionan un marco para su extracción.

\vspace{1cm}

\noindent\textbf{Palabras clave:} metástasis hepáticas, cáncer colorrectal,
análisis de hábitats de imagen, heterogeneidad intratumoral, radiómica,
tomografía computarizada, resonancia magnética multiparamétrica,
oncología de precisión

\cleardoublepage
