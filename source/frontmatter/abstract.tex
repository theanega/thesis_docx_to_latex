% ============================================================================
% ABSTRACT (English)
% ============================================================================

\chapter*{Abstract}
\addcontentsline{toc}{chapter}{Abstract}

In colorectal cancer, mortality is primarily driven by metastatic
disease, most commonly to the liver. Colorectal liver metastases are
biologically heterogeneous, composed of varying proportions of viable
tumor cells, fibrosis, and necrosis. Both the presence and spatial
organization of these tissue components influence treatment response and
patient outcome. In clinical practice, however, such information is
available only through biopsy, which samples a limited tumor region, or
after surgical resection, which is feasible in only a minority of
patients. As a result, most patients receive systemic therapy without
direct knowledge of whole-tumor tissue composition.

Computed tomography (CT) is the standard imaging modality for managing
colorectal liver metastases and is acquired repeatedly throughout the
disease course. Despite providing non-invasive, spatially resolved
information at the whole-tumor level, clinical interpretation largely
focuses on lesion size, number, and location. Consequently, the ability
of CT to characterize intratumor heterogeneity has not yet been fully
leveraged.

Habitat imaging has been proposed as a framework for capturing tumor
heterogeneity by partitioning tumors into spatial subregions with
similar imaging properties (habitats). Most habitat imaging studies rely
on multiparametric MRI (mpMRI), whose quantitative maps are biologically
interpretable, while applications to CT remain limited despite its
widespread clinical use. In addition, existing studies rarely assess the
robustness of CT-derived features, often define habitats through purely
data-driven optimization without biological grounding, and generally do
not report clinical relevance beyond tumor volume.

This thesis addresses these gaps by investigating whether routine CT can
capture biologically and clinically meaningful intratumor heterogeneity
in colorectal liver metastases. First, we identified 26 radiomic
features suitable for robust CT-based habitat computation based on
repeatability and reproducibility criteria. Second, we developed a
biologically anchored CT habitat model by incorporating co-registered
mpMRI as a reference during habitat definition, rather than relying
solely on statistical optimization. Within this framework, we compared
multiple CT representations, including handcrafted radiomic features and
deep learning embeddings, and found that handcrafted features produced
more biologically coherent habitats. The resulting three habitats
reflected vascular architecture: an avascular core, a cellular, perfused
intermediate zone, and a highly vascularized outer rim.

Third, we evaluated clinical relevance by assessing associations between
habitat-derived metrics and patient outcomes in two independent cohorts.
Habitat metrics provided prognostic information beyond tumor volume, but
only in specific treatment contexts. In particular, habitat entropy at
the tumor--liver interface predicted survival in settings in which
treatment may alter tissue composition without inducing measurable size
changes. Across all treatment contexts, prognostic information
consistently localized to the invasive tumor rim rather than being
uniformly distributed throughout the lesion.

Overall, this thesis contributes both methodological and clinical
advances: an open-source CT habitat imaging pipeline, a comprehensive
assessment of handcrafted radiomic feature robustness, the first
comparison of CT representations for habitat computation, biological
grounding of CT-derived habitats using mpMRI, and demonstration of their
context-dependent clinical relevance. Together, these findings establish
that routine CT scans contain clinically meaningful information about
tumor heterogeneity that current assessment strategies do not capture,
and provide a framework for extracting it.

\vspace{1cm}

\noindent\textbf{Keywords:} liver metastases, colorectal cancer, habitat imaging,
intratumor heterogeneity, radiomics, computed tomography,
multiparametric magnetic resonance imaging, precision oncology

\cleardoublepage
