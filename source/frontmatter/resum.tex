% ============================================================================
% RESUM (Català)
% ============================================================================

\chapter*{Resum}
\addcontentsline{toc}{chapter}{Resum}

En el càncer colorectal, la mortalitat està impulsada principalment per
la malaltia metastàtica, més habitualment al fetge. Les metàstasis
hepàtiques del càncer colorectal són biològicament heterogènies i estan
compostes per proporcions variables de cèl·lules tumorals viables,
fibrosi i necrosi. Tant la presència com l'organització espacial
d'aquests components tissulars influeixen en la resposta al tractament i
en el pronòstic del pacient. Tanmateix, en la pràctica clínica, aquesta
informació només està disponible mitjançant biòpsia, que avalua una
regió tumoral limitada, o després de la resecció quirúrgica, que només
és factible en una minoria de pacients. Com a resultat, la majoria dels
pacients rep tractament sistèmic sense coneixement directe de la
composició tissular del tumor complet.

La tomografia computada (TC) és la modalitat d'imatge estàndard per al
maneig de les metàstasis hepàtiques del càncer colorectal i s'adquireix
de manera repetida al llarg de l'evolució de la malaltia. Tot i
proporcionar informació no invasiva i amb resolució espacial a nivell de
tot el tumor, la interpretació clínica continua centrant-se
principalment en la mida, el nombre i la localització de les lesions. En
conseqüència, la capacitat de la TC per caracteritzar la heterogeneïtat
intratumoral encara no s'ha aprofitat plenament.

L'anàlisi d'hàbitats d'imatge s'ha proposat com una eina per capturar la
heterogeneïtat tumoral mitjançant la partició dels tumors en subregions
espacials amb propietats d'imatge similars (hàbitats). La majoria dels
estudis d'hàbitats d'imatge es basen en la ressonància magnètica
multiparamètrica (RMmp), els mapes quantitatius de la qual són
biològicament interpretables, mentre que les aplicacions en TC continuen
sent limitades malgrat el seu ampli ús clínic. A més, els estudis
existents rarament avaluen la robustesa de les característiques
radiómiques derivades de la TC, sovint defineixen els hàbitats
mitjançant una optimització purament basada en dades sense una base
biològica, i generalment no informen sobre la seva rellevància clínica
més enllà del volum tumoral.

Aquesta tesi aborda aquestes limitacions investigant si la TC de rutina
pot capturar una heterogeneïtat intratumoral biològica i clínicament
rellevant en les metàstasis hepàtiques del càncer colorectal. En primer
lloc, vam identificar 26 característiques radiómiques adequades per al
càlcul robust d'hàbitats basats en TC, segons criteris de repetibilitat
i reproductibilitat. En segon lloc, vam desenvolupar un model d'hàbitats
en TC amb ancoratge biològic incorporant la RMmp coregistrada com a
referència durant la definició dels hàbitats, en lloc de basar-nos
exclusivament en l'optimització estadística. Dins d'aquest marc, vam
comparar múltiples representacions de la TC, incloent-hi
característiques radiómiques artesanals i representacions obtingudes
mitjançant aprenentatge profund, i vam observar que les característiques
artesanals produïen hàbitats biològicament més coherents. Els tres
hàbitats resultants reflectien l'arquitectura vascular: un nucli
avascular, una zona intermèdia cel·lular i perfosa, i un marge extern
altament vascularitzat.

En tercer lloc, vam avaluar la rellevància clínica analitzant
l'associació entre mètriques derivades dels hàbitats i els resultats
clínics en dues cohorts independents. Les mètriques d'hàbitat van
proporcionar informació pronòstica més enllà del volum tumoral, però
només en contextos terapèutics específics. En particular, l'entropia de
l'hàbitat a la interfície tumor--fetge va predir la supervivència en
situacions en què el tractament pot alterar la composició tissular sense
induir canvis mesurables en la mida. En tots els contextos terapèutics,
la informació pronòstica es va localitzar de manera consistent al marge
tumoral invasiu, en lloc de distribuir-se uniformement a tota la lesió.

En conjunt, aquesta tesi aporta avenços tant metodològics com clínics:
un \emph{pipeline} d'imatge d'hàbitats en TC de codi obert, una
avaluació exhaustiva de la robustesa de les característiques radiómiques
artesanals, la primera comparació de representacions de TC per al càlcul
d'hàbitats, l'ancoratge biològic dels hàbitats derivats de la TC
utilitzant RMmp i la demostració de la seva rellevància clínica
dependent del context. En conjunt, aquests resultats estableixen que les
TC de rutina contenen informació clínicament rellevant sobre la
heterogeneïtat tumoral que les estratègies actuals d'avaluació no
capten, i proporcionen un marc per a la seva extracció.

\vspace{1cm}

\noindent\textbf{Paraules clau:} metàstasis hepàtiques, càncer colorectal, anàlisi
d'hàbitats d'imatge, heterogeneïtat intratumoral, radiòmica, tomografia
computada, ressonància magnètica multiparamètrica, oncologia de precisió

\cleardoublepage
