% ============================================================================
% ACKNOWLEDGEMENTS
% ============================================================================

\chapter*{Acknowledgements}
\phantomsection  % Fix hyperref anchor for TOC link
\addcontentsline{toc}{chapter}{Acknowledgements}

Cuando pienso en todas las personas que han ayudado a que este documento
cobre vida durante los últimos cuatro años, me doy cuenta de que las
palabras no bastan para expresar plenamente la gratitud que siento. Lo
que sigue es, por tanto, un agradecimiento necesariamente incompleto.

Gracias a mi directora de tesis, \textbf{Raquel Pérez-López}, por
``quererme con o sin beca'', por confiar en mí en un momento pospandemia
en el que muchas puertas estaban cerradas y por darme la oportunidad de
crecer con libertad. Admiro profundamente tu capacidad de trabajo y no
puedo esperar a ver todo lo que el grupo logrará en las próximas
décadas.

Thank you to my former co-director, \textbf{Kinga Bernatowicz}.
Beginnings are hard, and you were the first person to patiently teach me
what habitats were. Your early enthusiasm laid the foundation for
everything that followed.

I am eternally grateful to \textbf{Francesco Grussu}. You treated me as
a scientific peer from day one, encouraged me to submit habitats
research to conferences, and spent countless hours patiently discussing
experiments with me and teaching me how to write convincingly about
them. You pushed me to think deeper, first by sitting next to you in the
early days (one learns a lot by watching how someone works), and later
by working together. In every single meeting, you managed two small
miracles: convincing me that the work I was doing was important, and
convincing me that I could do it. You are a true mentor, and any future
impact of my research will be, in large part, a testament to your
guidance. E, a proposito, il mio cervello è infinitamente migliore dopo
averti incontrato, ma il mio fegato\ldots{} forse meno.

Gracias a \textbf{Marta Ligero}, por allanar el camino para los
radiodoctorandos y por ser la primera mano amiga.\\
Gracias a \textbf{Alonso García}, por hacer del laboratorio un lugar más
generoso y humano, y por cuidarme con palabras y memes a partes
iguales.\\
Thank you to \textbf{Christina Zatse}, for consistently and kindly
reminding me that I am loved.

Cuando una escoge hacer el doctorado, elige sobre todo proyecto y
supervisión, pero no a sus compañeros de laboratorio. En este sentido,
me considero extraordinariamente afortunada. Gracias de corazón al
\textbf{grupo de Radiomics}, pasado y presente: \textbf{Adrià Marcos,
Anna Voronova, Athanasios Grigoriou, Bente Gielen, Camilo Monreal,
Carlos Macarro, Caterina Tozzi, Daniel Navarro, Ella Fokkinga,
Humaira Abdul, Iker Zubieta, Ingrid Cayuela, Laia Coronas, Luz Atlagich, Maria
Balaguer, Marta Buetas, Nikos Staikoglou, Óscar Llorian} and
\textbf{Rosa Melano.} Thank you for the radiobeers, VHIOFindes,
Valladolid, Amsterdam, Porto, Vienna, the cantine lunches, the Christmas
dinners, and the endless discussions in lab meetings and project
meetings, where much of this thesis was born. Every idea described here
was the product of robust discussion between us. Your collective wisdom
has been invaluable in shaping both my research and my growth as a
scientist. Thank you for the million memories I will carry with me
forever.

My work would not have been possible without many collaborators:
\textbf{Emily Latacz, Garazi Serna, Javi Ros, Kate Connor, Maria Teresa
Salcedo, Mireia Sanchís, Nadia Saoudi, Nuria Ruiz, Peter Vermeulen,
Raquel Comas, Víctor Navarro,} and \textbf{Zynya Calixto}. Their
contributions have been truly invaluable. Special thanks to
\textbf{Francesc Salvà}, for always being available and for insisting
that good research starts with the right clinical questions.

I am also deeply grateful to all the people at \textbf{VHIO} who
influenced me through stimulating conversations and shared their vision,
making it a truly special place to do a PhD. This includes \textbf{VHIO
Academy}, especially \textbf{Imma Falero}, thank you for always
listening and for the coffees at Noni. Thank you to \textbf{Alex Mur, Agustín
Sánchez, Andreu Òdena, Ana Dueso, Andrea Herencia, Ariadna Grinyó,
Carmen Escudero, Cayetano Galera, Débora Cabot, Ester Aguado, Iñigo
González, Maria López, Marion Martínez, Mariona Cubells, Paula
Carrillo,} and \textbf{Setareh Kompanian}, for friendship, knowledge,
and adventure. It has been an honor and a privilege to get to know you
all. Special thanks to \textbf{Ariadna} and \textbf{Marion} for the
early days of the PhD council and for thinking together about how to
build a better community.

Thank you to the \textbf{Mireia Crispin-Ortuzar lab} at the University
of Cambridge for keeping me warm during the Cambridge autumn and winter.
Thank you to \textbf{Inés Machado}, for the Mediterranean vibe;
\textbf{Rebecca Wray}, for running under the rain while having most
interesting conversations; and \textbf{Emanuela Greco}, for hosting me
without me ever having to ask twice. And a major thank you to
\textbf{Fabio Giuntini}, for turning a shared flat into a home and for
feeding me for three months.

Gracias a la \textbf{Fundación ``la Caixa''} por financiar la mayor
parte de esta investigación y por darme la oportunidad de asistir a
congresos donde tanto aprendí. Gracias también a todas las personas que
me animaron y ayudaron a pedir la beca cuando estaba sin trabajo. En
especial, gracias a mi amiga \textbf{Sara Barrera}, porque nunca tuviste
ninguna duda.

Gracias a mi amiga más cara, \textbf{Ares Anfruns}, por hacer tan bien
tu trabajo.

Thank you to my faraway friends for the visits: \textbf{Danielle Sher,
Emma Lower, Fang Wen Lim, Gabe Velenosi, Guillaume Varvoux, Jorge Zaera,
Lindsey Milisits, Marie Rajon} and \textbf{Rushabh Malde.} Special
mention to \textbf{Rushabh, Fang, and Jorge}, for the cookies and the
love\ldots{} ten years and still counting.

Gracias a mis amigos de toda la vida por acompañarme sin condiciones y
por las aventuras que siempre me recuerdan de dónde vengo:
\textbf{Albert Lázaro, Alex Paredes, Blanca Fornesa, Blanca Montamat,
Beltrán Jiménez, Laura Martín, Natalia Ollé, Núria Rodríguez, Pablo
Jimeno, Pat Balaguer, Rafael Jaén,} y \textbf{Sara Alberch.}

Gràcies a la cinquena promoció d'Enginyeria Biomèdica de la Universitat
de Barcelona: \textbf{Àlex Fajas, Alba Iruela, Anna Graell, Daniel
Rodríguez, Enrique Vilalta, Genís Esplugas, Gerard Trias, Helena
Briegas, Joan Ortiz, Judit Giró, Júlia Soler, Laura Escot, Mariona Ruiz,
Marta Pérez, Marta Trullols,} i \textbf{Umi Mir.} Vau convertir la
universitat en una cosa molt més gran que una etapa acadèmica, i entre
riures vau construir una base sòlida sobre la qual es sosté gran part
del que sóc avui.

Por último, quiero agradecer especialmente esta tesis a mi familia.

Gràcies, \textbf{Padrí i Irene}, per totes les visites al Papi i per ser un
model de la vida que vull tenir.\\
A \textbf{Laura y Pat}, por aprender a ser adultas juntas.\\
A \textbf{Camila y Nala}, por el amor incondicional.\\
A mis cuñadas y cuñados, \textbf{Mari Carmen, Adri, Juana Mari y Miguel
Tercero}, gracias por hacerme tía y madrina y por enseñarme tanto. Si
llego a saber que me haríais un sobrino por año de tesis, lo alargo todo
un par de años más. Gracias \textbf{Héctor, Lluc y Leo}, por llenarnos
la vida de alegría.\\
A mis suegros, \textbf{Mari y Miguel}, gracias por apoyarme en los
buenos y malos momentos, por quererme tanto y tan bien, y por darme la
oportunidad de empezar un hogar con vuestro hijo.\\
Al meu cunyat estimat, \textbf{Enric}, per totes les converses
interessants i per arrencar-me somriures, sobretot amb cafès a Pamplona
i dièsel a Saragossa.\\
Gracias a mi hermana \textbf{Virginia}, por ser siempre el lugar donde
pensar en voz alta.\\
Gracias a mis padres, \textbf{María Gracia y Enrique}. Seguramente lo
más relevante que haga en mi carrera profesional es ser hija vuestra.
Gracias por inculcarme los valores de esfuerzo, respeto, honestidad y
excelencia. Treballar dur en una cosa que m'agrada i intentar fer-ho bé
és una manera de viure en la qual sempre em sento a prop teu, Papi.\\
Gracias a mi futuro (ex)marido, \textbf{Miguel}, por aguantarme en todas
mis versiones. Me alegra saber que tengo el resto de mi vida para 
devolverte todo lo que te debo, porque necesitaré como mínimo todo 
ese tiempo.

No se me habría ocurrido hacer un doctorado si no hubiera tenido la
suerte de descubrir la ciencia por primera vez de la mano de profesoras
maravillosas. Qué importante es tener docentes en el colegio que guían,
despiertan la curiosidad y plantan una semilla que crece con los años.
Gracias a \textbf{Julie Connolly} (Science), \textbf{Hortènsia Mallén}
(Biología), \textbf{Gloria López-Barrena} (Química) y \textbf{Ana
Fuertes} (Física). No me olvido de vuestras clases.

\vspace{1cm}

\begin{flushright}
Olivia Prior\\
Barcelona, enero de 2026
\end{flushright}
