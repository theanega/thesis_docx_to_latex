\chapter{Image Acquisition Details}\label{annex:image-acquisition-details}

\section{MRI Acquisition Details (PREDICT)}\label{sec:A.1}

Patients were scanned on either a 1.5T Siemens Avanto or a 3T GE SIGNA
Pioneer system. Each patient was scanned on only one of the two
scanners. The protocol included anatomical imaging (T2-weighted and
T1-weighted), diffusion MRI, variable flip angle spoiled gradient echo
(SGrE) imaging for T1 mapping, and dynamic contrast-enhanced (DCE) MRI.

\subsubsection*{1.5T Siemens Avanto system}\label{t-siemens-avanto-system}

The protocol included high-resolution anatomical T2w and T1w scans,
diffusion MRI and different spoiled gradient echo (SGrE) sequences, such
as those for T1 mapping and dynamic contrast enhanced (DCE) MRI.

\begin{itemize}
\item
  Anatomical T2w scan: turbo spin echo, TE = 82 ms, TR = 4500 ms, turbo
  factor of 29, echo spacing 8.2 ms, NEX = 8, 2 concatenations,
  resolution of 1.4mm × 1.4mm, slice thickness of 5 mm, GRAPPA = 2,
  acquisition in free breathing.
\item
  Anatomical T1w scan: turbo spin echo, TE = 6.3 ms, TR = 470 ms, turbo
  factor of 11, echo spacing 6.26 ms, NEX = 6, 6 concatenations,
  resolution of 1.4mm × 1.4mm, slice thickness of 5 mm, GRAPPA = 2,
  acquisition in free breathing.
\item
  Diffusion MRI: single-shot twice-refocused spin echo EPI, b = \{0, 50,
  100, 400, 900, 1200, 1600\} s/mm\textsuperscript{2}, TR = 7900 ms,
  averaging of 3 mutually-orthogonal directions, NEX = 2, 1
  concatenation, resolution of 1.9mm × 1.9mm, slice thickness of 6 mm,
  SPAIR fat suppression, GRAPPA = 2, EPI factor 150, echo spacing 0.82
  ms, each b-value acquired at TE = \{93 ms, 105 ms, 120 ms\},
  acquisition in free breathing. Additionally, one b = 0 image at TE =
  93 ms was acquired with reversed phase encoding polarity.
\item
  SGrE for T1 mapping: FLASH, TE = 1.76 ms, TR = 4.59 ms, NEX = 1, 1
  concatenation, resolution of 2.7mm × 2.7mm, slice thickness of 6 mm,
  flip angles of \{5°, 15°, 20°\}, GRAPPA = 2, acquisition in free
  breathing.
\item
  SGrE for DCE: same acquisition as for T1 mapping with fixed flip angle
  of 15°; 26 dynamic acquisitions with temporal resolution of 10s,
  Gadovist with dose of 0.5ml/Kg injected at 3ml/s followed by a bolus
  of physiological solution of 20ml at 3ml/s, injection delay of 10s,
  acquisition in free breathing.
\end{itemize}

\subsubsection*{3T GE SIGNA Pioneer system}\label{t-ge-signa-pioneer-system}

The protocol included high-resolution anatomical T2w and T1w scans,
diffusion MRI and different spoiled gradient echo (SGrE) sequences, as
those for T1 mapping and DCE imaging.

\begin{itemize}
\item
  Anatomical T2w scan: fast spin echo, TE = 50 ms, TR = 3750 ms, turbo
  factor of 16, NEX = 2, resolution of 1.4mm × 1.4mm, slice thickness of
  6 mm, respiratory-gated acquisition.
\item
  Anatomical T1w scan: navigated SGrE LAVA-Flex providing water/fat
  images, TE = 2.60 ms, TR = 5.38 ms, resolution of 1.4mm × 1.4mm, slice
  thickness of 6 mm, flip angle of 12°, acquisition in free-breathing
  after liver motion measurement.
\item
  Diffusion MRI: single-shot pulsed gradient spin echo EPI, b = \{0, 50,
  100, 400, 900, 1200, 1500\} s/mm\textsuperscript{2}, TR = 3500ms,
  averaging of 3 mutually-orthogonal directions, NEX = 2, resolution of
  2.4mm × 2.4mm, slice thickness of 6 mm, ASPIR fat suppression, ASSET =
  2, echo spacing 0.80 ms, each b-value acquired at TE = \{75 ms, 90 ms,
  105 ms\}, respiratory-gated acquisition.
\item
  SGrE for T1 mapping: LAVA, TE = 1.2 ms, TR = 2.72 ms, NEX = 1,
  resolution of 2.4mm × 2.4mm; slice thickness of 6 mm; flip angles of
  \{5°, 10°, 15°\}, ASSET = 2, acquisition of two separate images in
  breathhold, acquisition of the vendor's B1 map.
\item
  SGrE for DCE: same acquisition as for T1 mapping with fixed flip angle
  of 12°; 26 dynamic acquisitions with temporal resolution of 10s,
  Clariscan 0.5 mmol/ml with dose of 0.2ml/Kg injected at of 0.5ml/kg at
  3ml/s followed by a bolus of physiological solution of 20ml at 3ml/s,
  injection delay of 10s, acquisition in free breathing.
\end{itemize}

\section{mpMRI Maps Biological Ranges}\label{sec:A.2}

\begin{table}[htbp]
\centering
\small
\caption[mpMRI maps derived with biological ranges.]{\textbf{mpMRI maps derived with biological ranges.} Thirteen
quantitative maps were derived from diffusion-relaxation MRI, variable
flip angle T1 mapping, and dynamic contrast-enhanced MRI.}
\label{tab:A.1}
\begin{tabular}{@{}llllp{3cm}@{}}
\toprule
\multicolumn{2}{c}{\textbf{mpMRI metric}} & \textbf{Computed from} & \textbf{Units} & \textbf{Biological Range} \\
\midrule
ADCₜ & Tissue ADC & Diffusion-relaxation MRI & μm²/ms & 0.0--3.0 \\
ADCᵥ & Vascular ADC & Diffusion-relaxation MRI & μm²/ms & 3.0--150.0 \\
Kₜ & Tissue kurtosis excess & Diffusion-relaxation MRI & Dimensionless & 0.0--5.0 \\
fᵥ & Vascular signal fraction & Diffusion-relaxation MRI & Normalized & 0.0--1.0 \\
T₂ₜ & Tissue T₂ & Diffusion-relaxation MRI & ms & 0.0--800.0 \\
D₀ & Intrinsic diffusivity & Advanced diffusion model & μm²/ms & 0.8--3.0 \\
vCS & Volume-weighted cell size & Advanced diffusion model & μm & 5.0--40.0 \\
fᵢₙ & Intracellular fraction & Advanced diffusion model & Normalized & 0.0--1.0 \\
CD & Cell density & Advanced diffusion model & Cells/mm³ & 0.0--8000000.0 \\
T₁ & T₁ & Variable flip angle SGrE & ms & 0.0--5000.0 \\
T2\textsuperscript{*} & T2\textsuperscript{*} & Multiecho SGrE & ms & 0.0--200.0 \\
K\textsuperscript{trans} & Capillary permeability & DCE MRI & min⁻¹ & 0.0--4.0 \\
vₑ & Extracellular-extravascular volume & DCE MRI & Normalized & 0.0--1.0 \\
\bottomrule
\end{tabular}
\end{table}
