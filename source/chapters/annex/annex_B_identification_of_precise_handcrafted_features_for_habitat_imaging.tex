\chapter{Identification of Precise Handcrafted Features}\label{annex:identification-of-precise-handcrafted-features-for-habitat-imaging}

\section{Cohort Characteristics}\label{sec:B.1}

\begin{table}[ht]
    \centering
    \small
    \caption[Total number of patients, images, and lesions per cohort and lesion location]{\textbf{Total number of patients, images, and lesions per cohort and lesion location.}}
    \label{tab:annex_totals}
    \begin{tabular}{l c c c c c c}
        \toprule
        \textbf{Primary tumor} & \multicolumn{2}{c}{\textbf{Patients}} & \multicolumn{2}{c}{\textbf{Images}} & \multicolumn{2}{c}{\textbf{Lesions}} \\
        \cmidrule(lr){2-3} \cmidrule(lr){4-5} \cmidrule(lr){6-7}
        & \textbf{Liver} & \textbf{Lung} & \textbf{Liver} & \textbf{Lung} & \textbf{Liver} & \textbf{Lung} \\
        \midrule
        Colorectal      & 63 & 12 & 186 & 29 & 959 & 122 \\
        Lung            & 13 & 72 & 22 & 102 & 89 & 141 \\
        Neuroendocrine  & 86 & 0  & 86 & 0   & 447 & 0   \\
        Mixed           & 44 & 41 & 93 & 87  & 366 & 312 \\
        \midrule
        \textbf{Total}  & 206 & 125 & 387 & 218 & 1861 & 575 \\
        & \multicolumn{2}{c}{\textbf{331}} & \multicolumn{2}{c}{\textbf{605}} & \multicolumn{2}{c}{\textbf{2436}} \\
        \bottomrule
    \end{tabular}
\end{table}

\begin{table}[ht]
    \centering
    \small
    \caption[Image acquisition parameters per cohort]{\textbf{Image acquisition parameters per cohort.} (*) Median [IQR]}
    \label{tab:annex_acquisition}
    \begin{tabularx}{\textwidth}{@{} l c c c c @{}}
        \toprule
        & \textbf{\makecell{Colorectal\\(n=215)}} & \textbf{\makecell{Lung\\(n=124)}} & \textbf{\makecell{Neuroendocrine\\(n=86)}} & \textbf{\makecell{Mixed\\(n=180)}} \\
        \midrule
        \textbf{Manufacturers} & & & & \\
        \makecell[l]{SIEMENS/PHILIPS/\\TOSHIBA/GE} & \makecell{138/58/9/10} & \makecell{63/40/0/21} & \makecell{22/35/6/23} & \makecell{144/23/3/10} \\ \addlinespace
        \textbf{Tube Voltage (kVP)} & & & & \\
        \makecell[l]{100/110/120/\\130/140/unknown} & \makecell{25/17/161/\\0/0/12} & \makecell{6/1/105/\\3/0/9} & \makecell{10/3/70/\\1/2/0} & \makecell{14/7/158/\\0/1/0} \\ \addlinespace
        \textbf{Recon. kernel} & & & & \\
        \makecell[l]{SOFT/STANDARD/B\\B20f/B30f/B31f\\I31s/I50s/unknown} & \makecell{1/7/96\\16/38/12\\15/6/20} & \makecell{14/7/43\\11/23/0\\0/0/22} & \makecell{1/19/36\\2/2/4\\0/0/19} & \makecell{2/11/21\\9/112/4\\0/0/21} \\ \addlinespace
        \textbf{Slice thickness (mm)*} & 2.0 [2.0--5.0] & 2.5 [2.0--5.0] & 2.0 [2.0--3.0] & 5.0 [1.0--5.0] \\ \addlinespace
        \textbf{Pixel spacing (mm)*} & 0.92 [0.77--0.98] & 0.91 [0.81--0.98] & 0.75 [0.70--0.82] & 0.98 [0.82--0.98] \\ 
        \bottomrule
    \end{tabularx}
\end{table}

\begin{table}[ht]
    \centering
    \small
    \caption[List of primary tumor types included within the mixed cohort]{\textbf{List of primary tumor types included within the mixed cohort.}}
    \label{tab:annex_mixed_cohort}
    \begin{tabular}{l c c c c c c}
        \toprule
        \textbf{Primary tumor} & \multicolumn{2}{c}{\textbf{Patients}} & \multicolumn{2}{c}{\textbf{Images}} & \multicolumn{2}{c}{\textbf{Lesions}} \\
        \cmidrule(lr){2-3} \cmidrule(lr){4-5} \cmidrule(lr){6-7}
        & \textbf{Liver} & \textbf{Lung} & \textbf{Liver} & \textbf{Lung} & \textbf{Liver} & \textbf{Lung} \\
        \midrule
        Adrenal         & 3 & 0 & 5 & 0 & 35 & 5 \\
        Biliary Tract   & 11 & 5 & 24 & 11 & 71 & 66 \\
        Bladder         & 3 & 3 & 5 & 5 & 41 & 20 \\
        Bone            & 0 & 1 & 0 & 3 & 0 & 21 \\
        Breast          & 4 & 2 & 9 & 6 & 32 & 6 \\
        Cervix          & 2 & 2 & 3 & 3 & 21 & 21 \\
        Esophagus       & 1 & 2 & 2 & 4 & 8 & 9 \\
        Head \& Neck    & 2 & 4 & 4 & 11 & 13 & 13 \\
        Kidney          & 1 & 2 & 2 & 3 & 4 & 18 \\
        Liver           & 2 & 1 & 3 & 1 & 11 & 1 \\
        Ovary           & 1 & 2 & 2 & 4 & 4 & 26 \\
        Pancreas        & 2 & 0 & 4 & 0 & 12 & 0 \\
        Penis           & 1 & 0 & 3 & 0 & 9 & 0 \\
        Skin            & 6 & 11 & 13 & 19 & 68 & 39 \\
        Stomach         & 4 & 0 & 10 & 0 & 29 & 0 \\
        Thymus          & 1 & 0 & 4 & 0 & 8 & 0 \\
        Thyroid         & 0 & 6 & 0 & 17 & 0 & 52 \\
        \midrule
        \textbf{Total}  & 44 & 41 & 93 & 87 & 366 & 312 \\
        & \multicolumn{2}{c}{\textbf{85}} & \multicolumn{2}{c}{\textbf{180}} & \multicolumn{2}{c}{\textbf{678}} \\
        \bottomrule
    \end{tabular}
\end{table}

\clearpage

\section{Radiomics Features and Computation}\label{sec:B.2}

\begin{table}[ht]
    \centering
    \scriptsize 
    \caption[List of radiomics features analyzed in this study]{\textbf{List of radiomics features analyzed in this study.} 93 voxel-wise features were computed. 91 were analyzed after excluding GLCM\_MCC and FirstOrder\_TotalEnergy. Feature definitions are available in the IBSI reference manual \citep{zwanenburgImageBiomarkerStandardization2020}.}
    \label{tab:annex_features}
    \begin{tabularx}{\textwidth}{@{} l >{\raggedright\arraybackslash}X | l >{\raggedright\arraybackslash}X @{}}
        \toprule
        \textbf{Class} & \textbf{Feature} & \textbf{Class} & \textbf{Feature} \\
        \midrule
        \textbf{First Order} & 10Percentile, 90Percentile, Energy, Entropy, InterquartileRange, Kurtosis, Maximum, MeanAbsoluteDeviation, Mean, Median, Minimum, Range, RobustMeanAbsoluteDeviation, RootMeanSquared, Skewness, TotalEnergy, Uniformity, Variance & 
        \textbf{GLRLM} & GrayLevelNonUniformity, GrayLevelNonUniformityNorm., GrayLevelVariance, HighGrayLevelRunEmphasis, LongRunEmphasis, LongRunHighGrayLevelEmphasis, LongRunLowGrayLevelEmphasis, LowGrayLevelRunEmphasis, RunEntropy, RunLengthNonUniformity, RunLengthNonUniformityNorm., RunPercentage, RunVariance, ShortRunEmphasis, ShortRunHighGrayLevelEmphasis, ShortRunLowGrayLevelEmphasis \\
        \midrule
        \textbf{GLSZM} & GrayLevelNonUniformity, GrayLevelNonUniformityNorm., GrayLevelVariance, HighGrayLevelZoneEmphasis, LargeAreaEmphasis, LargeAreaHighGrayLevelEmphasis, LargeAreaLowGrayLevelEmphasis, LowGrayLevelZoneEmphasis, SizeZoneNonUniformity, SizeZoneNonUniformityNorm., SmallAreaEmphasis, SmallAreaHighGrayLevelEmphasis, SmallAreaLowGrayLevelEmphasis, ZoneEntropy, ZonePercentage, ZoneVariance &
        \textbf{GLDM} & DependenceEntropy, DependenceNonUniformity, DependenceNonUniformityNorm., DependenceVariance, GrayLevelNonUniformity, GrayLevelVariance, HighGrayLevelEmphasis, LargeDependenceEmphasis, LargeDependenceHighGrayLevelEmphasis, LargeDependenceLowGrayLevelEmphasis, LowGrayLevelEmphasis, SmallDependenceEmphasis, SmallDependenceHighGrayLevelEmphasis, SmallDependenceLowGrayLevelEmphasis \\
        \midrule
        \textbf{GLCM} & Autocorrelation, ClusterProminence, ClusterShade, ClusterTendency, Contrast, Correlation, DifferenceAverage, DifferenceEntropy, DifferenceVariance, Id, Idm, Idmn, Idn, Imc1, Imc2, InverseVariance, JointAverage, JointEnergy, JointEntropy, MCC (Excluded), MaximumProbability, SumAverage, SumEntropy, SumSquares & 
        \textbf{NGTDM} & Busyness, Coarseness, Complexity, Contrast, Strength \\
        \bottomrule
    \end{tabularx}
\end{table}

\begin{table}[ht]
    \centering
    \small
    \caption[Image processing and radiomics feature computation parameters]{\textbf{Image processing and radiomics feature computation parameters.}}
    \label{tab:annex_parameters}
    \begin{tabularx}{\textwidth}{@{} l >{\raggedright\arraybackslash}X @{}}
        \toprule
        \textbf{Parameter} & \textbf{Value/Description} \\
        \midrule
        \multicolumn{2}{l}{\textbf{Image Processing}} \\
        \midrule
        Software & PyRadiomics v3.0.1, installed in Python 3.7.10 \\
        Bounding box & Defined by the segmentation, extended by default padding distance. \\
        Resampled voxel spacing & 1 x 1 x 1 mm \\
        Image interpolation & B-spline \\
        Intensity rounding & None \\
        ROI interpolation & Nearest neighbor \\
        Resegmentation & None \\
        \midrule
        \multicolumn{2}{l}{\textbf{Feature Computation}} \\
        \midrule
        Kernel radius & 1 / 3 mm \\
        Discretization & 12 / 25 HU (Fixed Bin Size) \\
        Image filter & None \\
        maskedKernel & True (only voxels in kernel also in ROI used) \\
        Initvalue & NaN (voxels outside ROI considered transparent) \\
        Distance weighting & No weighting \\
        GLCM Symmetry & Symmetric \\
        Distance Metrics & Chebyshev distance $\delta = 1$ \\
        NGTDM Coarseness & Coarseness parameter $\alpha = 0$ \\
        \bottomrule
    \end{tabularx}
\end{table}

\clearpage

\section{Image Perturbation}\label{sec:B.3}

Image perturbation was carried out in three ways: rotation, translation,
and noise addition. While the first two emulate changes in patient
positioning, the latter represents the noise present in different voxel
intensities in CT images. Perturbations were performed as described in
\citet{bernatowiczRobustImagingHabitat2021}, where the authors demonstrated that the
combination of these three perturbations simulate the retest scenario.
Briefly, we added Gaussian noise (mean 0, standard deviation as present
in the image); for translation, we shifted the voxel grid by a fraction
of the image voxel spacing following; finally, we rotated the image
around the z-axis by an angle of 0.5°.

\section{Habitat Computation}\label{sec:B.4}

To take into account intravoxel heterogeneity, we decided to choose a
probabilistic model, Gaussian Mixture Models (GMMs), for clustering
rather than a deterministic approach. GMMs, which have been previously
used in similar contexts \citep{chenUnsupervisedClusteringQuantitative2019, jardim-perassiMultiparametricMRICoregistered2019}, are generative probabilistic models that find a mixture of
multiple Gaussian probability distributions that best fit the data. The
Expectation-Maximization (EM) algorithm is used to estimate the model
parameters \citep{bishopPatternRecognitionMachine2006}. A GMM is represented by the following
formula:

\[P\ (x) = \sum_{}^{}(\pi_{k}\ N\ (x\ |\ \mu_{k},\ \sum_{k})\]

where

P(x) : probability density of the data point x

$\mathbf{\pi}_{\mathbf{k}}$: mixing coefficient for the kth Gaussian
component

$\mathbf{N}\ (\mathbf{x}\ |\ \mathbf{\mu}_{\mathbf{k}},\ \sum_{\mathbf{k}})$:
kth Gaussian component with mean $\mathbf{\mu}_{\mathbf{k}}$ and
covariance matrix $\sum_{\mathbf{k}}$

To determine the optimal number of habitats (k), we used the Bayesian
Information Criterion (BIC). The formula for BIC is:

\[BIC = \ {- \ 2\ log}{(L)\  + \ d\ log(n)}\]

where

L : likelihood of the data given the model

d: number of parameters

n: number of data points

The BIC score is a measure of the trade-off between model complexity and
goodness of fit. It penalizes models with more parameters, such as GMMs
with more clusters. In general, lower BIC scores indicate better model
fit. However, depending on data characteristics, a clear minimum in BIC
scores might not be observed and thus, the gradient can be used to
determine the optimal number of clusters. This was our case and
therefore we performed a GMM fit for different values of clusters (k):
\{2, 3, 4 and 5\}. The maximum number of 5 clusters was determined by
being the maximum number of tissue types observed in histology by an
experienced pathologist. The optimal value of k was the one where the
change in BIC score with respect to k was maximal, which was an
indication that adding more clusters after that point does not improve
the model fit significantly. A cluster number was automatically selected
by BIC using the precise original radiomics data and was given as a
parameter to the GMM model to compute imaging habitats in both the
original and perturbed data. GMM was implemented using Python package
scikit-learn (v1.0.2) with a random seed of 123, and default parameters
(except for the number of clusters), specifically maximum iteration of
100, convergence threshold of 10\textsuperscript{-3}, full covariance
type and initialization with kmeans.

In addition, The Hungarian algorithm (also known as the Kuhn-Munkres
algorithm) \citep{kuhnHungarianMethodAssignment1955}, was used to match habitats between original and
perturbed data. The Hungarian algorithm is a combinatorial optimization
algorithm that solves the assignment problem in polynomial time. It
finds an optimal one-to-one matching between two sets by minimizing the
total cost (in our case, the difference in cluster assignments).

Finally, to quantify habitat stability, we computed the Dice Similarity
Coefficient (DSC) \citep{zouStatisticalValidationImage2004} between original and perturbed
habitats for each habitat within a lesion, across all lesions. The DSC
is a widely used metric for evaluating the overlap between two sets,
with a higher DSC indicating greater similarity.

All codes are publicly available at
\url{https://github.com/radiomicsgroup/precise-habitats}.

\section{Intraclass Correlation Coefficient}\label{sec:B.5}

An Intraclass Correlation Coefficient (ICC) value of 1 indicates that a
feature is highly repeatable/reproducible whereas a value of 0 implies
no reliability. Negative ICC values were truncated at 0 as proposed and
done previously \citep{bartkoVariousIntraclassCorrelation1976, fornacon-woodReliabilityPrognosticValue2020}. The ICC is
calculated by mean squares obtained through the analysis of variance
(ANOVA). In this study, we use two versions of the ICC that are based on
a two-way mixed effects ANOVA model, following Koo's guidelines. Below
we describe the formulas used to compute the ICC formulas. More
information regarding such formulas can be found in the highly cited
paper from McGraw and Wong.

To compute the ANOVA model let's consider a dataframe with dimensions
$\mathbf{n}\ \times \mathbf{k}$ dataframe where $\mathbf{n}$ is the
total number of voxels (rows) for one region of interest (ROI) and
$\mathbf{k}$ is the total number of conditions or measurements
(columns). In our case, $\mathbf{k}$=2. For repeatability the two
conditions are original-perturbed (test-retest) and for reproducibility
against kernel size the two conditions are computation with radius
kernel 1mm or radius kernel 3mm (or bin size 12HU or 25HU in the case of
reproducibility against bin size). Each voxel measurement is indexed as
$\mathbf{Y}_{\mathbf{ij}}$ where \emph{i} denotes the voxel (\emph{i}
= 1, \ldots{} \emph{n)} and \emph{j} denotes the measurement under the
repeatability/reproducibility condition (\emph{j = 1 \ldots{} k)}. We
define the following concepts:

${\overline{\mathbf{Y}}}_{\mathbf{i}}$: mean of all voxel values in a
column

\[{\overline{\mathbf{Y}}}_{\mathbf{i}} = \frac{\sum_{\mathbf{j} = \mathbf{1}}^{\mathbf{k}}\mathbf{Y}_{\mathbf{ij}}}{\mathbf{k}}\]

${\overline{\mathbf{Y}}}_{\mathbf{j}}:$ mean of all voxel values in a
column

\[{\overline{\mathbf{Y}}}_{\mathbf{j}} = \frac{\sum_{\mathbf{i} = \mathbf{1}}^{\mathbf{n}}\mathbf{Y}_{\mathbf{ij}}}{\mathbf{n}}\]

$\mathbf{\mu}:$ mean of all values (also called \emph{grand mean})

\[\mathbf{\mu} = \frac{\sum_{\mathbf{j} = \mathbf{1}}^{\mathbf{k}}{\sum_{\mathbf{i} = \mathbf{1}}^{\mathbf{n}}\mathbf{Y}_{\mathbf{ij}}}}{\mathbf{n}*\mathbf{k}}\]

${\mathbf{\sigma}_{\mathbf{w}}}^{\mathbf{2}}:\ $Within-voxel variance,
the estimated variance of repeated measurements

\[{\mathbf{\sigma}_{\mathbf{w}}}^{\mathbf{2}} = \frac{\sum_{\mathbf{j} = \mathbf{1}}^{\mathbf{k}}{(\mathbf{Y}_{\mathbf{ij}} - {\overline{\mathbf{Y}}}_{\mathbf{i}})\ }^{\mathbf{2}}}{\mathbf{k} - \mathbf{1}}\]

$\mathbf{\sigma}_{\mathbf{w}}\ :\ \ $Within-voxel standard deviation,
the standard deviation we get if we measure the voxel multiple times.
Calculated by averaging the within-subject sample variances. Since we
have a variance per voxel and we can't meaningfully take the average of
a list of standard deviations, we first calculate the variance for each
voxel, and then compute the average of those, and finally square root
that mean variance \citep{yeStatisticalConsiderationsRepeatability2022}.

\[\mathbf{\sigma}_{\mathbf{w}} = \sqrt{\frac{\sum_{\mathbf{i} = \mathbf{1}}^{\mathbf{n}}\frac{\sum_{\mathbf{j} = \mathbf{1}}^{\mathbf{k}}{(\mathbf{Y}_{\mathbf{ij}} - {\overline{\mathbf{Y}}}_{\mathbf{i}})\ }^{\mathbf{2}}}{\mathbf{k} - \mathbf{1}}}{\mathbf{n}}}\]

The degrees of freedom, sum squares and mean square expectations that
correspond to a two-way mixed ANOVA model are summarized below.

\needspace{10\baselineskip}
\begin{table}[H]
\centering
\small
\caption[Two-way Mixed ANOVA Model]{\textbf{Two-way Mixed ANOVA Model.} MSC: mean square columns, MSR: mean square rows, MSE: mean square error, SSC= sum of squares columns, SSR=sum of squares rows, SST= sum of squares total, SSE= sum of squares error, dfc= degrees of freedom columns, dfr=degrees of freedom rows, dfe=degrees of freedom errors}
\label{tab:annex_anova_model}
\begin{tabular}{@{}p{3.5cm}p{2cm}p{4.5cm}p{4cm}@{}}
\toprule
\textbf{Source of Variation} & \textbf{Degrees of freedom} & \textbf{Sum Squares} & \textbf{Mean Square Expectations} \\
\midrule
Conditions (columns) & dfc = k -1 &
$\mathbf{SSC} = \ \sum_{\mathbf{j} = \mathbf{1}}^{\mathbf{k}}{\mathbf{n}\  \times \ {({\overline{\mathbf{Y}}}_{\mathbf{j}} - \mathbf{\mu})\ }^{\mathbf{2}}}$
&
$\mathbf{MSC} = \frac{\mathbf{SSC}}{\mathbf{dfc}\  \times \ \mathbf{n}\ }$ \\
Voxels (rows) & dfr = n -1 &
$\mathbf{SSR} = \ \sum_{\mathbf{i} = \mathbf{1}}^{\mathbf{n}}{\mathbf{k}\  \times \ {({\overline{\mathbf{Y}}}_{\mathbf{i}} - \mathbf{\mu})\ }^{\mathbf{2}}}$
& $\mathbf{MSR} = \frac{\mathbf{SSR}}{\mathbf{dfr}\ }$ \\
Total & &
$\mathbf{SST} = \ \sum_{\mathbf{j} = \mathbf{1}}^{\mathbf{k}}{\sum_{\mathbf{i} = \mathbf{1}}^{\mathbf{n}}{(\mathbf{Y}_{\mathbf{ij}} - \mathbf{\mu})\ }^{\mathbf{2}}\ }$
& \\
Error (or residual) & dfe = (n -1)(k -1) &
$\mathbf{SSE} = \mathbf{SST} - \mathbf{SSC} - \mathbf{SSR}$ &
$\mathbf{MSE} = \frac{\mathbf{SSE}}{\mathbf{dfe}\ }$ \\
\bottomrule
\end{tabular}
\end{table}



We compute the two versions of ICC for repeatability and
reproducibility:

Repeatability ICC(3A,1): ICC based on single-measurement,
absolute-agreement, two-way mixed-effects model.

\[\mathbf{ICC}\ (\mathbf{3A},\mathbf{1}) = \frac{\mathbf{MSR} - \mathbf{MSE}}{\mathbf{MSR} + \mathbf{dfc}\  \times \ \mathbf{MSE} + \ \frac{\mathbf{k}}{\mathbf{n}\ }\  \times (\mathbf{MSC} - \mathbf{MSE})\ \ }\]

Reproducibility ICC(3C,1): ICC based on single-measurement, consistency,
two-way mixed-effects model.

\[\mathbf{ICC}(\mathbf{3C},\mathbf{1}) = \frac{\mathbf{MSR} - \mathbf{MSE}}{\mathbf{MSR} + \mathbf{dfc} \times \ \mathbf{MSE}\ }\]

We compute the lower bound of the 95\% CI of the ICC (LCL) and the upper
bound (UCL):

\[\mathbf{LCL =}\frac{\frac{\mathbf{FR}}{\mathbf{F}}\mathbf{- 1}}{\frac{\mathbf{FR}}{\mathbf{F}}\mathbf{+ k - \ 1}}\mathbf{\ \ \ \ \ \ \ \ \ \ \ \ \ \ \ \ \ \ \ \ \ \ \ \ \ \ \ \ \ \ \ \ \ \ \ \ \ \ \ \ \ \ \ \ \ \ \ \ \ \ \ \ \ \ \ \ \ \ \ \ \ \ \ \ \ \ \ \ \ \ \ \ \ \ \ \ \ \ \ \ \ \ \ \ UCL =}\frac{\mathbf{(FR\  \times \ F)\  - 1}}{\mathbf{(FR\  \times \ F) + k - \ 1}}\]

Where F is the (1-$\frac{\mathbf{\alpha}}{\mathbf{2}}$ ) x
100\textsuperscript{th} percentile of the F distribution with n-1
numerator degrees of freedom and (n-1)(k-1) denominator degrees of
freedom and FR is the F-statistic for voxels computed as:
$\mathbf{FR =}\frac{\mathbf{MSR}}{\mathbf{MSE}}$.

Custom codes used to calculate ICC (3A,1) and ICC (3C,1), based
on Nipype's \citep{estebanNipyNipype1832022} module ICC (v1.8.5) and approved by
a statistician (VN) are available at
\url{https://github.com/radiomicsgroup/precise-habitats}.

\section{Justification for NGTDM Coarseness Inclusion}\label{sec:B.6}

NGTDM (Neighborhood Gray-Tone-Difference Matrix) coarseness describes
the roughness (i.e. how fine or coarse) the texture of an image is. In
the radiomics literature, evidence has been found regarding its
usefulness to characterize heterogeneity and predict progression-free
survival in oncology \citep{guptaPredictingProgressionFreeSurvival2021}.

In our study, we identified precise features by linking repeatability
and reproducibility results. That is, for every feature, we considered
results obtained in the three relevant experiments: repeatability
(setting R3B12), reproducibility against R (fixed B=12HU), and
reproducibility against B (fixed R=3mm). A feature was selected as
precise if it presented LCL ≥ 0.50 (i.e. moderate, good or excellent
repeatability/reproducibility) in the three experiments. NGTDM
Coarseness presented excellent repeatability and reproducibility against
bin size, but was not selected as precise as it presented poor
reproducibility against kernel radius. However, by the nature of its
definition, the poor reproducibility against kernel radius is
acceptable: the feature captures the distribution of differences in
gray-tone values between pairs of neighboring pixels. Considering the
feature's excellent results in two out of three experiments, its
potential usefulness and in light of the fact that we were already being
stringent, first by using LCL rather than ICC and second by linking
results of three different experiments, we decided to include it as a
precise feature for both liver and lung lesions.

\clearpage
\section{Primary Tumor Has No Effect on Precision}\label{sec:B.7}

\begin{figure}[htbp]
\centering
\includegraphics[width=0.95\textwidth]{fig_B_1.png}
\caption[Repeatability by primary tumor type]{\textbf{Repeatability by primary tumor type.} Repeatability distribution of radiomics features computed with setting R1B12 (A), R1B25 (B), R3B12 (C) and R3B25 (D) per cohort for lung and liver lesions separately. Primary tumor has no effect on repeatability. LCL, 95\% lower confidence limit of the Intraclass Correlation Coefficient; R1B12, features computed with kernel radius 1mm and bin size 12HU; R1B25, features computed with kernel radius 1mm and bin size 25HU; R3B12, features computed with kernel radius 3mm and bin size 12HU; R3B25, features computed with kernel radius 3mm and bin size 25HU; CRC: colorectal cohort; NET: neuroendocrine cohort; ALL: all cohorts combined.}
\label{fig:B.1}
\end{figure}

\begin{figure}[htbp]
\centering
\includegraphics[width=0.95\textwidth]{fig_B_2.png}
\caption[Reproducibility by primary tumor type]{\textbf{Reproducibility by primary tumor type.} Reproducibility distribution against R of radiomics features computed with fixed bin size of 12HU (A) and fixed bin size of 25HU (B) per cohort for lung and liver lesions separately. Similary, (C) and (D) depict the reproducibility distribution against B of radiomics features computed with fixed radius of 1mm (C) and 3mm (D) per cohort for lung and liver lesions separately. LCL, 95\% lower confidence limit of the Intraclass Correlation Coefficient; CRC: colorectal cohort; NET: neuroendocrine cohort; ALL: all cohorts combined.}
\label{fig:B.2}
\end{figure}
