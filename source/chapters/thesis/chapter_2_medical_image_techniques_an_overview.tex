\chapter{Medical Image Techniques: An
Overview}\label{medical-image-techniques-an-overview}\label{ch:2}

Imaging guides clinical decisions at nearly every stage of cancer care.
The modalities most relevant to this thesis are computed tomography
(CT), magnetic resonance imaging (MRI), and histopathology. This
chapter describes how each technique generates images and captures
tumors at different scales. A note on terminology: because radiological
images are 3D volumes, the basic unit of measurement is the voxel (a
pixel extended into three dimensions). A single CT or MRI voxel
typically represents a tissue volume on the order of
1mm\textsuperscript{3}. Histopathological images, by contrast, are 2D
sections at micrometer resolution so the basic unit is the pixel.

\section{Computed Tomography}\label{computed-tomography}

CT produces cross-sectional images of the body
using X-rays \citep{sprawlsPhysicalPrinciplesMedical1995}. The physics begins with how X-ray photons
interact with matter. When photons pass through tissue, they can be
absorbed (the photoelectric effect) or deflected (Compton scattering).
Both processes remove photons from the X-ray beam reducing its
intensity. This reduction is known as attenuation and we quantify it
with the linear attenuation coefficient, \emph{μ}. In our bodies,
different tissues attenuate X-rays to different degrees depending on
their composition. For instance, bone, with its high calcium content,
attenuates strongly (i.e. has a high \emph{μ}); air attenuates almost
none; soft tissues like liver fall in between.

A CT scanner consists of an X-ray tube and a rotating detector, with
the patient lying on a table that passes through. The X-ray tube emits a
narrow beam that traverses the patient while detectors on the opposite
side count the photons that emerge. Each detector reading represents the
sum of attenuation along that beam\textquotesingle s
path---mathematically, a line integral of \emph{μ}. Fewer emerging
photons means more attenuation occurred along that path. Over a full
rotation, the scanner collects hundreds of such measurements from
different angles, effectively sampling line integrals of \emph{μ(x)}
through the body from all directions. The relationship follows
Lambert-Beer\textquotesingle s law: a beam from origin \emph{s,} with
unit direction vector \emph{θ} and initial intensity \(N_{0}\) is reduced
to \(N\) as it passes through tissue, where:

\[N\, = \, N_{0}\, e^{- \int_{}^{}{\, d\lambda\,\mu(s\, + \,\lambda\theta)}\,}\]

Turning these raw measurements into an image is an inverse problem:
recovering the spatial distribution of attenuation coefficients from
their line integrals. The standard solution is filtered back-projection,
which combines a sharpening filter with projection along original beam
paths to reconstruct each slice. The result is a 3D map of attenuation
coefficients, reconstructed as a stack of slices. Each slice represents
a cross-section of the body at a given thickness, typically 0.5--5 mm
depending on the clinical application. Each element in this 3D map
(voxel) represents the average attenuation within a small tissue volume.
All voxel values are rescaled to Hounsfield units (HU)\footnote{The
  method was developed in the early 1970s by Sir Godfrey Hounsfield,
  building on mathematical work by Allan Cormack on image
  reconstruction, for which they shared the 1979 Nobel Prize in
  Physiology or Medicine.}, a dimensionless scale defined so that water
is 0 HU and air is −1000 HU:

\[\text{HU} = 1000 \times \frac{\mu_{\text{tissue}} - \mu_{\text{water}}}{\mu_{\text{water}}}\]

On this scale (\Cref{fig:2.1}), soft tissues
typically fall around 20--80 HU, bone exceeds +400 HU, and fat is
slightly negative (around −100 HU). Radiologists can adjust the range of
HU values (known as window) to optimize visualization for different
tissues. Importantly, HU values depend on scanner model, acquisition
settings (tube voltage, reconstruction kernel), and whether contrast has
been administered. This variability is why only broad HU ranges exist
for tissues rather than a precise atlas. When contrast is used we refer
to the image as a CECT scan. Contrast agents
are usually iodine-based. This is because iodine has a high atomic
number, which makes it a strong X-ray absorber. Thus, tissues containing
the contrast show increased attenuation. Since contrast is administered
intravenously, the resulting enhancement reflects blood flow. A CECT
scan can be obtained in different phases related to where the contrast
is travelling through the body at that point:

\begin{enumerate}
\def\labelenumi{\arabic{enumi}.}
\item
  Arterial phase (\textasciitilde20--30 seconds post-injection):
  contrast fills the arterial system, highlighting arteries and
  hypervascular lesions that receive blood directly from arterial
  supply.
\item
  Portal venous phase (\textasciitilde60--70 seconds): contrast fills
  the portal venous system and liver tissue. This phase provides the
  best tumor-to-liver contrast for detection of liver metastases.
\item
  Delayed phase (several minutes): contrast has diffused into
  extracellular spaces and is washing out.
\end{enumerate}

CT is one of the most widely used imaging modalities due to its speed
(the total acquisition time is less than 15 minutes), availability, and
quantitative nature. The main disadvantage is ionizing radiation. For
primary liver tumors such as hepatocellular carcinoma, MRI is preferred
because its superior soft tissue contrast enables more accurate
diagnosis \citep{CancerFactsFigures}. For colorectal
liver metastases, however, CT remains the standard for diagnosis and
response assessment.

\begin{figure}[htbp]
\centering
\includegraphics[width=0.95\textwidth]{fig_2_1.png}
\caption[Hounsfield units and contrast enhancement in abdominal CT]{\textbf{Hounsfield units and contrast enhancement in abdominal CT.}
\textbf{(A)} Hounsfield unit (HU) scale with typical values for air
(−1000 HU), fat (≈ −100 HU), water (0 HU), soft tissue such as liver (≈
40--80 HU), and bone (>1000 HU). HU provide a quantitative
measure of X-ray attenuation, with values depending on scanner and
protocol. \textbf{(B)} Axial abdominal CT images acquired without
contrast and after contrast administration in the arterial and portal
venous phases, adapted from \citep{hartungAbdominalCTPhases2024}. Images are displayed
according to radiological convention (viewed from the patient's feet).
Arrows indicate the liver's vasculature (blue) and the aorta (green) to
highlight phase-specific enhancement. Created with BioRender.com.}
\label{fig:2.1}
\end{figure}

\section{Magnetic Resonance Imaging}\label{magnetic-resonance-imaging}

Magnetic resonance imaging\footnote{MRI was developed for clinical use
  around 1980, building on foundational work by Lauterbur (who described
  the first MR image in 1973) and Mansfield (who developed
  gradient-based spatial encoding). Both were awarded the Nobel Prize in
  Physiology or Medicine in 2003.} (MRI) generates images by exploiting
the magnetic properties of atomic nuclei, primarily hydrogen. Unlike
CT, MRI uses no ionizing radiation. Instead, it's based on
radiofrequency pulses and strong magnetic fields to induce and detect
signals from water protons in tissue. \Cref{fig:2.2} shows examples of MRI images of patients with liver metastases.

\subsection{Physical Principles}\label{physical-principles}

A hydrogen nucleus consists of a single proton, which possesses a
property called spin, an intrinsic angular momentum that gives it a
small magnetic moment. Normally, these magnetic moments are randomly
oriented. When placed in a strong external magnetic field (typically 1.5
or 3 Tesla in clinical scanners), they align preferentially with the
field direction, creating a net magnetization \citep{lauterburImageFormationInduced1973}. A
radiofrequency pulse at the resonant frequency (known as the Larmor
frequency) tips this magnetization away from equilibrium. When the pulse
ends, the system relaxes back to equilibrium, releasing energy as a
radiofrequency signal that can be detected by receiver coils. This
signal is used to construct the MR image.

Two relaxation processes govern signal behavior \citep{pooleyFundamentalPhysicsMR2005}. T1
relaxation (spin-lattice relaxation) describes the recovery of
longitudinal magnetization as protons release energy to their
surroundings. T2 relaxation (spin-spin relaxation) describes the decay
of transverse magnetization as protons lose phase coherence due to
interactions with each other. Different tissues have different T1 and T2
values depending on their molecular environment, which is what creates
contrast in MR images. By manipulating the timing of pulses and signal
acquisition, different contrasts can be generated. T1-weighted images
emphasize differences in T1 (fat appears bright, fluid dark) while
T2-weighted images emphasize differences in T2 (fluid appears bright).
This flexibility allows MRI to highlight different tissue properties
depending on clinical need. Spatial localization is achieved through
magnetic field gradients, which are small, controlled variations in
field strength across the imaging volume. These gradients encode spatial
position into the frequency and phase of the MR signal, enabling
reconstruction of a spatially resolved image through Fourier
transformation.

\subsection{Quantitative MRI}\label{quantitative-mri}

MRI can provide quantitative measurements of tissue properties through
specialized sequences and biophysical modeling. In this thesis, two
types of quantitative MRI are relevant: diffusion-weighted imaging and
dynamic contrast-enhanced imaging.

\subsubsection{Diffusion-Weighted
Imaging}\label{diffusion-weighted-imaging}

Diffusion-weighted MRI (DWI) measures the random motion of water
molecules in tissue \citep{lebihandImagerieDiffusionVivo1985}. A magnetic field gradient is
applied that labels water protons based on their position. After a short
interval, an opposite gradient is applied. If protons were stationary,
the second gradient would perfectly reverse the effect of the first and
full signal would be recovered. But water molecules undergo Brownian
motion, so protons move between the two gradients. Those that have
diffused further experience incomplete refocusing, leading to signal
loss. The more freely water can diffuse, the greater the signal
attenuation.

The strength and timing of the diffusion gradients are combined into a
single parameter called the b-value (with units s/mm²). At higher
b-values, the sequence becomes more sensitive to diffusion and signal
attenuation increases. The ADC
quantifies water mobility by fitting the signal decay across b-values.
For two b-values, b₁ and b₂, and corresponding signal intensities S(b₁)
and S(b₂), the ADC is:

\[\text{ADC} = \frac{1}{b_{2} - b_{1}}\ln\left( \frac{S\left( b_{1} \right)}{S\left( b_{2} \right)} \right)\]

Low ADC indicates restricted diffusion (water has diffused less so there
is less signal attenuation). Biologically, this can mean high
cellularity since cells obstruct water movement. Following the same
reasoning, high ADC indicates freer diffusion (water has diffused more
and therefore there is more signal attenuation), which means that the
tissue in question has low cellularity (e.g. necrosis or edema). While
ADC is used as a proxy for cellularity, we must acknowledge that it is a
measurement conflating multiple factors including cell density, cell
size, membrane permeability, and vascular perfusion. These factors
affect ADC in competing ways, which explains why correlations between
ADC and histological cellularity have been inconsistent across studies
\citep{chenCorrelationApparentDiffusion2013,surovCorrelationApparentDiffusion2017,yoshikawaRelationCancerCellularity2008}.

More advanced diffusion models attempt to disentangle these factors by
assuming a certain tissue microstructure. In these models, the signal is
represented as a mixture of compartments with distinct water
diffusivity, such as vascular, intracellular, and extracellular
components \citep{novikovModeling2018}. Many such models have been proposed,
and while they remain mostly investigational, they have shown promising
results for clinical applications. A well known example of a model is
the intravoxel incoherent motion (IVIM) model \citep{lebihanSeparationDiffusionPerfusion1988},
which assumes that tissue water is present in two different
compartments: vascular (pseudo-diffusing water inside blood vessels) and
nonvascular (diffusing water in and around cells). In this thesis, when
diffusion--relaxation modeling is employed, we therefore distinguish
between a tissue-related apparent diffusion coefficient (ADCₜ) and a
vascular-related apparent diffusion coefficient (ADCᵥ). The specific
models and fitting strategies used are described in Chapter~\ref{ch:5}.

\begin{figure}[htbp]
\centering
\includegraphics[width=0.85\textwidth]{fig_2_2.png}
\caption[Examples of MRI and multiparametric MRI maps]{\textbf{Examples of MRI and multiparametric MRI maps.}
Axial T2-weighted MRI slices from three representative patients with
liver metastases. The example mpMRI maps overlays are the apparent
diffusion coefficient (ADC) and the volume transfer constant
(K\textsuperscript{trans}), derived from DWI and DCE, respectively. In
ADC, darker regions indicate restricted diffusion (higher cellularity).
In the K\textsuperscript{trans} maps, brighter regions indicate higher
vascular permeability.}
\label{fig:2.2}
\end{figure}

\subsubsection{Dynamic Contrast-Enhanced
MRI}\label{dynamic-contrast-enhanced-mri}

DCE captures the flow of a contrast
agent through tissue over time. Unlike CT, the contrast agent is
gadolinium-based. Following the injection of the contrast, MRI
differentiates tissues based on how gadolinium shortens the T1
relaxation. Therefore, in tissues where gadolinium accumulates, the
signal is enhanced. Pharmacokinetic models analyze these enhancement
curves. The most common model used is the Tofts model \citep{toftsMeasurementBloodbrainBarrier1991} which assumes that the movement of the contrast agent in the
tissue is from the blood vessels to the extracellular space (outside the
cells). Thus, we can obtain a parameter related to the contrast agent
leakage (perfusion), \emph{K\textsuperscript{tran}}\textsuperscript{s},
and a parameter related to the volume of the extracellular space,
\emph{ve}.

Within a single examination, multiple MRI sequences can be acquired to
obtain functional information beyond anatomy. We refer to this
combination of sequences as mpMRI. After
preprocessing and model fitting, we can obtain the different
quantitative parameters such as ADC, $f_v$, and $K^{trans}$. We refer to these
as mpMRI maps. In this thesis, mpMRI serves as the biological reference
for developing and validating CT-derived habitats (Chapter~\ref{ch:7}). The
specific parameters used are detailed in Chapter~\ref{ch:5}.

\section{Histopathologic Light Microscopy}\label{histopathologic-light-microscopy}

Unlike the non-invasive imaging techniques described above,
histopathological assessment requires the surgical removal of tissue.
The extracted sample must then be processed through a series of
preparation steps \citep{paxtonLeedsHistologyGuide2003} prior to microscopic
examination.

First, to prevent tissue decay, the sample requires chemical fixation,
which is done by applying formalin, which inactivates enzymes and kills
bacteria that would otherwise degrade the tissue. Next, to enable
processing into thin slices, the tissue\textquotesingle s mechanical
stability must be increased. This is achieved by first dehydrating the
sample with alcohol and then embedding it in paraffin wax. After this
the tissue is a hardened block which is cut by a microtome (i.e. a
slicer) that produces sections typically around 5 μm thick. Once the
tissue sections are mounted on glass microscope slides, dyes are applied
to enhance visual contrast of cellular components. The most common
staining is Haematoxylin and eosin (H\&E), in which cell nuclei appear
purple and most other components different shades of pink
(\Cref{fig:2.3}).

The stained and mounted sections are then examined through conventional
light microscopy, which achieves resolutions down to approximately 200
nm. In the last few years pathologists have shifted from using
conventional microscopy to digitized images. This is known as
WSI \citep{pantanowitzWholeSlideImaging2015}. Here, automated
scanners that combine a light microscope and a digital camera digitize
whole tissue sections. WSI images are usually obtained at several
magnifications (stored as an image pyramid), with 40× magnification
producing pixels that are roughly 0.25 μm in size. The resulting files
grow to several gigabytes per slide, which presents storage issues but
also makes computational analysis possible. Algorithms that quantify
details that are impossible for the human eye to count can be fed the
same image that a pathologist examines.

Understanding the scale gap between radiology and pathology will allow
us to contextualize the results in Chapter~\ref{ch:7}. A single CT voxel may
contain tens of thousands of hepatocytes (liver cells) while in a
whole-slide image a single hepatocyte covers thousands of pixels. The
resolution gap is approximately nine orders of magnitude in
volume.\footnote{A CT voxel is approximately 1 mm³ volume (=10⁹ μm³).
  Assuming an average hepatocyte diameter of \textasciitilde30 μm
  (volume ≈ 14,000 μm³ for a sphere), this is roughly 10⁴--10⁵
  hepatocytes per voxel. Whole-slide imaging at 40× magnification has a
  pixel size of \textasciitilde0.25 × 0.25 μm (0.0625 μm²).
  Approximating a hepatocyte cross-section as a circle of diameter 30 μm
  (area ≈ 700 μm²), one cell spans \textasciitilde10⁴ pixels. Comparing
  volumes: one CT voxel (10⁹ μm³) versus one histology pixel extended
  to a 5 μm section thickness (0.0625 μm² × 5 μm ≈ 0.3 μm³) gives a
  ratio of \textasciitilde10⁹.} As an analogy, we can picture
Barcelona's beach. Histopathology is like examining individual grains of
sand while CT imaging is like viewing the beach from an airplane. From
altitude we can tell fine sand from rocky areas, but individual grains
are invisible.

\begin{figure}[htbp]
\centering
\includegraphics[width=0.95\textwidth]{fig_2_3.png}
\caption[From radiological imaging to histopathological examination]{\textbf{From radiological imaging to histopathological examination.}
Pre-operative CT reveals a colorectal liver metastasis (dashed
contour, $\sim$4 cm). Following surgical resection, the
specimen is sectioned and processed for whole-slide imaging (H\&E),
enabling examination at cellular resolution.}
\label{fig:2.3}
\end{figure}


