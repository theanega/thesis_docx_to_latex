\chapter{\texorpdfstring{Identification of Precise Handcrafted Features
for CT{} Habitats
Computation}{Identification of Precise Handcrafted Features for CT Habitats Computation}}\label{identification-of-precise-handcrafted-features-for-ct-habitats-computation}

Robustness is a prerequisite for useful representations. If a radiomic
feature\textquotesingle s value changes meaningfully when a patient
shifts position by a few millimeters, or when we adjust a minor
computational parameter, then any habitat computed with that feature
will be measuring noise, not biology. In this Chapter we describe a
comprehensive study of precision analysis of handcrafted features, with
the goal of identifying optimal features for stable habitat computation.

\textbf{Contributions}:

\begin{itemize}
\item
  We assess repeatability by perturbating images to simulate a
  test-retest dataset collected under routine clinical conditions.
\item
  We assess reproducibility against two key parameters for computing
  handcrafted radiomics: kernel radius (R, defining the neighborhood for
  feature calculation) and bin size (B, defining intensity
  discretization).
\item
  Combining repeatability and reproducibility resutls, we identify a
  subset of precise voxelwise features for both lung and liver lesions,
  and test whether habitats computed with only precise features are more
  stable.
\item
  We explore the biological meaning of habitats in an independent cohort
  of 13 liver metastases with mpMRI and histology.
\end{itemize}

This chapter reproduces many parts of the publication (Prior et al.,
2024), adapted to the thesis format. A note on scope: lung lesion
results are not directly relevant to the questions addressed in this
thesis and therefore are not discussed in depth, they have just been
kept for comparison.

\section{Rationale}\label{rationale}

CT{}-based habitat imaging offers a non-invasive, volumetric approach to
quantify tumor heterogeneity. Unlike biopsy, it samples the entire
tumor. Unlike whole-region summary statistics (e.g. mean kurtosis), it
preserves spatial structure. This makes habitats potentially valuable
for treatment planning, response monitoring, and understanding
resistance patterns---but only if the habitats themselves are stable.

For habitats to be stable, and therefore clinically useful, the
underlying radiomics features (RFs) (Lambin et al., 2017) must be
precise. This means that RFs should be both \emph{repeatable} (i.e.,
exhibiting measurement precision under the same set of computation
conditions, also known as test-retest) and \emph{reproducible} (i.e.,
exhibiting measurement precision under different computation conditions)
(Kessler et al., 2015; Sullivan et al., 2015).

Most existing studies of radiomics precision, however, focus on features
extracted from entire regions of interest (ROIs) and evaluate them as
independent predictive biomarkers. Many such studies also fail to
provide critical information such as whether the features were computed
in 2D (pixelwise) or 3D (voxelwise) (Pfaehler et al., 2021; Traverso et
al., 2018). This is especially relevant for habitat compuation since
pixelwise features are computed slice-by-slice and ignore out-of-plane
texture information, making them less representative of true 3D tumor
architecture. Few studies address the precision of voxelwise 3D features
specifically, and fewer still in the context of clustering-based methods
like habitat imaging (Ng et al., 2013; Xu et al., 2019). Thus, there is
a lack of knowledge on precision of 3D RF for CT{} tumor habitat
computation.

We therefore set out to answer: Which 3D radiomics features are precise
enough to support stable habitat computation? We evaluated precision
under three sources of variability:

\begin{enumerate}
\def\labelenumi{\arabic{enumi}.}
\item
  Repeatability: Do features remain consistent when the same patient is
  scanned twice (test-retest)?
\item
  Reproducibility against kernel radius (R): Does changing the size of
  the neighborhood used for feature calculation alter feature values
  significantly?
\item
  Reproducibility against bin size (B): Does changing the intensity
  discretization design alter feature values significantly?
\end{enumerate}

Features that ``passed'' all three tests were identified as
\emph{precise} and suitable for habitat modeling. We then asked whether
using only precise features resulted in more stable habitats, using a
Gaussian mixture model (GMM) clustering approach (Divine et al., 2016;
Jardim-Perassi et al., 2019). Finally, we explored the biological
plausibility of the resulting CT{} habitats by comparing them to
multiparametric MRI and histopathology in an independent liver
metastasis cohort.

\section{Methods}\label{methods}

\subsection{Patient Cohorts}\label{patient-cohorts-1}

We analyzed 1,861 liver lesions and 575 lung lesions from 318 patients
(mean age 64.5 years ± 10.1 SD; 185 male) imaged at multiple timepoints
between November 2010 and December 2021 (\hyperref[_Ref219819126]{Figure
6.1}A). All patients had advanced cancer with liver metastases.
Intravenous contrast-enhanced CT{} scans were acquired as part of
routine clinical care at Vall d\textquotesingle Hebron University
Hospital. The analysis of anonymized imaging data was approved by the
Vall d\textquotesingle Hebron Ethics Committee with waiver of informed
consent. The cohort comprised four primary tumor types: (1) colorectal,
(2) lung, (3) gastrointestinal neuroendocrine tumors, and (4) a mixed
group of other cancers. Patients with neuroendocrine tumors were
enrolled in the multicenter phase II TALENT trial (NCT02678780). Full
cohort characteristics and imaging protocols are detailed in Annex B.

A small independent cohort of 13 patients with liver metastases,
enrolled in the PREDICT prospective trial (PR{[}AG{]}29/2020), underwent
CT{}, multiparametric MRI (mpMRI), and image-guided biopsy. This cohort
was used solely for exploratory biological validation. All participants
provided written informed consent for acquisition and analysis of
imaging and tissue samples.

\subsection{Image Segmentation and Perturbation}\label{image-segmentation-and-perturbation}

A radiologist with more than 10 years of experience in oncologic imaging
manually segmented all measurable lesions (maximal diameter ≥10 mm per
RECIST 1.1 (Eisenhauer et al., 2009)) in 3D using 3D Slicer
(v4.11.20210226) (Fedorov et al., 2012).

To assess repeatability, we simulated test-retest scenarios by applying
controlled image perturbations, implemented using the
\href{https://github.com/oncoray/mirp}{Medical Image Radiomics Processor
(MIRP) toolkit} (v1.2.0) (Zwanenburg et al., 2019). Details of the
perturbation protocol are provided in Annex B.

\begin{figure}[htbp]
\centering
\includegraphics[width=0.95\textwidth]{fig_6_1.png}
\caption{Precision analysis workflow.
\textbf{(A)} Distribution of lung and liver lesions across different
cohorts for precision analysis. \textbf{(B)} Precision analysis design.
3D radiomic features were computed from both original and perturbed
images, four times per image, each time with a different combination of
kernel radius, R (1mm/3mm), and bin size, B (12HU/25HU). To study
repeatability, original-perturbed feature pairs were evaluated for every
combination of extraction settings (R1B12, R1B25, R3B12 and R3B25). To
study reproducibility against extraction parameters, we compared
original feature pairs extracted with different extraction settings: for
reproducibility against R, we first compared features extracted with
fixed B=12HU and different kernel radius, and then repeated for features
extracted with fixed B=25HU; analogously, for reproducibility against
B, we compared features extracted from original images with fixed R=1mm
and different bin size, and then repeated with those extracted with
fixed R=3mm. Precise features were selected by linking reproducibility
and repeatability results. R, kernel size; B, bin size.}
\label{fig:6.1}
\end{figure}

\subsection{Radiomics Feature Computation}\label{radiomics-feature-computation}

We computed 91 voxelwise radiomics features using PyRadiomics (v3.0.1)
(Van Griethuysen et al., 2017). The full feature list is provided in
Annex B. Features were calculated on original (unfiltered) images at
each voxel within the segmented lesion volume. For each lesion we
computed features four times, using the different combinations of the
two computational parameteres we're studying:

\begin{itemize}
\item
  \textbf{Kernel radius (R)}: the size of the neighborhood used for
  texture computation (1 mm or 3 mm).
\item
  \textbf{Bin size (B)}: the intensity discretization used before
  feature calculation (12 HU or 25 HU).
\end{itemize}

The four parameter combinations---denoted R1B12, R1B25, R3B12, and
R3B25---were chosen based on common practice in the radiomics
literature. PyRadiomics defaults to R=1 mm and B=25 HU; we selected R=3
mm and B=12 HU as alternative values frequently reported in prior
studies. All computations followed the Image Biomarker Standardization
Initiative (IBSI) guidelines (Zwanenburg et al., 2020). Computation
details are provided in Annex B.

\subsection{Precision Analysis: Repeatability and Reproducibility}\label{precision-analysis-repeatability-and-reproducibility}

\hyperref[_Ref219819126]{Figure 6.1}B summarizes the precision analysis
workflow. We evaluated:

\begin{itemize}
\item
  \textbf{Repeatability} (test-retest stability): We compared feature
  values from original and perturbed images for each of the four
  parameter settings (R1B12, R1B25, R3B12, R3B25).
\item
  \textbf{Reproducibility against kernel radius (R)}: We compared
  features computed with R=1 mm versus R=3 mm, holding bin size
  constant. Two experiments were conducted: one at B=12 HU (R1B12 vs.
  R3B12) and one at B=25 HU (R1B25 vs. R3B25).
\item
  \textbf{Reproducibility against bin size (B)}: We compared features
  computed with B=12 HU versus B=25 HU, holding kernel radius constant.
  Two experiments were conducted: one at R=1 mm (R1B12 vs. R1B25) and
  one at R=3 mm (R3B12 vs. R3B25).
\end{itemize}

All experiments were performed on the full dataset, on liver and lung
lesions separately, and stratified by primary tumor type. This allowed
us to assess whether precision varied by lesion location or primary
cancer.

We quantified precision using the intraclass correlation coefficient
(ICC) (Koo \& Li, 2016). For repeatability, we used a
single-measurement, absolute-agreement, two-way mixed-effects model. For
reproducibility, we used a single-measurement, consistency, two-way
mixed-effects model. Following Koo and Li, we classified features based
on the lower 95\% confidence limit (LCL) of the ICC: poor (LCL
\textless0.50), moderate (0.50 ≤ LCL \textless0.75), good (0.75 ≤ LCL
\textless0.90), and excellent (LCL ≥0.90). A feature was selected as
\textbf{precise} if the 95\% lower confidence limit (LCL) of the ICC was
≥ 0.50 across the three relevant experiments: repeatability (R3B12),
reproducibility against R (B=12HU), and reproducibility against B
(R=3mm).

\subsection{Habitat Computation and Stability Assessment}\label{habitat-computation-and-stability-assessment}

Habitats were computed using Gaussian mixture models (GMMs). We
clustered voxels based on their radiomics feature values, treating each
lesion independently. To avoid redundancy, we first removed highly
correlated features (Spearman\textquotesingle s r ≥0.70, p
\textless0.001) (Schwarz, 1978). The optimal number of clusters
(habitats) for each lesion was determined using the Bayesian Information
Criterion (BIC) (Cohen, 1992). We computed habitats four times per
lesion:

\begin{itemize}
\item
  Using all 91 features, on original images
\item
  Using all 91 features, on perturbed images
\item
  Using only precise features, on original images
\item
  Using only precise features, on perturbed images
\end{itemize}

Habitat stability was quantified by comparing the spatial overlap of
habitats derived from original versus perturbed images, using the Dice
Similarity Coefficient (DSC). Higher DSC indicates greater stability. We
tested the hypothesis that habitats derived from precise features would
be more stable than those derived from all features. Details of the GMM
implementation and BIC optimization are provided in Annex B.

\subsection{Exploratory Biological Case Study}\label{exploratory-biological-case-study}

To assess whether CT{} habitats capture biologically meaningful tissue
compartments rather than imaging artifacts, we conducted an exploratory
analysis in a subset of 13 patients from the PREDICT cohort (see Section
5.1) who had co-acquired CT{}, multiparametric MRI (mpMRI) and digitized
hematoxylin-eosin (HE) histopathology from biopsy.

CT{} habitats were computed using the precise liver features identified
in Section 6.2.4, clustered with GMM and BIC as described above. For
comparison, mpMRI habitats were computed by voxelwise clustering of nine
quantitative maps---warped non-linearly onto a high-resolution
T2-weighted anatomical scan---including tissue T2 (T2t), longitudinal
relaxation time (T1), tissue and vascular apparent diffusion
coefficients (ADCt, ADCv), tissue apparent kurtosis (Kt), vascular
fraction (fv), capillary permeability constant (Ktrans), extracellular
extravascular volume fraction (ve), and plasma volume fraction (vp). We
then compared the number of habitats selected by BIC for both CT{} and
mpMRI habitats in each lesion and the qualtiative correspondence between
imaging habitats and tissue phenotypes identified by a pathologist on HE
slides.

This analysis was deliberately exploratory. Given the sampling bias, our
goal was not to establish definitive biological ground truth, but to
confirm that CT{} habitats have a plausible relationship to known tissue
phenotypes such as necrosis and viable tumor.

\subsection{Statistical Analysis}\label{statistical-analysis}

Differences in feature reproducibility across parameter settings, lesion
locations, and habitat stability were assessed using paired two-sided
Wilcoxon signed-rank tests. Effect sizes were calculated using
Cohen\textquotesingle s d and classified as small (d ≥0.20), medium (d
≥0.50), or large (d ≥0.80) (Cohen, 1992). Statistical significance was
defined as p \textless0.05. All statistical analyses were reviewed by a
statistician and performed in Python (v3.7.10). Code for reproducibility
and repeatability analysis is publicly available at
\url{https://github.com/radiomicsgroup/precise-habitats}.

\section{Results}\label{results}

\subsection{Precision Analysis}\label{precision-analysis}

\textbf{Repeatability.} \hyperref[_Ref219819183]{Figure 6.2} shows the
distribution of repeatability across the four parameter settings. Most
features exhibited poor repeatability (ICC LCL \textless0.50) regardless
of settings, but clear differences emerged between parameter choices.

Features computed with a kernel radius of 3 mm were substantially more
repeatable than those computed with 1 mm, regardless of bin size. The
R3B12 setting yielded the highest repeatability, with a median ICC LCL
of 0.442 (IQR: 0.312--0.516) across all features, compared to 0.191
(0.116--0.382) for R1B12, 0.199 (0.103--0.344) for R1B25, and 0.415
(0.306--0.516) for R3B25. Bin size had minimal impact: changing B from
12 HU to 25 HU did not meaningfully alter repeatability for either
kernel radius.

The type of repeatable features varied by lesion location
(\hyperref[_Ref219819183]{Figure 6.2}B). In liver lesions, first-order
and Gray-Level Run-Length Matrix (GLRLM) features were more repeatable.
Primary tumor type (colorectal, lung, neuroendocrine, other) had no
detectable effect on repeatability (Annex B).

\begin{figure}[htbp]
\centering
\includegraphics[width=0.95\textwidth]{fig_6_2.png}
\caption{Repeatability of voxelwise radiomics features.
\textbf{(A)} Repeatability distribution of radiomics features per
setting. Most radiomic features exhibit poor repeatability. Features
extracted with kernel radius (R) of 3mm were more repeatable than those
extracted with R=1mm. Bin size changes didn't affect repeatability.
\textbf{(B)} Repeatability distribution of radiomics features extracted
with setting R3B12 per feature class for lung and liver lesions
separately. First order and GLRLM features were more repeatable in liver
lesions while GLCM features were more repeatable in lung lesions. LCL,
95\% lower confidence limit of the Intraclass Correlation Coefficient;
R3B12, features extracted with kernel radius 3mm and bin size 12HU; FO,
First-Order; GLCM, Grey Level Co-occurrence Matrix features; GLDM, Grey
Level Dependence Matrix; GLRLM, Grey Level Run Length Matrix; GLSZM,
Grey Level Size Zone Matrix; NGTDM, Neighboring Grey Tone Difference
Matrix Features.}
\label{fig:6.2}
\end{figure}

\textbf{Reproducibility.} Features were far more sensitive to changes in
kernel radius than to changes in bin size
(\hyperref[_Ref219819213]{Figure 6.3}A-B). Changing R from 1 mm to 3 mm
substantially altered feature values: median ICC LCL was only 0.440
(IQR: 0.330--0.526) when B=12 HU and 0.437 (0.355--0.524) when B=25 HU.
In contrast, changing B from 12 HU to 25 HU had little effect: median
ICC LCL was 0.929 (0.853--0.988) when R=3 mm and 0.833 (0.706--0.946)
when R=1 mm.

\textbar{} Reproducibility against kernel radius and bin size

Median (IQR) lower confidence limit of the Intraclass Correlation
Coefficient reported. B = bin size, HU = Hounsfield units, R = kernel
radius.

{\def\LTcaptype{none} % do not increment counter
\begin{longtable}[]{@{}
  >{\centering\arraybackslash}p{(\linewidth - 16\tabcolsep) * \real{0.1666}}
  >{\centering\arraybackslash}p{(\linewidth - 16\tabcolsep) * \real{0.1047}}
  >{\centering\arraybackslash}p{(\linewidth - 16\tabcolsep) * \real{0.1053}}
  >{\centering\arraybackslash}p{(\linewidth - 16\tabcolsep) * \real{0.1036}}
  >{\centering\arraybackslash}p{(\linewidth - 16\tabcolsep) * \real{0.1049}}
  >{\centering\arraybackslash}p{(\linewidth - 16\tabcolsep) * \real{0.1044}}
  >{\centering\arraybackslash}p{(\linewidth - 16\tabcolsep) * \real{0.1051}}
  >{\centering\arraybackslash}p{(\linewidth - 16\tabcolsep) * \real{0.1040}}
  >{\centering\arraybackslash}p{(\linewidth - 16\tabcolsep) * \real{0.1014}}@{}}
\toprule\noalign{}
\begin{minipage}[b]{\linewidth}\centering
\end{minipage} &
\multicolumn{4}{>{\centering\arraybackslash}p{(\linewidth - 16\tabcolsep) * \real{0.4186} + 6\tabcolsep}}{%
\begin{minipage}[b]{\linewidth}\centering
\textbf{Reproducibility against R}
\end{minipage}} &
\multicolumn{4}{>{\centering\arraybackslash}p{(\linewidth - 16\tabcolsep) * \real{0.4149} + 6\tabcolsep}@{}}{%
\begin{minipage}[b]{\linewidth}\centering
\textbf{Reproducibility against B}
\end{minipage}} \\
\midrule\noalign{}
\endhead
\bottomrule\noalign{}
\endlastfoot
&
\multicolumn{2}{>{\centering\arraybackslash}p{(\linewidth - 16\tabcolsep) * \real{0.2100} + 2\tabcolsep}}{%
\textbf{Fixed B = 12HU}} &
\multicolumn{2}{>{\centering\arraybackslash}p{(\linewidth - 16\tabcolsep) * \real{0.2085} + 2\tabcolsep}}{%
\textbf{Fixed B = 25HU}} &
\multicolumn{2}{>{\centering\arraybackslash}p{(\linewidth - 16\tabcolsep) * \real{0.2095} + 2\tabcolsep}}{%
\textbf{Fixed R = 1mm}} &
\multicolumn{2}{>{\centering\arraybackslash}p{(\linewidth - 16\tabcolsep) * \real{0.2053} + 2\tabcolsep}@{}}{%
\textbf{Fixed = 3mm}} \\
\emph{Lesion type} & \emph{Liver} & \emph{Lung} & \emph{Liver} &
\emph{Lung} & \emph{Liver} & \emph{Lung} & \emph{Liver} & \emph{Lung} \\
LCL {[}Median(IQR){]} & 0.422 (0.346-0.513) & 0.573 (0.403-0.701) &
0.407 (0.291-0.536) & 0.573 (0.443-0.696) & 0.805 (0.672-0.919) & 0.929
(0.823-0.997) & 0.921 (0.821-0.982) & 0.967 (0.93-0.999) \\
\end{longtable}
}

In other words, \textbf{most features were reproducible against bin
size} (good or excellent agreement), \textbf{but poorly reproducible
against kernel radius}. The choice of neighborhood size matters more
than the choice of discretization bin. Features computed with B=12 HU
were slightly more reproducible against kernel radius than those with
B=25 HU (p \textless0.001). Features computed with R=3 mm were more
reproducible against bin size than those with R=1 mm (p \textless0.001).

As with repeatability, lesion location affected reproducibility
(\hyperref[_Ref219819213]{Figure 6.3}C-D, \hyperref[_Ref219819227]{Table
6.2}\hyperref[_Ref219819242]{Table 6.1}). Features from lung lesions
were more reproducible than those from liver lesions, both against R and
against B (p \textless0.001), particularly for GLCM and GLRLM feature
classes. Primary tumor type again had no detectable effect (Annex B).

\begin{figure}[htbp]
\centering
\includegraphics[width=0.95\textwidth]{fig_6_3.png}
\caption{Reproducibility against kernel radius and bin size.
\textbf{(A)} Reproducibility distribution against kernel radius, R, for
features extracted with fixed bin size of 12HU and bin size 25HU. Most
features present poor reproducibility against R. Features extracted with
B=12HU were more reproducible (p$<$.001). \textbf{(B)}
Reproducibility distribution against bin size, B, for features extracted
with fixed kernel radius 3mm and fixed kernel radius 1mm. Most features
present good or excellent reproducibility against B. Features extracted
with R=3mm were more reproducible (p$<$.001). \textbf{(C)}
Reproducibility distribution against kernel radius for features
extracted with fixed bin size of 12HU per feature class for lung and
liver lesions separately. Features extracted from lung lesions are more
reproducible against R, especially for features belonging to GLCM and
GLRLM classes. \textbf{(D)} Reproducibility distribution against bin
size for features extracted with fixed kernel radius 3mm per feature
class for lung and liver lesions separately. Features extracted from
lung lesions are more reproducible against B, especially for features
belonging to GLCM, first-order and NGTDM classes. LCL, 95\% lower
confidence limit of the Intraclass Correlation Coefficient; FO,
First-Order features; GLCM, Grey Level Co-occurrence Matrix features;
GLDM, Grey Level Dependence Matrix; GLRLM, Grey Level Run Length Matrix;
GLSZM, Grey Level Size Zone Matrix; NGTDM, Neighboring Grey Tone
Difference Matrix Features.}
\label{fig:6.3}
\end{figure}

\subsection{\texorpdfstring{6.3.2. Identification of Precise Features
}{6.3.2. Identification of Precise Features }}\label{identification-of-precise-features}

We defined a feature as precise if it met three criteria: ICC LCL ≥0.50
for (1) repeatability (R3B12 setting), (2) reproducibility against
kernel radius (at B=12 HU), and (3) reproducibility against bin size (at
R=3 mm). These thresholds represent the minimum acceptable stability for
clinical use.

Of the 91 features tested, 25 met all three criteria for liver lesions.
A 26th feature---NGTDM Coarseness---was added based on its consistently
high performance across experiments (top 3 most repeatable and
reproducible against B; see Annex B for justification). The final set
thus comprised 26 precise features for liver lesions. The same analysis
for lung lesions also identified 26 precise features
(\hyperref[_Ref219819227]{Table 6.2}). \hyperref[_Ref219819288]{Figure
6.4} displays the ICC LCL values for all 91 features across the three
experiments, for liver and lung lesions separately.

\textbar{} Precise 3D Radiomics Features in Liver and Lung Lesions

A precise radiomic feature was defined as lower confidence limit ≥ 0.50
in the repeatability and reproducibility experiments. FO = first-order;
GLCM = Grey Level Co-occurrence Matrix features; GLDM = Grey Level
Dependence Matrix; GLRLM = Grey Level Run Length Matrix; GLSZM = Grey
Level Size Zone Matrix; NGTDM = Neighboring Grey Tone Difference Matrix
Features.

{\def\LTcaptype{none} % do not increment counter
\begin{longtable}[]{@{}
  >{\raggedright\arraybackslash}p{(\linewidth - 2\tabcolsep) * \real{0.5464}}
  >{\raggedright\arraybackslash}p{(\linewidth - 2\tabcolsep) * \real{0.4536}}@{}}
\toprule\noalign{}
\begin{minipage}[b]{\linewidth}\centering
\textbf{Liver lesions}
\end{minipage} & \begin{minipage}[b]{\linewidth}\centering
\textbf{Lung} \textbf{lesions}
\end{minipage} \\
\midrule\noalign{}
\endhead
\bottomrule\noalign{}
\endlastfoot
FO\_10Percentile

FO\_90Percentile

FO\_Energy

FO\_Mean

FO\_Median

FO\_Minimum

FO\_RootMeanSquared

GLCM\_Autocorrelation

GLCM\_JointAverage

GLCM\_SumAverage

GLDM\_DependenceEntropy

GLDM\_GrayLevelNonUniformity

GLDM\_HighGrayLevelEmphasis

GLDM\_LargeDependenceLowGrayLevelEmphasis

GLDM\_LowGrayLevelEmphasis

GLDM\_SmallDependenceHighGrayLevelEmphasis

GLRLM\_GrayLevelNonUniformity

GLRLM\_HighGrayLevelRunEmphasis

GLRLM\_LongRunHighGrayLevelEmphasis

GLRLM\_LongRunLowGrayLevelEmphasis

GLRLM\_LowGrayLevelRunEmphasis

GLRLM\_RunLengthNonUniformity

GLRLM\_RunPercentage

GLRLM\_RunVariance

GLRLM\_ShortRunHighGrayLevelEmphasis & FO\_90Percentile

GLCM\_Id

GLCM\_Idm

GLCM\_Imc1

GLCM\_InverseVariance

GLCM\_JointEntropy

GLDM\_DependenceEntropy

GLDM\_DependenceNonUniformityNormalized

GLDM\_GrayLevelNonUniformity

GLDM\_LargeDependenceEmphasis

GLDM\_LargeDependenceHighGrayLevelEmphasis

GLDM\_SmallDependenceEmphasis

GLDM\_SmallDependenceHighGrayLevelEmphasis

GLRLM\_GrayLevelNonUniformity

GLRLM\_LongRunEmphasis

GLRLM\_LongRunHighGrayLevelEmphasis

GLRLM\_RunLengthNonUniformity

GLRLM\_RunLengthNonUniformityNormalized

GLRLM\_RunPercentage

GLRLM\_RunVariance

GLRLM\_ShortRunEmphasis

GLSZM\_LargeAreaEmphasis

GLSZM\_LargeAreaHighGrayLevelEmphasis

GLSZM\_ZonePercentage

GLSZM\_ZoneVariance \\
NGTDM\_Coarseness & NGTDM\_Coarseness \\
\multicolumn{2}{@{}>{\raggedright\arraybackslash}p{(\linewidth - 2\tabcolsep) * \real{1.0000} + 2\tabcolsep}@{}}{%
} \\
\end{longtable}
}

\begin{figure}[htbp]
\centering
\includegraphics[width=0.9\textwidth]{fig_6_4.png}
\caption{Summary of Precision Analysis.
Heatmap displaying the lower 95\% confidence limit (LCL) of the
Intraclass Correlation Coefficient results obtained in the three
experiments used to identify precise features: repeatability (setting
R3B12), reproducibility against R (fixed B=12HU), and reproducibility
against B (fixed R=3mm), for lung and liver lesions separately. FO,
First-Order features; GLCM, Grey Level Co-occurrence Matrix features;
GLDM, Grey Level Dependence Matrix; GLRLM, Grey Level Run Length Matrix;
GLSZM, Grey Level Size Zone Matrix; NGTDM, Neighboring Grey Tone
Difference Matrix Features.}
\label{fig:6.4}
\end{figure}

\subsection{Imaging Habitats Computed with Precise Features Show Increased Stability}\label{imaging-habitats-computed-with-precise-features-show-increased-stability}

Having identified which features are precise, we asked whether using
only these features improves habitat stability. For each lesion, we
computed habitats twice: once using all 91 features, and once using only
the 26 precise features. In both cases, highly correlated features
(Spearman\textquotesingle s r ≥0.70) were removed before clustering.
Habitats were generated using Gaussian mixture models (GMMs), with the
number of clusters determined by the Bayesian Information Criterion
(BIC). We then compared the spatial overlap of habitats derived from
original versus perturbed images using the Dice Similarity Coefficient
(DSC). Higher DSC indicates greater stability.

Habitats computed from precise features were significantly more stable
than those computed from all features (p \textless0.001, Wilcoxon
signed-rank test; \hyperref[_Ref219819367]{Figure 6.5}C). This held for
both liver and lung lesions, with small-to-medium effect sizes
(Cohen\textquotesingle s d = 0.34 for lung, 0.43 for liver).

For liver lesions, the median DSC was 0.587 (IQR: 0.465--0.703) when
using all features, compared to 0.651 (0.520--0.784) when using only
precise features. For lung lesions, median DSC was 0.532 (0.424--0.637)
with all features versus 0.601 (0.494--0.712) with precise features.

\hyperref[_Ref219819367]{Figure 6.5}B illustrates this difference for a
single liver lesion. Habitats derived from precise features maintained
their spatial structure across test-retest (DSC: 0.976, 0.891, 0.915 for
the three habitats), whereas habitats derived from all features
exhibited substantial variation (DSC: 0.751, 0.328, 0.570).

\textbf{\hfill\break
}

\begin{figure}[htbp]
\centering
\includegraphics[width=0.9\textwidth]{fig_6_5.png}
\caption{Precise features increase habitat stability.
\textbf{(A)} Original and perturbed CT scans for one liver lesion.
\textbf{(B)} Example of habitats obtained for the same lesion. Habitats
computed with precise features show higher stability (measured via Dice
Similarity Coefficient [DSC] of original-perturbed habitat pairs).
Top row: habitats obtained with precise features computed from original
image (left) and perturbed image (right). DSC scores for habitats 1, 2
and 3 are 0.976, 0.891 and 0.915, respectively. Bottom row: habitats
obtained with non-precise (i.e., all computed features) features
computed from original image (left) and perturbed image (right). DSC
scores for habitats 1, 2 and 3 are 0.751, 0.328 and 0.57, respectively.
\textbf{(C)} Quantification of habitat stability computed with precise
features and non-precise features for all lung and liver lesions.
Habitats computed with precise features present higher stability
(Wilcoxon signed rank test, p $<$ 0.0001). DSC: Dice Similarity
Coefficient.}
\label{fig:6.5}
\end{figure}

\subsection{Exploratory Biological Case Study}\label{exploratory-biological-case-study-1}

As a preliminary check of biological plausibility, we examined CT{}
habitats in 13 PREDICT patients with multiparametric MRI (mpMRI) and
histopathology (biopsy, not whole tumor resection).
\hyperref[_Ref219819404]{Figure 6.6} shows an example from a patient
with liver metastasis from melanoma.

CT{} habitats (computed from precise features) displayed spatial
distributions qualitatively consistent with habitats derived from nine
mpMRI maps (\hyperref[_Ref219819404]{Figure 6.6}A--B). For example, a
CT{} habitat depicted in blue---spatially concordant with an mpMRI
habitat in the lesion core---showed elevated T2t, T1, and ADCt
(consistent with lower cellularity) and reduced fv, Ktrans, ve, and vp
(consistent with lower vascularization). This pattern is compatible with
necrosis. Histopathology confirmed a necrotic core surrounded by viable
tumor (\hyperref[_Ref219819404]{Figure 6.6}C), matching the spatial
organization suggested by both imaging modalities.

This exploratory analysis relied on visual comparison rather than
voxel-wise co-registration or quantitative spatial metrics. Its purpose
was not to validate the biological meaning of habitats definitively, but
to rule out the possibility that they reflect pure imaging noise or
reconstruction artifacts. The qualitative agreement between CT{}
habitats, mpMRI habitats, and pathologist-identified tissue types
(necrosis, viable tumor) suggests that CT{} habitats encode biologically
relevant information about tumor microenvironment heterogeneity.

\begin{figure}[htbp]
\centering
\includegraphics[width=0.95\textwidth]{fig_6_6.png}
\caption{Exploration of the biological relevance of imaging habitats.
One exemplificatory patient (liver metastasis of melanoma). \textbf{(A)}
CT scan with visible lesion (yellow arrow) and resulting CT habitats
computed with precise liver radiomic features (also shown). \textbf{(B)}
Anatomical MRI T2 scan with visible lesion (yellow arrow) and resulting
mpMRI habitats computed with the 9 mpMRI maps (also shown). mpMRI and
CT habitats presented comparable distributions. \textbf{(C)} Image
guided biopsy with needle (N) and liver (L) tumor lesion (T), and
resulting histologic image (H\&E stain) with observable tissue types,
annotated by a pathologist. The H\&E-stained histological material reveals
areas of necrosis in the core of the lesion. ADCt: tissue apparent
diffusion coefficient, ADCv: vascular apparent diffusion coefficient,
T2t: tissue transverse relaxation time, AKt: tissue apparent kurtosis
coefficient, Fv: vascular signal fraction, T1: total longitudinal
relaxation time, Ktrans: capillary permeability constant, Vp: plasma
volume fraction, Ve: extravascular and extracellular volume fraction.}
\label{fig:6.6}
\end{figure}

\section{Discussion}\label{discussion}

To achieve an effective clinical translation of CT{} habitats, excellent
robustness of the underlying imaging features is necessary. In this
chapter we examined 3D radiomics\textquotesingle{} repeatability in a
simulated test-retest scenario and reproducibility against kernel radius
(R) and bin size (B), two relevant computation parameters.

Of the 91 features we evaluated, only 26 met our criteria for acceptable
repeatability and reproducibility in liver lesions. These precise
features were not a random subset---they exhibited consistent patterns.
Texture features generally outperformed first-order (histogram)
features, likely because histogram features depend on absolute intensity
values and are more vulnerable to outliers. Features computed with a
larger kernel radius (R=3 mm) were more repeatable than those with a
smaller radius (R=1 mm), suggesting that averaging over a larger
neighborhood reduces sensitivity to noise. Features were far more
sensitive to changes in kernel radius than to changes in bin size,
meaning the choice of neighborhood matters more than the choice of
discretization.

Using only precise features produced more stable habitats. When we
clustered voxels based on all 91 features, habitats shifted
substantially between test and retest (median DSC: 0.587 for liver
lesions). When we used only the 26 precise features, habitat boundaries
remained far more consistent (median DSC: 0.651). The improvement was
statistically significant (p \textless0.001) and, while modest in
absolute terms, represents a meaningful gain in reliability. Without
ground truth, we cannot say which habitat map is "correct," but we can
say which is more robust---and robustness is a prerequisite for clinical
utility.

The set of precise features differed between liver and lung lesions.
This was true both in number and in type: GLCM features were more
precise in lung, while first-order and GLRLM features were more precise
in liver. One plausible explanation is the difference in
contrast-to-noise ratio between the two tissue types. Lung lesions,
surrounded by air, exhibit high contrast on CT{}; liver lesions,
embedded in soft tissue, do not. This suggests that radiomics models are
not universally portable---features that work in one anatomical site may
not work in another. Primary tumor type (colorectal, lung,
neuroendocrine, other) had no detectable effect on precision, implying
that lesion location matters more than tumor origin.

Our exploratory case study attempted to explore the biological relevance
of CT{} habitats, inspired by previous studies that highlighted the
value of quantitative MRI-derived habitats for characterizing tumor
heterogeneity (Divine et al., 2016; Jardim-Perassi et al., 2019). In the
13 patients examined, CT{} habitats showed qualitative agreement with
habitats derived from nine mpMRI maps and with tissue features visible
on histology (e.g., necrotic cores, viable tumor rims). Though
conclusions are difficult to draw in view of our limited sample size,
CT{} and mpMRI habitats may be capturing biologically relevant imaging
phenotypes, potentially serving as non-invasive markers of cancer
aggressiveness. This underscores the potential clinical utility of our
approach, still in an exploratory context.

To our knowledge, this is the first study to systematically evaluate the
precision of 3D voxelwise radiomics features against both kernel radius
and bin size in liver and lung lesions. Direct comparison with prior
work is therefore limited. Previous studies reported higher proportions
of repeatable features (Berenguer et al., 2018; Bernatowicz et al.,
2021; Jha et al., 2021; Mottola et al., 2021), which could be attributed
to differences in the number of RF, perturbation methods (Jha et al.,
2021; Mottola et al., 2021) and the use of phantoms instead of real
patients (Berenguer et al., 2018)

Our finding that texture features are more precise than first-order
features contrasts with some earlier reports (Traverso et al., 2018).
This discrepancy may reflect our focus on voxelwise computation:
histogram features, which summarize intensity distributions, are
inherently less local and may be more stable when computed over large
regions but less stable when computed voxel-by-voxel.

The high sensitivity of features to kernel radius has received little
attention in the literature, yet it has important implications. Many
radiomics studies do not report the kernel radius used, or report it
inconsistently. If a feature\textquotesingle s value changes
substantially when R is varied from 1 mm to 3 mm---a seemingly minor
adjustment---then comparing results across studies becomes difficult.
Our results underscore the need for standardized reporting of all
computational parameters, not just acquisition protocols.

The generalizability of our results is subject to several key
limitations. First, we evaluated only original (unfiltered) features.
Convolutional filters---such as wavelet and Laplacian of Gaussian
filters---are commonly used in radiomics and have been shown to improve
predictive performance in some contexts (Demircioğlu, 2022). Whether
filtered features are more or less precise than original features, and
whether they improve habitat stability, remains unknown. Filter
standardization is still under development, and extending our precision
framework to filtered features is a natural next step.

Second, all segmentations were performed by a single radiologist. While
this ensures consistency, it does not capture variability introduced by
inter-observer differences in lesion delineation. Features depend on
contours, and contours vary across readers. Assessing the impact of
segmentation variability on habitat stability is important future work,
but it was beyond the scope of this chapter. In addition, we did not
evaluate reproducibility across different scanners. Scanner variability
is a well-studied problem in radiomics and has been addressed
extensively elsewhere. Our focus was on sources of variability that
arise even when imaging is performed on the same scanner under identical
protocols---namely, patient repositioning and computational parameter
choices. These sources of variability are less explored and, we argue,
equally important.

Fourth, our precision analysis was limited to CT{}. Whether the same
features are precise in MRI or PET, and whether habitats derived from
different modalities overlap spatially, are open questions. The
exploratory mpMRI analysis in this chapter suggests qualitative
agreement, but quantitative validation requires dedicated studies with
co-registered multimodal imaging and ground truth from spatially matched
biopsies.

Finally, our habitat stability analysis relied on voxel-by-voxel
comparison (Dice coefficient). This may underestimate global similarity:
two habitat maps with slightly shifted boundaries but similar overall
structure would receive a low DSC despite being qualitatively similar.
Alternative metrics---such as mutual information or topology-based
measures---might capture stability more holistically. The modest
absolute DSC values we observed (median \textasciitilde0.65 for precise
features) likely reflect this limitation, along with genuine variability
in how habitats are defined when clustering is applied to noisy
features.

\section{Summary}\label{summary}

Our comprehensive repeatability and reproducibility analysis identified
two subsets of precise RF for effectively computing stable CT{} tumor
habitats in lung and liver lesions. By employing these precise RF and
using unsupervised clustering models, we demonstrated the ability to
identify distinct tumor phenotypes in an exploratory case study. CT{}
tumor habitats correlated with biologically meaningful tumor aspects
such as cellularity, vascularization, and necrosis, but further studies
with larger sample sizes are needed to validate these findings. This
approach to computing CT{} habitats holds great potential for studying
intra-tumoral heterogeneity and monitoring cancer evolution throughout
the course of the disease.

\textbf{Key Points:}

\begin{itemize}
\item
  Voxelwise 3D radiomics features showed poor repeatability (median ICC
  LCL: 0.442) and reproducibility against kernel radius (ICC LCL:
  0.440), but excellent reproducibility against bin size (ICC LCL:
  0.929).
\item
  Of 91 features, 26 were classified as precise for liver lesions. The
  set of precise features differed between liver and lung lesions.
\item
  Habitats computed from precise features were 11\% more stable (median
  DSC: 0.651 vs. 0.587, p \textless0.001) than those computed from all
  features.
\item
  Exploratory validation with mpMRI and histopathology suggested that
  CT{} habitats capture biologically relevant tissue compartments (e.g.,
  necrosis, viable tumor, vascularization differences).
\end{itemize}


