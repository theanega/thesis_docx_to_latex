\chapter{\texorpdfstring{Conclusions }{Conclusions }}\label{conclusions}\label{ch:10}

\begin{enumerate}
\def\labelenumi{\arabic{enumi}.}
\item
  \textbf{Only a minority of handcrafted radiomics features are suitable
  for robust habitat computation.} Of the 91 handcrafted CT radiomics
  features evaluated, only 26 met repeatability and reproducibility
  criteria in liver lesions. Feature stability was strongly influenced
  by computational choices, with kernel radius having a greater impact
  than intensity discretization.
\item
  \textbf{Handcrafted texture features outperform deep learning
  embeddings for CT habitat computation.} Handcrafted features produced
  more spatially contiguous and biologically coherent habitats than
  embeddings from both a liver tumor segmentation model and a foundation
  model. The optimal representation depends on the downstream task, data
  type, and sample size, not model complexity alone.
\item
  \textbf{CT-derived habitats reflect vascular organization rather than
  discrete histological compartments.} While habitats were initially
  expected to distinguish necrosis, fibrosis, and viable tumor, they
  instead consistently mapped gradients of vascular organization:
  avascular cores, cellular--perfused regions, and vascularized rims.
\item
  \textbf{The prognostic value of CT habitat metrics depends on
  context.} Rim entropy was associated with survival after neoadjuvant
  chemotherapy and anti-angiogenic treatment, but not with cytotoxic
  chemotherapy alone, where tumor volume remained dominant. CT habitats
  may complement volumetric assessment when treatment effects are not
  captured by size change.
\item
  \textbf{Prognostic information is spatially concentrated at the
  invasive tumor rim.} Metrics derived from a 2 mm outer rim
  consistently outperformed whole-tumor and core-based measures for
  survival prediction. Rim heterogeneity may capture biologically
  relevant features related to histopathological growth patterns.
  Focusing analysis on the invasive margin, rather than averaging across
  the tumor, preserves this signal and improves prognostic performance.
\end{enumerate}
