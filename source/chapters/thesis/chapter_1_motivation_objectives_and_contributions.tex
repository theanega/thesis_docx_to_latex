\chapter{Motivation, Objectives, and
Contributions}\label{motivation-objectives-and-contributions}\label{ch:1}

\section{Motivation}\label{motivation}

A patient with metastatic colorectal cancer undergoes a radiological
examination after several cycles of therapy. The report indicates that
her liver metastases have neither increased nor decreased in size. For
the oncologist, this finding is ambiguous. The disease may be
biologically controlled, with treatment-induced changes not yet
translating into size reduction. Alternatively, the tumor may be
adapting, with early biological resistance not yet visible as measurable
growth. Size alone cannot distinguish between these possibilities. This
uncertainty, encountered routinely in oncologic practice, motivates the
work presented in this thesis.

Cancer is a leading cause of death worldwide \citep{brayGlobalCancerStatistics2024}, with
most cancer-related mortality driven by metastatic disease rather than
primary tumors \citep{chafferPerspectiveCancerCell2011}. The liver is one of the most
common sites of metastasis across cancer types \citep{tsilimigrasLiverMetastases2021}, and colorectal cancer is the leading source of liver metastases
\citep{hessMetastaticPatternsAdenocarcinoma2006}. Approximately 25\% of patients with colorectal
cancer present with liver metastases at diagnosis, and up to 50\%
develop them during the course of disease. Surgical resection can be
curative, but only a subset of patients are eligible. For the majority,
systemic therapy is the main treatment option, with limited long-term
survival \citep{engstrandColorectalCancerLiver2018}.

Pathology studies have long recognized the biological heterogeneity of
colorectal liver metastases and its clinical relevance. These tumors
mostly contain varying proportions of viable tumor cells, fibrosis (i.e.
scar tissue), and necrosis (i.e. dead cells) \citep{ozakiLiverMetastasesCorrelation2022,poultsidesPathologicResponsePreoperative2012}. Importantly, both the presence of tissue
components and their spatial location within the tumor have therapeutic
implications. Fibrotic tissue at the tumor periphery following treatment
has been associated with improved survival \citep{lataczHistopathologicalGrowthPatterns2022},
whereas extensive necrosis, particularly at the tumor core, has been
linked to more aggressive disease and poorer prognosis \citep{rubbia-brandtImportanceHistologicalTumor2007,vandeneyndenMultifacetedRoleMicroenvironment2013}. Despite their clinical
relevance, such tissue phenotypes can only be assessed through surgical
resection or biopsy. Resection provides comprehensive tissue but is
limited to eligible patients at a single timepoint. Biopsies are more
widely performed, often to obtain molecular biomarkers for treatment
selection, but they sample only a small fraction of the tumor. They
cannot characterize spatial heterogeneity across the lesion or account
for differences between metastases. Treatment decisions for most
patients are therefore made without direct knowledge of tissue
composition or its evolution over time.

CT is the standard imaging modality for the
clinical management of colorectal liver metastases. CT scans are
routinely acquired for staging and treatment monitoring, enabling
repeated, non-invasive assessment of the whole liver over time. However,
the information extracted from CT in clinical practice is largely
size-based, focusing on lesion number and diameter. While size is simple
to measure and interpret, it disregards changes in tissue composition
within tumors (e.g. treatment-induced fibrosis or necrosis) that may
precede or occur independently of volumetric shrinkage.

The field of radiomics treats medical images as quantitative data,
computing radiomic features (i.e. numerical descriptors of intensity
distributions and spatial texture patterns) to extract information
beyond visual assessment \citep{lambinRadiomicsExtractingMore2012}. Most radiomics studies
summarize each tumor using features averaged across the entire lesion,
implicitly assuming that global aggregation is sufficient to
characterize tumor biology despite known intratumor heterogeneity.
Habitat imaging challenges this assumption \citep{gatenbyQuantitativeImagingCancer2013}.
Rather than collapsing tumors into summary statistics, habitat imaging
partitions lesions into spatially distinct subregions based on imaging
characteristics. These subregions, referred to as habitats, may
correspond to different tissue phenotypes within the same tumor.

To date, habitat imaging has been applied predominantly to magnetic
resonance imaging (MRI) \citep{liMRIbasedHabitatImaging2024}. mpMRI
combines multiple quantitative sequences acquired in the same imaging
session, each sensitive to different tissue properties. For instance,
the ADC reflects tissue cellularity \citep{lebihandImagerieDiffusionVivo1985}, and the volume transfer constant,
K\textsuperscript{trans} , provides information on vascular permeability
and perfusion \citep{toftsMeasurementBloodbrainBarrier1991}. These biologically interpretable
maps make MRI particularly suitable for studying intratumor
heterogeneity in research settings. In contrast, CT-derived habitats are
generally unexplored, despite CT being the dominant modality for
managing colorectal liver metastases in routine clinical practice. This
imbalance represents a translational limitation: methods developed for
MRI are not readily applicable to clinical workflows where CT is
standard.

Moreover, the habitat imaging literature lacks systematic evaluation of
how data representation influences habitat computation. Tumor
characteristics can be encoded using handcrafted (i.e. predefined)
radiomics features \citep{vangriethuysenComputationalRadiomicsSystem2017} or learned embeddings
from deep learning models, in which image representations are optimized
automatically rather than explicitly defined \citep{paiVisionFoundationModels2025}. While
learned representations are often assumed to be superior, their
suitability for CT-derived habitats has not been established.

In addition to representation choices, important methodological and
translational limitations remain. Feature robustness is often
insufficiently assessed, biological validation is frequently qualitative
or restricted to preclinical settings, and clinical relevance is rarely
benchmarked against tumor volume, the current standard imaging
biomarker. These limitations are particularly relevant for colorectal
liver metastases, where systemic therapies may alter tissue composition
before inducing measurable size change. Anti-angiogenic agents, for
example, target tumor vasculature and can modify perfusion and tissue
organization without producing immediate shrinkage. In such settings,
biomarkers that capture tissue composition changes could improve both
treatment assessment and patient stratification.

These gaps motivate the central question of this thesis: can routine CT
imaging capture biologically and clinically meaningful intratumor
heterogeneity in colorectal liver metastases? Answering this requires
assessing technical robustness of CT-derived features, evaluating which
data representations produce biologically coherent habitats,
characterizing what tissue phenotypes these habitats represent, and
demonstrating clinical value beyond tumor volume. The following section
formalizes these problems as five research questions.

\section{Hypothesis and Objectives}\label{hypothesis-and-objectives}

This thesis aims to develop and validate a biologically grounded CT
habitat imaging framework to characterize intratumor heterogeneity in
colorectal liver metastases, and to assess its clinical relevance. This
aim is based on the hypothesis that CT-derived habitats capture
biologically meaningful heterogeneity that provides clinically relevant
information beyond tumor volume. Towards this aim, we pose the following
research questions:

\begin{enumerate}
\def\labelenumi{\arabic{enumi}.}
\item
  Which CT handcrafted radiomics features are repeatable and
  reproducible enough to support stable habitat computation?
\item
  Which CT data representation produces habitats that best separate
  tissue with different cellularity and vascularity?
\item
  What tissue phenotypes do CT-derived habitats represent?
\item
  Do CT-derived habitats provide prognostic information independent of
  tumor volume?
\item
  If prognostic information exists, is it spatially localized (i.e. at
  the core or the rim), or is it uniformly distributed within a tumor?
\end{enumerate}

These research questions correspond to five specific objectives:

\begin{enumerate}
\def\labelenumi{\arabic{enumi}.}
\item
  \textbf{Objective 1:} To identify precise CT handcrafted radiomics
  features suitable for robust habitat computation. \emph{(Chapter~\ref{ch:6})}
\item
  \textbf{Objective 2:} To determine the optimal CT data
  representation for capturing biologically distinct tissue regions.
  \emph{(Chapter~\ref{ch:7})}
\item
  \textbf{Objective 3:} To characterize the biological meaning of
  CT-derived habitats using multiparametric MRI and histopathology.
  \emph{(Chapter~\ref{ch:7})}
\item
  \textbf{Objective 4:} To evaluate whether CT-derived habitats
  provide clinically relevant information beyond tumor volume.
  \emph{(Chapter~\ref{ch:8})}
\item
  \textbf{Objective 5:} To investigate the spatial localization of
  clinically relevant heterogeneity within tumors. \emph{(Chapter~\ref{ch:8})}
\end{enumerate}

\section{Contributions to Knowledge}\label{contributions-to-knowledge}

This thesis contributes to the field of imaging-based tumor
heterogeneity analysis by addressing key methodological, biological, and
clinical limitations of existing habitat imaging approaches.

\subsubsection{Main Contributions}\label{main-contributions}

\begin{enumerate}
\def\labelenumi{\arabic{enumi}.}
\item
  \textbf{An open-source, modality-agnostic habitat imaging pipeline was
  developed for the first time.} The pipeline accepts any voxelwise
  feature maps as input and outputs cluster assignments and
  habitat-derived metrics. It is described in Chapter 5 and available
  at: \url{https://github.com/radiomicsgroup/imaging-habitats-pipeline}.
\item
  \textbf{The precision of voxelwise CT}{} \textbf{radiomics features
  for habitat imaging was evaluated.} This is the first comprehensive
  assessment of repeatability and reproducibility for 3D voxelwise
  features in liver and lung lesions. This work is described in Chapter
  6 and was published as:

  \begin{itemize}[label={--}]
  \item
    \textbf{Prior O}, Macarro C, Navarro V, et al; Bernatowicz K,
    Perez-Lopez R. Identification of Precise 3D CT Radiomics for Habitat
    Computation by Machine Learning in Cancer. \emph{Radiol Artif Intell.}
    2024;6(2):e230118. DOI:
    \href{https://doi.org/10.1148/ryai.230118}{10.1148/ryai.230118}.
  \end{itemize}

\item
  \textbf{Multiple CT}{} \textbf{data representations were compared for
  the first time for habitat imaging.} Raw Hounsfield units, handcrafted
  radiomics, and deep learning embeddings were evaluated using
  co-registered multiparametric MRI as biological reference. This work
  is described in Chapter~\ref{ch:7}.
\item
  \textbf{A biologically anchored CT}{} \textbf{habitat framework was
  developed.} Habitat definitions were constrained by tissue properties
  (cellularity, vascularity) measured with quantitative MRI rather than
  post-hoc outcome associations. This work is described in Chapter~\ref{ch:7}.
  Preliminary results were presented at international meetings as:

  \begin{itemize}[label={--}]
  \item
    \textbf{Prior O}; Grussu F; Garcia-Ruiz A; et al; Perez-Lopez R.
    Decoding liver intra-tumour heterogeneity with co-localized CT and
    multi-parametric MRI. \emph{ISMRM Diffusion MRI Workshop} (Amsterdam,
    The Netherlands, 2022). Oral presentation.
  \item
    \textbf{Prior O}; Grussu F; Macarro C; et al; Perez-Lopez R.
    Dissecting heterogeneity in liver metastases: an mpMRI and CT
    approach. \emph{ISMRM Iberian Chapter} (Valladolid, Spain, 2023).
    Poster.
  \item
    \textbf{Prior O}; Macarro C; Grigoriou A; et al; Perez-Lopez R.
    Non-invasive Characterization of Intratumor Heterogeneity: Comparing
    MRI-Radiomics and Histological Habitats. \emph{ISMRM Iberian Chapter}
    (Porto, Portugal, 2024). Oral presentation.
  \end{itemize}


\item
  \textbf{The clinical relevance of CT}{} \textbf{habitats was
  demonstrated to depend on context.} Habitat metrics were prognostic in
  anti-angiogenic and post-neoadjuvant settings but not with
  chemotherapy alone, and added information beyond tumor volume. This
  work is described in Chapter~\ref{ch:8}.
\item
  \textbf{Prognostic information was shown to concentrate at the
  invasive rim.} Rim-based metrics outperformed whole-tumor
  heterogeneity measures. This work is described in Chapter~\ref{ch:8}.
\end{enumerate}

\begin{itemize}[label={--}]
\item
  \emph{Manuscript in preparation (Contributions 3--6):} \textbf{Prior
  O}, et al; Grussu F, Perez-Lopez R. Mapping Tumor Heterogeneity of
  Colorectal Liver Metastases via CT Habitats Improves Prognosis.
  \emph{In preparation.}
\end{itemize}

\subsubsection{Other Contributions During the
PhD}\label{other-contributions-during-the-phd}

\textbf{\ul{Publications}}

\begin{itemize}[label={--}]
\item
  Ligero M, Gielen B, Navarro V, \textbf{Prior O}, et al; Perez-Lopez R.
  A whirl of radiomics-based biomarkers in cancer immunotherapy, why is
  large scale validation still lacking? \emph{npj Precis. Onc.}
  2024;8(1):42. DOI:
  \href{https://www.nature.com/articles/s41698-024-00534-9}{10.1038/s41698-024-00534-9.}
\item
  Bernatowicz K, Amat R, \textbf{Prior O}, et al; Perez-Lopez R.
  Radiomics signature for dynamic monitoring of tumor inflamed
  microenvironment and immunotherapy response prediction. \emph{J
  Immunother Cancer.} 2025;13(1):e009140. DOI:
  \href{https://www.google.com/search?q=https://doi.org/10.1136/jitc-2024-009140}{10.1136/jitc-2024-009140}.
\item
  de Grandis MC, Baraibar I, \textbf{Prior O}, et al; Perez-Lopez R.
  Differentiating low tumor burden from oligometastatic disease in
  colorectal cancer: a call for individualized therapeutic approaches.
  \emph{ESMO Open.} 2025;10(8):105520. DOI:
  \href{https://doi.org/10.1016/j.esmoop.2025.105520}{10.1016/j.esmoop.2025.105520~}.
\item
  Voronova AK, \textbf{Prior O}, Grigoriou A, et al; Perez-Lopez R.
  Simulation-informed evaluation of microvascular parameter mapping for
  diffusion MR imaging of solid tumours. \emph{medRxiv.} 2025. DOI:
  \href{https://www.google.com/search?q=https://doi.org/10.1101/2025.08.27.25334553}{10.1101/2025.08.27.25334553}.
\end{itemize}

\textbf{\ul{Conference Presentations}}

\begin{itemize}[label={--}]
\item
  \textbf{Prior O}; Bernatowicz K; Ligero M; et al; Perez-Lopez R.
  Artificial Intelligence for Predicting Response to Standard of Care
  Therapy in MSS RASmt mCRC Patients. \emph{Cancer Core Europe Summer
  School} (Albufeira, Portugal, 2022). Oral presentation.
\item
  \textbf{Prior O}; Gielen B; Ligero M; et al; Perez-Lopez R.
  Translating Imaging into Insight: Can Radiology and Machine Learning
  predict Immunotherapy Response in a Multi-Tumor Landscape?
  \emph{ASEICA 40th Anniversary Congress} (A Coruña, Spain, 2023). Oral
  presentation.
\end{itemize}

\textbf{\ul{International Research Projects}}

\begin{itemize}
\item
  \textbf{COLOSSUS (EU-funded H2020 project, led by the Royal College of
  Surgeons in Ireland):} Contribution to the development of candidate
  imaging biomarkers for colorectal cancer patients.

  \begin{itemize}[label={--}]
  \item
    \emph{Manuscript in preparation:} Connor K*, \textbf{Prior O.}*,
    Shiels LP, et al; Byrne AT. ``Co-clinical CT radiomics pipeline to
    establish candidate imaging biomarkers for colorectal cancer.'' (*equal
    contribution)
  \end{itemize}
\end{itemize}


\begin{itemize}
\item
  \textbf{POEM Study (led by the Oncology Center Antwerp, Belgium)}:
  Participation in the prospective characterization of histological
  growth patterns of liver metastases.

  \begin{itemize}
  \item
    Latacz E, Prior Palomares O., Ruiz Roig N, et al. "Prospective
    complete histopathological characterization of liver metastases from
    colorectal and breast carcinoma to predict the histopathological
    growth patterns by medical imaging (POEM)." \emph{Eur J Surg Oncol.}
    2024.
    DOI:~\href{https://doi.org/10.1016/j.ejso.2024.109241}{10.1016/j.ejso.2024.109241~}.
  \end{itemize}
\end{itemize}

\textbf{\ul{Internal Pipelines - Radiomics Group at VHIO}}

\begin{itemize}
\item
  \textbf{CT}{} \textbf{preprocessing pipeline}: Developed and
  implemented standardized CT preprocessing pipelines for the
  Radiomics Group at VHIO, ensuring reproducibility in feature
  extraction across imaging datasets from multiple clinical studies.
\item
  \textbf{Machine learning framework}: Developed a reusable machine
  learning framework for the Radiomics Group with emphasis on nested
  cross-validation to ensure model generalizability and prevent data
  leakage.
\end{itemize}

\section{Thesis Structure}\label{thesis-structure}

This thesis consists of five parts across ten chapters. Part I
introduces the problem and research objectives (\textbf{Chapter~\ref{ch:1}}).
Part II provides foundations: medical imaging techniques
(\textbf{Chapter~\ref{ch:2}}), oncological context with focus on colorectal liver
metastases and treatment response assessment (\textbf{Chapter 3}), and
imaging heterogeneity methods including the state of the art in habitat
imaging (\textbf{Chapter~\ref{ch:4}}). Part III describes data sources, the
general methodological framework, and the habitat imaging pipeline
developed for this work (\textbf{Chapter~\ref{ch:5}}). Part IV presents the core
experimental contributions: precision analysis of radiomics features
(\textbf{Chapter~\ref{ch:6}}), development and biological validation of CT
habitats using mpMRI (\textbf{Chapter~\ref{ch:7}}), and clinical
relevance assessment across treatment contexts (\textbf{Chapter~\ref{ch:8}}).
Part V synthesizes findings and draws conclusions: general discussion
including limitations and future directions (\textbf{Chapter~\ref{ch:9}}), and
main conclusions (\textbf{Chapter~\ref{ch:10}}).

Each experimental chapter is self-contained and can be read
independently. Sequentially, they follow a logical progression: from
identifying reliable imaging features, to constructing
biologically-informed habitats, to demonstrating their clinical utility.
Extended methods and supplementary results are provided in the Annexes.

