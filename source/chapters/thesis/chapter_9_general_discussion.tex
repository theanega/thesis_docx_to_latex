\chapter{General Discussion}\label{general-discussion}\label{ch:9}

This thesis set out to determine whether routine contrast-enhanced CT
imaging can capture biologically and clinically meaningful intratumor
heterogeneity in colorectal liver metastases. The work was structured
around five interconnected research questions, each building on the
findings of the previous one. Taken together, the results support a
qualified affirmative answer: CT can capture meaningful tumor
heterogeneity, but only under specific conditions. In this chapter, the
main findings of the thesis are synthesized, their biological and
clinical implications are discussed, and directions for future research
are outlined.

\section{Synthesis of Findings}\label{synthesis-of-findings}

The \textbf{central hypothesis} of this thesis was that CT-derived
habitats could capture biologically meaningful heterogeneity and provide
clinically relevant information beyond tumor volume. This hypothesis was
supported, with important qualifications: CT habitats primarily reflect
vascular organization rather than discrete histological compartments,
and their prognostic value depends on context.

\begin{itemize}
\item
  \textbf{RQ1: Which CT handcrafted radiomics features are repeatable
  and reproducible enough to support stable habitat computation?}
  \emph{(Chapter 6)}
\end{itemize}

Only a minority of handcrafted features met repeatability and
reproducibility criteria in liver lesions. Out of 91 features evaluated,
26 were sufficiently precise to support robust habitat computation.
Feature stability was more sensitive to kernel radius than to intensity
discretization, and the set of robust features identified in liver
tumors differed from those identified in lung. These findings highlight
the site-specific nature of radiomics precision and the need for task-
and anatomy-specific validation.

\begin{itemize}
\item
  \textbf{RQ2: Which CT}{} \textbf{data representation produces habitats
  that best separate tissue with different cellularity and vascularity?}
  \emph{(Chapter~\ref{ch:7})}
\end{itemize}

Contrary to expectations, handcrafted texture features produced habitats
with stronger biological coherence than deep learning embeddings derived
from both a liver tumor segmentation model and a foundation model.
Handcrafted features yielded more spatially contiguous clusters and
greater separation on co-registered mpMRI metrics of cellularity and
perfusion (ADC, K\textsuperscript{trans}, and f\textsubscript{v}).

\begin{itemize}
\item
  \textbf{RQ3: What tissue phenotypes do CT}{}\textbf{-derived habitats
  represent?} \emph{(Chapter~\ref{ch:7})}
\end{itemize}

Rather than mapping onto necrosis, fibrosis, and viable tumor, CT
habitats consistently reflected gradients of vascular organization:
avascular cores, transitional cellular--perfused regions, and
vascularized rims at the tumor--liver interface. Histopathological
assessment confirmed qualitative spatial correspondence between habitats
and tissue composition, but the dominant signal was vascular rather than
categorical histology.

\begin{itemize}
\item
  \textbf{RQ4: Do CT-derived habitats provide prognostic information
  independent of tumor volume?} \emph{(Chapter~\ref{ch:8})}
\end{itemize}

Habitat-derived metrics added prognostic information beyond tumor volume
in specific clinical contexts. Rim entropy predicted survival after
neoadjuvant chemotherapy in resectable disease and prior to
anti-angiogenic therapy in unresectable disease. In contrast, with
cytotoxic chemotherapy alone, tumor volume remained the dominant
predictor.

\begin{itemize}
\item
  \textbf{RQ5: If prognostic information exists, is it spatially
  localized (i.e. at the core or the rim), or is it uniformly
  distributed within a tumor?} \emph{(Chapter~\ref{ch:8})}
\end{itemize}

Across cohorts and treatment settings, metrics derived from a 2 mm outer
rim consistently outperformed whole-tumor and core-based metrics.
Exploratory pathology-correlated exploratory analyses suggested that rim
entropy may reflect histopathological growth patterns, with higher
entropy observed in desmoplastic growth compared to replacement growth.

\section{Biological Interpretation of CT-Derived Habitats}\label{biological-interpretation-of-ct-derived-habitats}

The findings of this thesis indicate that CT-derived habitats in
colorectal liver metastases do not primarily represent discrete
histological compartments such as necrosis, fibrosis, and viable tumor.
Instead, they capture gradients of vascular organization and perfusion,
with a consistent spatial structure across lesions. This observation is
biologically plausible given the physics of contrast-enhanced CT, which
is inherently sensitive to iodine distribution within blood vessels.

Across cohorts, habitats consistently delineated an avascular tumor
core, a transitional cellular--perfused zone, and a vascularized rim at
the tumor--liver interface. This spatial organization aligns with known
aspects of tumor biology. The invasive margin is the region where
angiogenesis occurs, where tumor cells interact with host tissue, and
where histopathological growth patterns are defined. In colorectal liver
metastases, desmoplastic growth is characterized by a fibrotic rim
separating tumor cells from liver parenchyma, whereas replacement growth
involves direct infiltration of tumor cells along hepatic sinusoids
without an intervening fibrotic barrier. The consistent concentration of
prognostic information at the invasive rim suggests that CT-derived
habitats may be sensitive to these growth-related vascular
architectures. Tumors with heterogeneous rims may reflect mixed or
unstable angiogenic patterns, variable perfusion, or coexistence of
distinct growth behaviors within the same lesion. When heterogeneity is
averaged across the entire tumor volume, this signal is diluted,
explaining the superior performance of rim-based metrics compared to
whole-tumor measures.

These findings also provide a framework for understanding the
context-dependent prognostic value of habitat-derived metrics. Cytotoxic
chemotherapy primarily induces tumor cell death, leading to volumetric
shrinkage that is well captured by size-based criteria. Anti-angiogenic
therapies, in contrast, target the tumor vasculature, often altering
tissue organization and perfusion without producing immediate changes in
tumor diameter. In this setting, heterogeneity in contrast enhancement
at the invasive margin captures treatment-relevant biology that is
invisible to volumetric assessment.

A recurring ambition in imaging research is the concept of a ``virtual
biopsy'', implying that non-invasive imaging might replace tissue
sampling in the future. While appealing, this framing ignores the
complementary nature of radiology and pathology. These modalities
operate at different spatial scales and capture different aspects of
tumor biology.

Histopathology provides molecular specificity and cellular detail,
enabling assessment of genetic alterations, microarchitecture, and
immune contexture. Imaging, in contrast, provides spatial context across
the entire tumor and its surroundings, non-invasively and
longitudinally. As of today, a habitat map cannot identify a BRAF
mutation, but it can characterize vascular organization across the whole
lesion at every timepoint. This information cannot be provided by a
single biopsy due to sampling constraints. CT-derived habitats should
therefore be understood as a distinct representation of tumor biology
rather than an imperfect surrogate for histology. As illustrated in
\Cref{fig:9.1}, different representations of the
same object emphasize different features without one being inherently
superior. The clinical value of a model lies not in its fidelity to
histology, but in its usefulness for decision-making.

\begin{figure}[t!]
\centering
\includegraphics[width=0.7\textwidth]{fig_9_1.png}
\caption[Different representations capture different information]{Different representations capture different information.
Leonardo da Vinci's anatomical sketch and finished
portrait of the Mona Lisa depict the same subject but emphasize
different features. Similarly, histopathology and imaging habitats are
both valid representations of tumor biology, suited to different
purposes. Adapted from \citep{novikovModeling2018}.}
\label{fig:9.1}
\end{figure}

\section{Technical and Translational Considerations}\label{technical-and-translational-considerations}

Several limitations must be acknowledged when considering the
translational potential of habitat imaging. First, all clinical analyses
performed in this thesis were retrospective. Although associations
between rim-based habitat metrics and outcome were consistent across
cohorts and treatment contexts, prospective validation is required
before these metrics can be used to inform clinical decision-making.

A further limitation concerns the treatment of multiple liver
metastases. In this thesis, habitat-derived metrics were aggregated at
the patient level using volume-weighted averages, implicitly assuming
that larger tumors carry more clinically relevant information. While
this is a reasonable first approximation, it does not account for
dissociated responses, where some metastases may respond biologically
despite stable or increasing size in others. Before habitat imaging can
be evaluated in clinical trials, it will be essential to study how
\emph{inter}tumor heterogeneity (i.e. between tumors) should be
combined, whether specific lesions dominate outcome, and how conflicting
signals across metastases should be interpreted.

Moreover, habitat imaging relies on a multi-step computational pipeline
involving tumor segmentation, feature extraction, and clustering. Each
step introduces variability and design choices. One reason tumor size
remains the dominant imaging biomarker in oncology is its simplicity: it
is easy to measure and widely interpretable. In contrast, habitat
imaging currently requires multiple parameters to be specified, which
may limit robustness and reproducibility. An important byproduct of this
thesis is the development of a complete, end-to-end CT habitat imaging
pipeline implemented from first principles. Although methodological
standardization was not an explicit objective, addressing the research
questions posed here required constructing a transparent and
reproducible framework. This represents the first fully described
habitat imaging pipeline and constitutes an initial step toward
standardization. Future work should aim to simplify this framework,
establish robust defaults, and enable automated deployment suitable for
clinical workflows.

Within this technical context, the choice of feature representation
deserves particular attention. During the course of this thesis,
expectations surrounding medical image analysis shifted rapidly with the
adoption of deep learning and the emergence of foundation models
\citep{paschaliFoundationModelsRadiology2025}. These have transformed representation learning
in other fields such as pathology \citep{songArtificialIntelligenceDigital2023} and therefore it
was reasonable to expect that learned embeddings would outperform
handcrafted radiomics features for habitat computation. In this task and
with the datasets used in this thesis, however, handcrafted texture
features produced more spatially coherent and biologically meaningful CT
habitats than deep learning embeddings derived from pretrained
segmentation and foundation models. This likely reflects constraints
imposed by data size and task type: first, none of the evaluated models
were optimized for spatial clustering or habitat discovery; second, in
limited datasets, lower-dimensional and interpretable representations
such as handcrafted features may also be less prone to overfitting.

These findings do not argue against deep learning, but rather emphasize
that representation choice must be aware of both the task and the scale.
Many clinically established imaging biomarkers, including tumor size and
the multiparametric MRI maps used in this thesis, are handcrafted and
remain valuable due to their interpretability and biological grounding.
Learned representations may eventually surpass them, but such claims
must be demonstrated empirically for each application.

Additional challenges include robustness across scanners and
institutions, dependence on tumor segmentation accuracy, and extension
of biological validation beyond qualitative histopathological
correspondence. One promising approach to address this last challenge
involves imaging resected liver metastases with ex vivo MRI and CT using
preclinical scanners. Such datasets would enable accurate
co-registration across pathology and radiology, providing a quantitative
link between voxel-level imaging habitats and underlying tissue
composition.

\section{Final Reflections}\label{final-reflections}

The global burden of cancer is projected to increase substantially over
the coming decades. At the same time, advances in targeted therapies,
immunotherapy, and supportive care raise the possibility that many
cancers may increasingly be managed as chronic diseases rather than
acute, fatal conditions. In this context, the need for robust,
non-invasive, and informative biomarkers becomes even more urgent.

Medical imaging occupies a unique position in this landscape. It is
routinely acquired, longitudinal, and captures the entire tumor and its
microenvironment. As emphasized in recent perspectives on artificial
intelligence in cancer research \citep{changHallmarksArtificialIntelligence2025}, the
challenge is no longer the lack of data, but the ability to extract
meaningful and clinically relevant information from complex datasets.
This thesis argues that part of this challenge can be addressed by
asking better questions of the imaging data already available. Habitat
imaging seeks to move beyond the question of how large a tumor is,
toward how it is organized and how it interacts with host tissue.
Inevitably, this framework introduces new complexities and opens new
methodological questions, which future PhD theses will need to address.

This work began with the image of a patient whose CT scan showed stable
disease, offering little insight into whether her tumor was responding
or adapting. One can envision a future in which intratumor heterogeneity
assessment becomes a routine component of imaging interpretation,
allowing radiologists to ask not only how large a tumor is, but how it
is responding biologically. Narrowing the gap between what imaging
captures and what clinicians need will require continued biological
validation, technical rigor, and prospective testing, but the
information is already present in the images we acquire every day.


