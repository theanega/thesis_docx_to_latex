\chapter{Colorectal Liver Metastases}\label{colorectal-liver-metastases}

This chapter provides the clinical and biological context for
understanding the importance of imaging heterogeneity in colorectal
liver metastases.

\section{Background}\label{background}

Liver metastases are tumors that originate from a primary cancer
elsewhere in the body and establish growth in hepatic tissue
(Tsilimigras et al., 2021). The liver is one of the most common
metastatic sites regardless of primary tumor type (Budczies et al.,
2015). There are two main theories explaining this organotropism (i.e.
preference of cancer for specific organs). On one hand, Paget (Paget,
1889) proposed that metastasis requires compatibility between tumor
cells (the "seed") and the host organ (the "soil"). The liver is highly
vascularized, rich in growth factors, and immunologically more tolerant
than other organs (Valderrama-Treviño et al., 2017). These features make
it a hospitable environment for circulating cancer cells. On the other
hand, Ewing (Ewing, 1919) emphasized the role of blood flow. The liver
receives a dual blood supply: oxygenated blood from the hepatic artery
and blood from the portal vein, which drains the intestines and other
abdominal organs. For tumor cells that enter the circulation from an
abdominal primary, the liver is often the first organ they encounter.

Among solid tumors, colorectal cancer is the most common source of liver
metastases, followed by pancreatic cancer, breast cancer, lung cancer,
and melanoma (Hess et al., 2006). Colorectal cancer originates in the
epithelial lining (i.e. tissue walls) of the colon or rectum. It is the
third most common malignancy worldwide, with approximately 1.9 million
new cases and over 900,000 deaths annually (Morgan et al., 2023). Around
25\% of patients present with liver metastases at diagnosis, and half
will develop them during their disease course (Maher et al., 2017). This
results in a significantly reduced life expectancy with a 5-year overall
survival of 17\% compared to patients without liver metastases (5-years
survival rate of 70\%) (Engstrand et al., 2018). Tumors that originate
in the left side of the colon are more likely to metastasize to the
liver, whereas right-sided tumors more commonly spread fo the peritoneum
(i.e. membrane that covers abdominal organs) (Siegel, Jakubowski, et
al., 2020).

\section{Histopathological Heterogeneity}\label{histopathological-heterogeneity}

Colorectal liver metastases are not uniform masses. Different tissue
types coexist within the same lesion, and their proportions vary from
tumor to tumor (\hyperref[_Ref219653911]{Figure 3.1}A). The three main
tissue types in colorectal liver metastases are viable tumor, fibrosis,
and necrosis (Bird et al., 2006). Viable tumor consists of densely
packed cancer cells, usually forming glandular patterns inherited from
the primary colorectal tumor. These regions have high blood supply to
match their metabolic demands. Fibrosis is the deposition of collagen
and extracellular matrix by stromal cells. Necrosis represents cell
death, typically resulting from inadequate blood supply or treatment
effect. Necrotic regions lack cellular architecture and contain dying
cells or cellular debris.

In the histopathology literature of colorectal liver metastases, we can
find two types of necrosis: usual necrosis and infarct-like necrosis
(ILN) (H. H. L. Chang et al., 2012). Usual necrosis is characterized by
cell debris mixed with some viable cells while ILN presents as large
confluent areas with absent or minimal cell debris. It is hypothesized
that usual necrosis is the result of the tumor outgrowing its blood
supply while ILN is a consequence of treatment and may represent an
intermediate stage in the evolution from viable tumor to fibrosis
(Loupakis et al., 2013). These two definitions have been relevant in
research settings but are not routinely distinguished in clinical
practice.

These tissue types do not distribute randomly within tumors, as can be
observed in \hyperref[_Ref219653911]{Figure 3.1}. A common spatial
pattern is concentric organization: necrosis or fibrosis predominates at
the center while viable tumor glands are located at the periphery. The
presence of viable cells at the periphery may be explained by the
periphery\textquotesingle s proximity to healthy liver vasculature. A
scattered distribution of tissue types throughout the metastasis, or a
pattern with viable cells at the center, is less commonly observed.

Importantly, necrosis, fibrosis and viable tumor carry prognostic
significance, though interpretation depends on context. Baseline
necrosis (i.e. present before treatment, usual necrosis) suggests
aggressive biology and predicts worse outcomes (Van den Eynden et al.,
2013), perhaps due to poor drug penetration into the tumor core (Wong \&
Neville, 2007). Fibrosis, in contrast, is linked to improved outcomes
after treatment, following the hypothesis that when tumor cells are
killed, the liver replaces them with scar tissue. As a matter of fact,
fibrosis is the only tissue type that is formally measured. This only
happens when patients have been selected for surgery and have received
presurgical (i.e. neoadjuvant) treatment. After surgical resection,
pathologists evaluate resected tumors using tumor regression grading
(TRG), a histological scoring system that quantifies the extent of
residual viable tumor versus fibrosis (Rubbia-Brandt et al., 2007).

In this system, TRG 1 indicates complete regression with no residual
tumor and dense fibrosis, while TRG 5 indicates no response, with
abundant viable tumor and no fibrosis. This scoring system illustrates
the prognostic importance of fibrosis. TRG, however, can only be
assessed in patients who undergo surgery, which represents a minority of
patients with colorectal liver metastases. For the majority, tissue
composition remains unknown throughout treatment.

Another tissue type that may be present in colorectal liver metastases
is mucin (i.e. acellular collections of mucus). This occurs in mucinous
adenocarcinomas, which represent 10--15\% of colorectal cancers (Hugen
et al., 2016). Colorectal liver metastases with mucinous histology are
not studied in this thesis.

Beyond internal composition, the interface between tumor and liver
carries prognostic significance. This interaction has been characterized
through histopathological growth patterns (HGPs), described not only for
colorectal liver metastases but for liver metastases from other
primaries as well (Latacz et al., 2022) (\hyperref[_Ref219653911]{Figure
3.1}B). Liver metastases present three main HGPs: desmoplastic,
replacement and less common pushing pattern. Desmoplastic, also known as
encapsulated, growth is characterized by a fibrous rim separating tumor
from liver parenchyma. It is not clear whether this rim is a reaction
from the liver or a consequence of the tumor composition determining
this specific growth. Tumors with desmoplastic HGP associate with
improved outcomes. Replacement growth occurs when tumor cells infiltrate
along existing liver architecture, co-opting host vasculature rather
than inducing new vessel formation. This pattern is associated with
worse prognosis and has been hypothesized to be the default pattern of
successful aggressive tumor invasion (Fernández Moro et al., 2023).
Pushing growth, which is less common, describes tumors that compress but
do not infiltrate surrounding liver. The tumor-liver boundary is
well-defined but lacks a fibrous capsule, and prognosis is intermediate.

The distinction between these patterns has therapeutic implications.
Solid tumors require blood supply to grow beyond a few millimeters. They
can acquire this supply through two mechanisms: angiogenesis or vascular
co-option. Angiogenesis is the formation of new blood vessels, typically
driven by hypoxia. When tumors outgrow their blood supply, hypoxic
regions upregulate vascular endothelial growth factor (VEGF), which
promotes new vessel formation. Vascular co-option occurs when tumor
cells `hijack' existing vasculature without inducing new vessels.
Replacement-pattern tumors rely on vascular co-option: they grow along
the liver\textquotesingle s vascular network rather than building their
own vessels. Anti-angiogenic therapy (e.g. bevacizumab) targets
VEGF-driven new vessel formation, so tumors that co-opt existing vessels
may be inherently resistant (Frentzas et al., 2016).

\begin{figure}[htbp]
\centering
\includegraphics[width=0.95\textwidth]{fig_3_1.png}
\caption{Histopathological heterogeneity in colorectal liver metastases.
(A) Representative histology sections of a colorectal liver metastasis
showing the three dominant tissue compartments: viable tumor (cellular,
vascularized), necrosis (central, avascular), and fibrosis (often
treatment-induced). (B) The two main histopathological growth patterns
at the tumor-liver interface: desmoplastic (fibrous rim indicated by
arrowheads) and replacement (infiltration along liver architecture
indicated by arrows). Asterisks indicate areas of infarct-like necrosis.
Adapted from Frentzas et al., 2016.}
\label{fig:3.1}
\end{figure}

\section{Treatment Landscape and Selection Challenges}\label{treatment-landscape-and-selection-challenges}

The prognosis for patients with metastatic colorectal cancer (mCRC) is
poor, but outcomes vary substantially depending on whether the tumors
can be resected or not. This distinction, primarily determined by CT{}
imaging, separates patients into two fundamentally different treatment
paths: surgery or systemic therapy.

\subsection{3.3.1. Resectable disease}\label{resectable-disease}

Hepatic resection is considered to be the only potentially curative
treatment. The goal is complete resection with negative margins (i.e.
resection margins free of tumor cells) while preserving enough
functional liver volume. Eligibility for hepatic resection depends on
patient fitness, absence of other distant metastases, tumor number,
size, and location. CT{} imaging plays a central role in this
assessment. Five-year survival rates after resection are 58\% compared
with 27\% for patients without resection (Zeineddine et al., 2023).

For patients whose disease is initially unresectable due to tumor burden
or location, neoadjuvant chemotherapy can sometimes reduce tumor size
sufficiently to enable surgery. This conversion strategy has expanded
the population eligible for potentially curative resection. The response
to neoadjuvant treatment also provides prognostic information: patients
whose tumors respond favorably have better long-term outcomes even after
complete resection. As discussed in Section 3.2, neoadjuvant therapy
does not merely shrink tumors but remodels their internal composition,
and pathological response (assessed via TRG) correlates with survival.

\subsection{Unresectable disease}\label{unresectable-disease}

The majority of patients with colorectal liver metastases are not
candidates for surgery. For these patients, systemic therapy is given
with the goal of controlling the disease, prolonging survival, and
maintaining quality of life. There are three main types of treatment:
chemotherapy, targeted therapies and immunotherapy. Treatment selection
is guided by molecular biomarkers and primary tumor sidedness
(\hyperref[_Ref219654146]{Figure 3.2}).

Standard chemotherapy regimens combine fluoropyrimidines (5-fluorouracil
or capecitabine) with either oxaliplatin (FOLFOX) or irinotecan
(FOLFIRI). These agents target rapidly dividing cells by disrupting DNA
synthesis and repair. When effective, they kill tumor cells, which is
why we refer to chemotherapy as being a cytotoxic (i.e. causing damage
or death to cells) treatment. Unlike chemotherapy, targeted therapies
and immunotherapy are cytostatic (i.e. inhibiting cell growth and
proliferation, which may lead to cell death eventually). ~In
immunotherapy, the patient's immune system is enhanced to recognize and
control cancer cells by blocking inhibitory signals, thereby slowing
tumor growth without immediately killing cells outright. Immunotherapy
has transformed outcomes for mCRC patients, with median overall survival
being more than twice as long as in patients treated with chemotherapy
(André et al., 2025). Only 5\% of mCRC patients are eligible for
approved immunotherapy (Bari et al., 2025). These are patients with
microsatellite instability (MSI) status, which indicates defective DNA
mismatch repair system.

Targeted therapies are treatments that target specific proteins involved
in tumor growth and survival. In metastatic colorectal cancer, the main
targets are the epidermal growth factor receptor (EGFR) and vascular
endothelial growth factor (VEGF). EGFR is a receptor on the cell surface
that, when activated, triggers signaling pathways promoting cell
proliferation and survival. The RAS proteins (KRAS and NRAS) are
downstream components of this pathway. In tumors with RAS mutations, the
pathway is active regardless of whether EGFR is blocked. This is why
anti-EGFR therapy benefits only patients with RAS wild-type tumors. In
RAS-mutant disease, blocking EGFR has no effect because the downstream
signal is already permanently switched on (Goldberg \& Kirkpatrick,
2005). RAS mutation status is therefore a predictive biomarker: it
identifies patients who will not benefit from anti-EGFR therapy.

BRAF is another protein in the same signaling cascade, downstream of
RAS. About 10\% of mCRC patients have BRAF mutations, which
significantly worsens prognosis with standard chemotherapy. These
patients are now treated with a triplet combination of chemotherapy,
BRAF inhibitors and anti-EGFR therapy, following clinical trials
demonstrating improved outcomes compared to chemotherapy alone (Kopetz
et al., 2024).

VEGF is a growth factor that promotes angiogenesis. Anti-angiogenic
treatments target VEGF to prevent angiogenesis. The best known treatment
is bevacizumab, an antibody that binds VEGF and prevents it from
activating its receptor, thereby inhibiting angiogenesis. Unlike
anti-EGFR therapy, there is currently no validated biomarker to predict
which patients will benefit from bevacizumab. It is added to
chemotherapy regimens empirically, and while it modestly improves
survival in unselected populations, individual responses vary
considerably (Loupakis et al., 2014).

Primary tumor sidedness has emerged as an additional factor guiding
treatment selection. Left-sided tumors (descending colon, sigmoid,
rectum) have better prognosis and respond more favorably to anti-EGFR
therapy than right-sided tumors (cecum, ascending colon) (Cervantes et
al., 2023; Morris et al., 2023). The biological basis is not fully
understood but likely relates to differences in molecular subtypes and
embryological origin.

These criteria for treatment selection have been established through a
series of landmark clinical trials over the past two decades (Di
Nicolantonio et al., 2021). The result is that metastatic colorectal
cancer has become one of the success stories of precision oncology. We
now define patient subtypes based on RAS status, BRAF status, MSI
status, and primary tumor sidedness, and treatment is tailored
accordingly. Despite this progress, many patients with favorable
biomarker profiles do not respond as expected, and many with unfavorable
profiles outlive predictions. Current biomarkers guide treatment
selection but do not guarantee treatment success. We need better
biomarkers, both to select treatments more precisely and to assess
whether selected treatments are working. This is where imaging may have
a role to play.

\begin{figure}[htbp]
\centering
\includegraphics[width=0.95\textwidth]{fig_3_2.pdf}
\caption{Treatment landscape for colorectal liver metastases.
Simplified flowchart showing treatment decision-making based on
resectability and molecular profiling. Resectable disease proceeds to
surgery with or without perioperative chemotherapy. Unresectable disease
is treated with systemic therapy, with regimen selection guided by RAS,
BRAF, and MSI status. Anti-angiogenic therapy (bevacizumab) is added to
chemotherapy in appropriate patients. Adapted from Cervantes et al., 2023. Created with BioRender.com.}
\label{fig:3.2}
\end{figure}

\section{Response Assessment and Its Limitations}\label{response-assessment-and-its-limitations}

In routine clinical practice, treatment response assessment is primarily
based on two sources of information: how the patient feels (symptoms,
performance status, tolerance of treatment) and what imaging shows. For
solid tumors, the imaging assessment is dominated by size. CT{} is the
standard imaging modality for monitoring colorectal liver metastases.
Scans are acquired at baseline and at regular intervals during
treatment. Radiologists measure tumor dimensions, count lesions, and
look for new disease. In day-to-day clinical practice, response
assessment is often straightforward: if tumors are smaller, treatment is
working; if tumors are larger or new lesions appear, it is not.

For clinical trials, this assessment is formalized through the Response
Evaluation Criteria in Solid Tumors (RECIST). The original criteria were
published in 2000 (Therasse et al., 2000) and were updated to version
1.1 in 2009 (Eisenhauer et al., 2009) . RECIST provides standardized
definitions that enable comparison of treatment effects across studies.
Under RECIST 1.1, up to five target lesions (maximum two per organ) are
selected at baseline, each measuring at least 10 mm in longest diameter.
The sum of these diameters constitutes baseline tumor burden. At
subsequent assessments, percentage change determines the response
category out of four options (\hyperref[_Ref219662305]{Figure 3.3}).
Complete response (CR) requires disappearance of all target lesions.
Partial response (PR) requires at least 30\% decrease in sum of
diameters. Progressive disease (PD) is defined as at least 20\% increase
with a minimum 5 mm absolute increase, or the appearance of new lesions.
Stable disease (SD) encompasses everything in between.

RECIST has served clinical trials well by providing common language and
enabling multi-center studies. However, it has well-recognized
limitations (Burton, 2007; Nishino, 2018). By focusing exclusively on
size, RECIST creates a system where treatments that shrink tumors appear
successful and treatments that alter tumor biology without changing size
appear ineffective. RECIST was designed for a different era of oncology,
one dominated by cytotoxic chemotherapy where treatment either killed
cells and tumors shrank or it did not. As discussed in Section 3.3,
modern oncology includes targeted agents, immunotherapies, and
combination regimens that affect tumor composition in ways size alone
cannot capture.

The gaps found in treatment selection and response assessment motivate
the search for imaging biomarkers that capture heterogeneity (Gerwing et
al., 2019). The following chapter reviews how imaging heterogeneity has
been defined and measured, from qualitative visual assessment through
radiomics to habitat imaging, establishing the scientific context for
the methods developed in this thesis.

\begin{figure}[htbp]
\centering
\includegraphics[width=0.95\textwidth]{fig_3_3.pdf}
\caption{Response Evaluation Criteria in Solid Tumors (RECIST).
Illustration of RECIST 1.1 response categories based on percentage
change in sum of longest diameters from baseline. Complete response (CR)
requires disappearance of all target lesions. Partial response (PR)
requires $\geq$30\% decrease. Progressive disease (PD) is defined as $\geq$20\%
increase (with minimum 5 mm absolute increase) or appearance of new
lesions. Stable disease (SD) encompasses changes that do not meet
criteria for PR or PD. Created with BioRender.com.}
\label{fig:3.3}
\end{figure}
