\documentclass{article}
\usepackage[utf8]{inputenc}
\usepackage[margin=1in]{geometry} % Sets standard 1 inch margins

% --- REQUIRED PACKAGES FOR TABLES ---
\usepackage{booktabs}   % For professional horizontal rules
\usepackage{tabularx}   % For auto-sizing columns
\usepackage{longtable}  % For tables that span multiple pages
\usepackage{multirow}   % For cells spanning multiple rows
\usepackage{makecell}   % For line breaks inside cells
\usepackage{ragged2e}   % For better text alignment
\usepackage{caption}    % For better caption spacing
\usepackage{amsmath}    % For math symbols
\usepackage{textcomp}   % For symbols like textbullet

\title{Thesis Tables Conversion}
\date{\today}

\begin{document}

\maketitle
\listoftables
\newpage

% =================================================================
% TABLE 4.1
% =================================================================
\begin{table}[ht]
    \centering
    \small
    \caption{\textbf{Summary of selected habitat imaging studies.} Studies are listed chronologically. Technical validation refers to test-retest reproducibility or multi-site stability; biological validation refers to correlation with histopathology or known tissue biology; clinical validation refers to outcome prediction. GBM: glioblastoma; GMM: Gaussian mixture model.}
    \label{tab:habitat_studies}
    
    \begin{tabularx}{\textwidth}{@{} l >{\raggedright\arraybackslash}X c l c >{\raggedright\arraybackslash}X >{\raggedright\arraybackslash}X >{\raggedright\arraybackslash}X @{}}
        \toprule
        \textbf{Study} & \textbf{Cancer Type} & \textbf{Modality} & \textbf{Method} & \textbf{\textit{k}} & \textbf{Tech. Val.} & \textbf{Bio. Val.} & \textbf{Clin. Val.} \\
        \midrule
        (Carano et al., 2004) & Colorectal (preclinical) & MRI & K-means & 4 & No & Yes (histology) & No \\ \addlinespace
        (Henning et al., 2007) & Sarcoma (preclinical) & MRI & K-means & 4 & No & Yes (histology) & No \\ \addlinespace
        (Chaudhury et al., 2015) & Breast & MRI & K-means & 4 & No & No & Yes (molecular prediction) \\ \addlinespace
        (Divine et al., 2016) & Breast (preclinical) & PET/MRI & GMM & 5 & No & Yes (histology) & No \\ \addlinespace
        (Katiyar et al., 2017) & GBM (preclinical) & MRI & Spectral & 3 & No & Yes (histology) & No \\ \addlinespace
        (Juan-Albarracín et al., 2018) & GBM & MRI & K-means & 4 & No & Yes (vascular) & Yes (survival) \\ \addlinespace
        (Jardim-Perassi et al., 2019) & Breast (preclinical) & MRI & GMM & 4 & Yes (repeatability) & Yes (histology) & No \\ \addlinespace
        (Franklin et al., 2020) & Colorectal Liver Metastases & MRI & K-means & 2 & No & Yes (histology) & No \\ \addlinespace
        (Katiyar et al., 2023) & Colorectal Liver Metastases & PET/MRI & Spectral & 2 & No & Yes (histology) & No \\
        \bottomrule
    \end{tabularx}
\end{table}

\newpage

% =================================================================
% TABLE 5.1
% =================================================================
\begin{table}[ht]
    \centering
    \small
    \caption{\textbf{Summary of datasets and their role in this thesis.} PREDICT provides co-registered mpMRI for biologically-informed habitat model development; POEM provides histology for independent validation. TCIA and VHIO provide large-scale CT imaging with survival outcomes for applying the habitat model and testing its prognostic value. OS: overall survival; PFS: progression-free survival; DFS: disease-free survival; mpMRI: multiparametric MRI; HE: hematoxylin and eosin.}
    \label{tab:datasets}
    
    \begin{tabularx}{\textwidth}{@{} l c c >{\raggedright\arraybackslash}X c >{\raggedright\arraybackslash}X @{}}
        \toprule
        \textbf{Dataset} & \textbf{\makecell{N\\patients}} & \textbf{\makecell{N\\tumors}} & \textbf{Imaging modalities} & \textbf{\makecell{Clinical\\outcome}} & \textbf{Primary use} \\
        \midrule
        PREDICT & 10 & 42 & \textbullet~CT (Portal Phase) \newline \textbullet~MRI (anatomical and quantitative) & --- & Habitat model development (mpMRI-anchored) and validation (Chapter 7) \\ \addlinespace
        POEM & 6 & 6 & \textbullet~CT (Portal Phase) \newline \textbullet~Histology (HE-stained whole resected tumor) & --- & Biological validation of habitat model (Chapter 7) \\ \addlinespace
        TCIA & 189 & 389 & \textbullet~CT (Portal Phase) & OS, DFS & Application of habitat model and clinical outcome modeling (Chapter 8) \\ \addlinespace
        VHIO & 343 & 1759 & \textbullet~CT (Portal Phase) & OS, PFS & Application of habitat model and clinical outcome modeling (Chapter 8) \\
        \bottomrule
    \end{tabularx}
\end{table}

\newpage

% =================================================================
% TABLE 5.2
% =================================================================
\begin{table}[ht]
    \centering
    \small
    \caption{\textbf{Multiparametric MRI maps used for CT habitat development.} Thirteen quantitative maps were derived from diffusion-relaxation MRI, variable flip angle T1 mapping, and dynamic contrast-enhanced MRI.}
    \label{tab:mpmri_maps}
    
    \begin{tabularx}{\textwidth}{@{} l l l c >{\raggedright\arraybackslash}X @{}}
        \toprule
        \multicolumn{2}{c}{\textbf{mpMRI metric}} & \textbf{Computed from} & \textbf{Units} & \textbf{Description} \\
        \cmidrule(r){1-2}
        $ADC_t$ & Tissue ADC & Diffusion-relaxation MRI & $\mu m^2/ms$ & Apparent diffusivity of water in tissue (excluding vasculature) \\ \addlinespace
        $ADC_v$ & Vascular ADC & Diffusion-relaxation MRI & $\mu m^2/ms$ & Apparent diffusivity in vascular compartment (IVIM effect) \\ \addlinespace
        $K_t$ & Tissue kurtosis excess & Diffusion-relaxation MRI & Dim.less & Quantifies non-Gaussian diffusion due to heterogeneity or restriction \\ \addlinespace
        $f_v$ & Vascular signal fraction & Diffusion-relaxation MRI & Norm. & Signal arising from vascular compartment (IVIM) \\ \addlinespace
        $T_{2t}$ & Tissue $T_2$ & Diffusion-relaxation MRI & ms & Transverse relaxation time of tissue (excluding vasculature) \\ \addlinespace
        $D_0$ & Intrinsic diffusivity & Advanced diffusion model & $\mu m^2/ms$ & Intrinsic intracellular diffusivity \\ \addlinespace
        $vCS$ & Vol-weighted cell size & Advanced diffusion model & $\mu m$ & Average cell size weighted by cell volume \\ \addlinespace
        $f_{in}$ & Intracellular fraction & Advanced diffusion model & Norm. & Fraction of signal arising from intracellular water \\ \addlinespace
        $CD$ & Cell density & Advanced diffusion model & Cells/$mm^3$ & Estimated cell density (derived from $f_{in}$ and $vCS$) \\ \addlinespace
        $T_1$ & $T_1$ & Variable flip angle SGrE & ms & Total longitudinal relaxation time \\ \addlinespace
        $T_2^*$ & $T_2^*$ & Multiecho SGrE & ms & Effective transverse relaxation time \\ \addlinespace
        $K^{trans}$ & Capillary permeability & DCE MRI & $min^{-1}$ & Influx mass transfer rate of contrast agent \\ \addlinespace
        $v_e$ & EES volume & DCE MRI & Norm. & Volume of extracellular, extravascular space per unit tissue volume \\
        \bottomrule
    \end{tabularx}
\end{table}

\newpage

% =================================================================
% TABLE 6.1
% =================================================================
\begin{table}[ht]
    \centering
    \footnotesize
    \setlength{\tabcolsep}{3pt}
    \caption{\textbf{Reproducibility against kernel radius and bin size.} Median (IQR) lower confidence limit of the Intraclass Correlation Coefficient reported. B = bin size, HU = Hounsfield units, R = kernel radius.}
    \label{tab:reproducibility}
    \begin{tabular}{l c c c c c c c c}
        \toprule
        & \multicolumn{4}{c}{\textbf{Reproducibility against R}} & \multicolumn{4}{c}{\textbf{Reproducibility against B}} \\
        \cmidrule(lr){2-5} \cmidrule(lr){6-9}
        & \multicolumn{2}{c}{Fixed B = 12 HU} & \multicolumn{2}{c}{Fixed B = 25 HU} & \multicolumn{2}{c}{Fixed R = 1 mm} & \multicolumn{2}{c}{Fixed R = 3 mm} \\
        \cmidrule(lr){2-3} \cmidrule(lr){4-5} \cmidrule(lr){6-7} \cmidrule(lr){8-9}
        \textbf{Lesion type} & \textbf{Liver} & \textbf{Lung} & \textbf{Liver} & \textbf{Lung} & \textbf{Liver} & \textbf{Lung} & \textbf{Liver} & \textbf{Lung} \\
        \midrule
        LCL [Median(IQR)] & 0.422 & 0.573 & 0.407 & 0.573 & 0.805 & 0.929 & 0.921 & 0.967 \\
         & \scriptsize(0.346-0.513) & \scriptsize(0.403-0.701) & \scriptsize(0.291-0.536) & \scriptsize(0.443-0.696) & \scriptsize(0.672-0.919) & \scriptsize(0.823-0.997) & \scriptsize(0.821-0.982) & \scriptsize(0.93-0.999) \\
        \bottomrule
    \end{tabular}
\end{table}

\newpage

% =================================================================
% TABLE 6.2
% =================================================================
\begin{longtable}{l l}
    \caption{\textbf{Precise 3D Radiomics Features in Liver and Lung Lesions.} A precise radiomic feature was defined as lower confidence limit $\ge$ 0.50. FO = first-order; GLCM = Grey Level Co-occurrence Matrix; GLDM = Grey Level Dependence Matrix; GLRLM = Grey Level Run Length Matrix; GLSZM = Grey Level Size Zone Matrix; NGTDM = Neighboring Grey Tone Difference Matrix.}
    \label{tab:radiomics_features} \\
    \toprule
    \textbf{Liver lesions} & \textbf{Lung lesions} \\
    \midrule
    \endfirsthead
    \caption[]{Precise 3D Radiomics Features (continued)} \\
    \toprule
    \textbf{Liver lesions} & \textbf{Lung lesions} \\
    \midrule
    \endhead
    \bottomrule
    \multicolumn{2}{r}{\textit{Continued on next page...}} \\
    \endfoot
    \bottomrule
    \endlastfoot
    FO\_10Percentile & FO\_90Percentile \\
    FO\_90Percentile & GLCM\_Id \\
    FO\_Energy & GLCM\_Idm \\
    FO\_Mean & GLCM\_Imc1 \\
    FO\_Median & GLCM\_InverseVariance \\
    FO\_Minimum & GLCM\_JointEntropy \\
    FO\_RootMeanSquared & GLDM\_DependenceEntropy \\
    GLCM\_Autocorrelation & GLDM\_DependenceNonUniformityNorm. \\
    GLCM\_JointAverage & GLDM\_GrayLevelNonUniformity \\
    GLCM\_SumAverage & GLDM\_LargeDependenceEmphasis \\
    GLDM\_DependenceEntropy & GLDM\_LargeDependenceHighGrayLevelEmphasis \\
    GLDM\_GrayLevelNonUniformity & GLDM\_SmallDependenceEmphasis \\
    GLDM\_HighGrayLevelEmphasis & GLDM\_SmallDependenceHighGrayLevelEmphasis \\
    GLDM\_LargeDependenceLowGrayLevelEmphasis & GLRLM\_GrayLevelNonUniformity \\
    GLDM\_LowGrayLevelEmphasis & GLRLM\_LongRunEmphasis \\
    GLDM\_SmallDependenceHighGrayLevelEmphasis & GLRLM\_LongRunHighGrayLevelEmphasis \\
    GLRLM\_GrayLevelNonUniformity & GLRLM\_RunLengthNonUniformity \\
    GLRLM\_HighGrayLevelRunEmphasis & GLRLM\_RunLengthNonUniformityNorm. \\
    GLRLM\_LongRunHighGrayLevelEmphasis & GLRLM\_RunPercentage \\
    GLRLM\_LongRunLowGrayLevelEmphasis & GLRLM\_RunVariance \\
    GLRLM\_LowGrayLevelRunEmphasis & GLRLM\_ShortRunEmphasis \\
    GLRLM\_RunLengthNonUniformity & GLSZM\_LargeAreaEmphasis \\
    GLRLM\_RunPercentage & GLSZM\_LargeAreaHighGrayLevelEmphasis \\
    GLRLM\_RunVariance & GLSZM\_ZonePercentage \\
    GLRLM\_ShortRunHighGrayLevelEmphasis & GLSZM\_ZoneVariance \\
    NGTDM\_Coarseness & NGTDM\_Coarseness \\
\end{longtable}

\newpage

% =================================================================
% TABLE 7.1
% =================================================================
\begin{table}[ht]
    \centering
    \small
    \setlength{\tabcolsep}{4pt}
    \caption{\textbf{Comparison of candidate CT feature representations.} *p < 0.05 after Bonferroni correction. W = Kendall's W effect size.}
    \label{tab:ct_feature_comparison}
    \begin{tabular}{l c c c c c c c}
        \toprule
        & \multicolumn{2}{c}{\textbf{ADCt}} & \multicolumn{2}{c}{\textbf{Vascular Fraction}} & \multicolumn{2}{c}{\textbf{Ktrans}} & \multirow{2}{*}{\textbf{\makecell{Biophysical\\Score}}} \\
        \cmidrule(lr){2-3} \cmidrule(lr){4-5} \cmidrule(lr){6-7}
        & W & P-value & W & P-value & W & P-value & \\
        \midrule
        Raw HU & 0.31 & 0.2928 & 0.12 & 0.4894 & 0.04 & 0.6703 & 0.1567 \\ \addlinespace
        Handcrafted & 0.16 & 0.3281 & 0.67 & 0.0053* & 0.52 & 0.0055* & 0.4500 \\ \addlinespace
        DL-SALSA & 0.1481 & 0.3808 & 0.4815 & 0.0853 & 0.3457 & 0.0446* & 0.3251 \\ \addlinespace
        DL-FM & 0.13 & 0.4429 & 0.19 & 0.3889 & 0.13 & 0.2725 & 0.1500 \\
        \bottomrule
    \end{tabular}
\end{table}

\newpage

% =================================================================
% TABLE 7.2
% =================================================================
\begin{table}[ht]
    \centering
    \small
    \setlength{\tabcolsep}{3pt}
    \caption{\textbf{CT Habitat Characterization with mpMRI metrics.} Patient-level median values for 13 mpMRI-derived biophysical parameters across the three CT habitats. W = Kendall's W. *p < 0.05, **p < 0.01.}
    \label{tab:habitat_characterization}
    \begin{tabular}{l l c c c c c c}
        \toprule
        \multicolumn{2}{c}{\textbf{mpMRI metric}} & \textbf{Units} & \multicolumn{3}{c}{\textbf{Habitats (medians)}} & \textbf{\makecell{p-value\\(BH)}} & \textbf{\makecell{Effect\\Size (W)}} \\
        \cmidrule(r){1-2} \cmidrule(lr){4-6}
        & & & \textbf{H1} & \textbf{H2} & \textbf{H3} & & \\
        \midrule
        $ADC_t$ & Tissue ADC & $\mu m^2/ms$ & 1.45 & 1.30 & 1.36 & 0.328 & 0.16 \\ \addlinespace
        $ADC_v$ & Vascular ADC & $\mu m^2/ms$ & 16.6 & 16.8 & 17.6 & 0.044* & 0.39 \\ \addlinespace
        $K_t$ & Tissue kurtosis excess & Dim.less & 0.75 & 0.69 & 0.70 & 0.587 & 0.07 \\ \addlinespace
        $f_v$ & Vasc. signal fraction & Norm. & 0.068 & 0.081 & 0.112 & 0.005** & 0.67 \\ \addlinespace
        $T_{2t}$ & Tissue $T_2$ & ms & 81.4 & 100.8 & 82.9 & 0.726 & 0.04 \\ \addlinespace
        $D_0$ & Intrinsic diffusivity & $\mu m^2/ms$ & 37.5 & 54.1 & 71.6 & 0.001** & 0.84 \\ \addlinespace
        $vCS$ & Vol-weighted cell size & $\mu m$ & 24.0 & 23.6 & 23.9 & 0.529 & 0.09 \\ \addlinespace
        $f_{in}$ & Intracellular fraction & Norm. & 0.46 & 0.50 & 0.48 & 0.113 & 0.28 \\ \addlinespace
        $CD$ & Cell density & \makecell{$\times 10^3$\\Cells/$mm^3$} & 51.2 & 56.8 & 53.6 & 0.529 & 0.09 \\ \addlinespace
        $T_1$ & $T_1$ & ms & 1063 & 945 & 911 & 0.019* & 0.49 \\ \addlinespace
        $T_2^*$ & $T_2^*$ & ms & 27.5 & 26.5 & 23.7 & 0.001** & 0.84 \\ \addlinespace
        $K^{trans}$ & Capillary permeability & $min^{-1}$ & 0.41 & 0.52 & 0.51 & 0.018* & 0.52 \\ \addlinespace
        $v_e$ & EES volume & Norm. & 0.66 & 0.73 & 0.74 & 0.741 & 0.03 \\
        \bottomrule
    \end{tabular}
\end{table}

\newpage

% =================================================================
% TABLE 8.1
% =================================================================
\begin{longtable}{l c c}
    \caption{\textbf{Clinical characteristics of the TCIA and VHIO cohorts.}}
    \label{tab:clinical_characteristics} \\
    \toprule
    \textbf{Variables} & \textbf{\makecell{TCIA\\(n=189)}} & \textbf{\makecell{VHIO\\(n=344)}} \\
    \midrule
    \endfirsthead
    \caption[]{Clinical characteristics (continued)} \\
    \toprule
    \textbf{Variables} & \textbf{\makecell{TCIA\\(n=189)}} & \textbf{\makecell{VHIO\\(n=344)}} \\
    \midrule
    \endhead
    \bottomrule
    \endlastfoot
    \textbf{Age} [years, Median(range)] & 61 (30 -- 88) & 69 (32 -- 88) \\ \addlinespace
    \textbf{Sex} [n (\%)] & & \\
    \quad Male & 111 (58.7) & 205 (59.6) \\
    \quad Female & 78 (41.3) & 139 (40.4) \\ \addlinespace
    \textbf{Primary Tumor Location} [n (\%)] & & \\
    \quad Right & & 139 (40.4) \\
    \quad Left & & 183 (53.2) \\
    \quad Rectum & & 17 (4.9) \\
    \quad Unknown & 189 (100) & 5 (1.5) \\ \addlinespace
    \textbf{RAS Status} [n (\%)] & & \\
    \quad Wild-type & & 136 (39.5\%) \\
    \quad Mutant & & 195 (56.7\%) \\
    \quad Unknown & & 13 (3.8\%) \\ \addlinespace
    \textbf{Synchronous CRLM} [n(\%)] & 104 (55.0) & 277 (80.5) \\ \addlinespace
    \textbf{Extrahepatic Disease} [n(\%)] & 15 (7.9) & 203 (59.0) \\ \addlinespace
    \textbf{No. of liver metastases per patient} & 2.0 (1.0, 3.0) & 3.0 (2.0, 7.0) \\
    \quad [Median (IQR)] & & \\ \addlinespace
    \textbf{Median liver metastasis size} & 3.2 (1.2, 10.9) & 5.1 (2.0, 15.0) \\
    \quad [cm$^3$, Median (IQR)] & & \\ \addlinespace
    \textbf{Liver Disease Volume} & 10.7 (4.1, 32.8) & 32.4 (6.8, 180.3) \\
    \quad [cm$^3$, Median (IQR)] & & \\ \addlinespace
    \textbf{First-line Treatment Type} [n (\%)] & -- & \\
    \quad Chemotherapy Only & & 122 (35.5) \\
    \quad Chemotherapy + Antiangiogenic & & 133 (38.7) \\
    \quad Chemotherapy + Targeted & & 69 (20.1) \\
    \quad Other & & 20 (5.8\%) \\ \addlinespace
    \textbf{Neoadjuvant chemotherapy} [n (\%)] & 115 (60.8) & -- \\ \addlinespace
    \textbf{Progression Free Survival} & -- & 8.9 (5.0, 15.3) \\
    \quad [months, Median (IQR)] & & \\ \addlinespace
    \textbf{Overall Survival} & 67.1 (34.4, 97.5) & 19.1 (11.1, 32.8) \\
    \quad [months, Median (IQR)] & & \\ \addlinespace
    \textbf{Disease Free Survival} & 22.3 (9.4, 69.3) & 8.9 (5.0, 15.3) \\
    \quad [months, Median (IQR)] & & \\
\end{longtable}

\newpage

% =================================================================
% TABLE 8.2
% =================================================================
\begin{table}[ht]
    \centering
    \small
    \caption{\textbf{Cox regression for overall survival in neoadjuvant-treated TCIA patients.} Multivariable model includes variables with univariable p$<$0.10. Dashes indicate variables not entered. Bold indicates p$<$0.05.}
    \label{tab:cox_regression}
    \begin{tabular}{l c c c c}
        \toprule
        & \multicolumn{2}{c}{\textbf{Univariable}} & \multicolumn{2}{c}{\textbf{Multivariable}} \\
        \cmidrule(lr){2-3} \cmidrule(lr){4-5}
        \textbf{Variable} & \textbf{HR [95\% CI]} & \textbf{P-value} & \textbf{HR [95\% CI]} & \textbf{P-value} \\
        \midrule
        \textbf{Clinical} & & & & \\
        \quad Extrahepatic disease & 2.61 [1.28--5.30] & \textbf{0.008} & 2.64 [0.96--7.27] & 0.059 \\
        \quad Synchronous metastases & 0.76 [0.46--1.25] & 0.280 & --- & --- \\ \addlinespace
        \textbf{Tumor volume} & & & & \\
        \quad Tumor volume & 2.01 [1.31--3.07] & \textbf{0.001} & 1.90 [0.75--4.80] & 0.173 \\ \addlinespace
        \textbf{Habitats-Whole Tumor} & & & & \\
        \quad Whole entropy & 0.62 [0.47--0.83] & \textbf{0.001} & 0.73 [0.32--1.66] & 0.455 \\
        \quad Whole Avasc. prop. & 0.92 [0.69--1.23] & 0.585 & --- & --- \\
        \quad Whole Cell.-Perf. prop. & 1.49 [1.10--2.01] & \textbf{0.010} & 1.66 [1.04--2.65] & \textbf{0.033} \\ \addlinespace
        \textbf{Habitats-Rim} & & & & \\
        \quad Rim entropy & 0.59 [0.45--0.78] & \textbf{$<$0.001} & 0.90 [0.60--1.35] & 0.622 \\
        \quad Rim Avasc. prop. & 0.60 [0.45--0.80] & \textbf{$<$0.001} & 0.60 [0.29--1.24] & 0.168 \\
        \quad Rim Cell.-Perf. prop. & 1.65 [1.21--2.23] & \textbf{0.001} & 0.59 [0.26--1.33] & 0.201 \\ \addlinespace
        \textbf{Habitats-Core} & & & & \\
        \quad Core entropy & 1.22 [0.95--1.57] & 0.121 & --- & --- \\
        \quad Core Avasc. prop. & 1.11 [0.88--1.41] & 0.370 & --- & --- \\
        \quad Core Cell.-Perf. prop. & 0.87 [0.69--1.09] & 0.224 & --- & --- \\ \midrule
        \textbf{Model C-index} & & & \multicolumn{2}{c}{\textbf{0.699}} \\
        \bottomrule
    \end{tabular}
\end{table}

\newpage

% =================================================================
% TABLE 8.3
% =================================================================
\begin{table}[ht]
    \centering
    \small
    \setlength{\tabcolsep}{3pt}
    \caption{\textbf{Context-dependent prognostic value of habitat metrics in VHIO.} Multivariable hazard ratios adjusted for clinical covariates (extrahepatic disease, primary site, age, synchronous metastases where p$<$0.10). Bold indicates p$<$0.05. Rim entropy is independently prognostic in bevacizumab-treated and RAS-mutant patients—contexts where anti-angiogenic therapy is relevant.}
    \label{tab:context_dependent_prognostic}
    \begin{tabular}{l c c c c c c c}
        \toprule
        & & & \multicolumn{2}{c}{\textbf{Volume}} & \multicolumn{2}{c}{\textbf{Rim Entropy}} & \\
        \cmidrule(lr){4-5} \cmidrule(lr){6-7}
        & \textbf{n} & \textbf{Events} & \textbf{HR [95\% CI]} & \textbf{p} & \textbf{HR [95\% CI]} & \textbf{p} & \textbf{C-index} \\
        \midrule
        \textbf{All patients} & 344 & 246 & \textbf{1.29 [1.08--1.55]} & \textbf{0.006} & 0.88 [0.76--1.02] & 0.10 & 0.665 \\ \addlinespace
        \textbf{By Treatment} & & & & & & & \\
        \quad Chemo alone & 122 & -- & \textbf{5.46 [3.19--9.34]} & \textbf{$<$0.001} & 0.88 [0.54--1.43] & 0.61 & 0.703 \\
        \quad Chemo + Bevacizumab & 133 & -- & \textbf{1.69 [1.31--2.17]} & \textbf{$<$0.001} & \textbf{0.68 [0.52--0.88]} & \textbf{0.004} & 0.674 \\
        \quad Chemo + Targeted & 69 & -- & \textbf{1.21 [1.01--1.47]} & \textbf{0.043} & 1.14 [0.81--1.60] & 0.46 & 0.723 \\ \addlinespace
        \textbf{By RAS Status} & & & & & & & \\
        \quad RAS Wild-Type & 136 & -- & \textbf{1.23 [1.02--1.48]} & \textbf{0.028} & 0.96 [0.77--1.21] & 0.76 & 0.705 \\
        \quad RAS Mutant & 195 & -- & \textbf{1.89 [1.41--2.53]} & \textbf{$<$0.001} & \textbf{0.80 [0.65--1.00]} & \textbf{0.047} & 0.700 \\
        \bottomrule
    \end{tabular}
\end{table}

\end{document}